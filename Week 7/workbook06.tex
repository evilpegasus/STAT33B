% Options for packages loaded elsewhere
\PassOptionsToPackage{unicode}{hyperref}
\PassOptionsToPackage{hyphens}{url}
%
\documentclass[
]{article}
\usepackage{lmodern}
\usepackage{amssymb,amsmath}
\usepackage{ifxetex,ifluatex}
\ifnum 0\ifxetex 1\fi\ifluatex 1\fi=0 % if pdftex
  \usepackage[T1]{fontenc}
  \usepackage[utf8]{inputenc}
  \usepackage{textcomp} % provide euro and other symbols
\else % if luatex or xetex
  \usepackage{unicode-math}
  \defaultfontfeatures{Scale=MatchLowercase}
  \defaultfontfeatures[\rmfamily]{Ligatures=TeX,Scale=1}
\fi
% Use upquote if available, for straight quotes in verbatim environments
\IfFileExists{upquote.sty}{\usepackage{upquote}}{}
\IfFileExists{microtype.sty}{% use microtype if available
  \usepackage[]{microtype}
  \UseMicrotypeSet[protrusion]{basicmath} % disable protrusion for tt fonts
}{}
\makeatletter
\@ifundefined{KOMAClassName}{% if non-KOMA class
  \IfFileExists{parskip.sty}{%
    \usepackage{parskip}
  }{% else
    \setlength{\parindent}{0pt}
    \setlength{\parskip}{6pt plus 2pt minus 1pt}}
}{% if KOMA class
  \KOMAoptions{parskip=half}}
\makeatother
\usepackage{xcolor}
\IfFileExists{xurl.sty}{\usepackage{xurl}}{} % add URL line breaks if available
\IfFileExists{bookmark.sty}{\usepackage{bookmark}}{\usepackage{hyperref}}
\hypersetup{
  pdftitle={STAT 33B Workbook 6},
  pdfauthor={Ming Fong (3035619833)},
  hidelinks,
  pdfcreator={LaTeX via pandoc}}
\urlstyle{same} % disable monospaced font for URLs
\usepackage[margin=1in]{geometry}
\usepackage{color}
\usepackage{fancyvrb}
\newcommand{\VerbBar}{|}
\newcommand{\VERB}{\Verb[commandchars=\\\{\}]}
\DefineVerbatimEnvironment{Highlighting}{Verbatim}{commandchars=\\\{\}}
% Add ',fontsize=\small' for more characters per line
\usepackage{framed}
\definecolor{shadecolor}{RGB}{248,248,248}
\newenvironment{Shaded}{\begin{snugshade}}{\end{snugshade}}
\newcommand{\AlertTok}[1]{\textcolor[rgb]{0.94,0.16,0.16}{#1}}
\newcommand{\AnnotationTok}[1]{\textcolor[rgb]{0.56,0.35,0.01}{\textbf{\textit{#1}}}}
\newcommand{\AttributeTok}[1]{\textcolor[rgb]{0.77,0.63,0.00}{#1}}
\newcommand{\BaseNTok}[1]{\textcolor[rgb]{0.00,0.00,0.81}{#1}}
\newcommand{\BuiltInTok}[1]{#1}
\newcommand{\CharTok}[1]{\textcolor[rgb]{0.31,0.60,0.02}{#1}}
\newcommand{\CommentTok}[1]{\textcolor[rgb]{0.56,0.35,0.01}{\textit{#1}}}
\newcommand{\CommentVarTok}[1]{\textcolor[rgb]{0.56,0.35,0.01}{\textbf{\textit{#1}}}}
\newcommand{\ConstantTok}[1]{\textcolor[rgb]{0.00,0.00,0.00}{#1}}
\newcommand{\ControlFlowTok}[1]{\textcolor[rgb]{0.13,0.29,0.53}{\textbf{#1}}}
\newcommand{\DataTypeTok}[1]{\textcolor[rgb]{0.13,0.29,0.53}{#1}}
\newcommand{\DecValTok}[1]{\textcolor[rgb]{0.00,0.00,0.81}{#1}}
\newcommand{\DocumentationTok}[1]{\textcolor[rgb]{0.56,0.35,0.01}{\textbf{\textit{#1}}}}
\newcommand{\ErrorTok}[1]{\textcolor[rgb]{0.64,0.00,0.00}{\textbf{#1}}}
\newcommand{\ExtensionTok}[1]{#1}
\newcommand{\FloatTok}[1]{\textcolor[rgb]{0.00,0.00,0.81}{#1}}
\newcommand{\FunctionTok}[1]{\textcolor[rgb]{0.00,0.00,0.00}{#1}}
\newcommand{\ImportTok}[1]{#1}
\newcommand{\InformationTok}[1]{\textcolor[rgb]{0.56,0.35,0.01}{\textbf{\textit{#1}}}}
\newcommand{\KeywordTok}[1]{\textcolor[rgb]{0.13,0.29,0.53}{\textbf{#1}}}
\newcommand{\NormalTok}[1]{#1}
\newcommand{\OperatorTok}[1]{\textcolor[rgb]{0.81,0.36,0.00}{\textbf{#1}}}
\newcommand{\OtherTok}[1]{\textcolor[rgb]{0.56,0.35,0.01}{#1}}
\newcommand{\PreprocessorTok}[1]{\textcolor[rgb]{0.56,0.35,0.01}{\textit{#1}}}
\newcommand{\RegionMarkerTok}[1]{#1}
\newcommand{\SpecialCharTok}[1]{\textcolor[rgb]{0.00,0.00,0.00}{#1}}
\newcommand{\SpecialStringTok}[1]{\textcolor[rgb]{0.31,0.60,0.02}{#1}}
\newcommand{\StringTok}[1]{\textcolor[rgb]{0.31,0.60,0.02}{#1}}
\newcommand{\VariableTok}[1]{\textcolor[rgb]{0.00,0.00,0.00}{#1}}
\newcommand{\VerbatimStringTok}[1]{\textcolor[rgb]{0.31,0.60,0.02}{#1}}
\newcommand{\WarningTok}[1]{\textcolor[rgb]{0.56,0.35,0.01}{\textbf{\textit{#1}}}}
\usepackage{graphicx}
\makeatletter
\def\maxwidth{\ifdim\Gin@nat@width>\linewidth\linewidth\else\Gin@nat@width\fi}
\def\maxheight{\ifdim\Gin@nat@height>\textheight\textheight\else\Gin@nat@height\fi}
\makeatother
% Scale images if necessary, so that they will not overflow the page
% margins by default, and it is still possible to overwrite the defaults
% using explicit options in \includegraphics[width, height, ...]{}
\setkeys{Gin}{width=\maxwidth,height=\maxheight,keepaspectratio}
% Set default figure placement to htbp
\makeatletter
\def\fps@figure{htbp}
\makeatother
\setlength{\emergencystretch}{3em} % prevent overfull lines
\providecommand{\tightlist}{%
  \setlength{\itemsep}{0pt}\setlength{\parskip}{0pt}}
\setcounter{secnumdepth}{-\maxdimen} % remove section numbering
\ifluatex
  \usepackage{selnolig}  % disable illegal ligatures
\fi

\title{STAT 33B Workbook 6}
\author{Ming Fong (3035619833)}
\date{Oct 8, 2020}

\begin{document}
\maketitle

This workbook is due \textbf{Oct 8, 2020} by 11:59pm PT.

The workbook is organized into sections that correspond to the lecture
videos for the week. Watch a video, then do the corresponding exercises
\emph{before} moving on to the next video.

Workbooks are graded for completeness, so as long as you make a clear
effort to solve each problem, you'll get full credit. That said, make
sure you understand the concepts here, because they're likely to
reappear in homeworks, quizzes, and later lectures.

As you work, write your answers in this notebook. Answer questions with
complete sentences, and put code in code chunks. You can make as many
new code chunks as you like.

In the notebook, you can run the line of code where the cursor is by
pressing \texttt{Ctrl} + \texttt{Enter} on Windows or \texttt{Cmd} +
\texttt{Enter} on Mac OS X. You can run an entire code chunk by clicking
on the green arrow in the upper right corner of the code chunk.

Please do not delete the exercises already in this notebook, because it
may interfere with our grading tools.

You need to submit your work in two places:

\begin{itemize}
\tightlist
\item
  Submit this Rmd file with your edits on bCourses.
\item
  Knit and submit the generated PDF file on Gradescope.
\end{itemize}

If you have any last-minute trouble knitting, \textbf{DON'T PANIC}.
Submit your Rmd file on time and follow up in office hours or on Piazza
to sort out the PDF.

\hypertarget{functions}{%
\section{Functions}\label{functions}}

Watch the ``Functions'' lecture video.

No exercises for this section.

\hypertarget{default-arguments}{%
\section{Default Arguments}\label{default-arguments}}

Watch the ``Default Arguments'' lecture video.

\hypertarget{exercise-1}{%
\subsection{Exercise 1}\label{exercise-1}}

Write a function \texttt{to\_kelvin()} to convert temperatures in
degrees Celsius to temperatures in degrees Kelvin (you can find the
formula online). Your function should have a parameter \texttt{celsius}
for the temperature to convert.

Make sure that your function is vectorized in \texttt{celsius}, so that
it can convert an entire vector of temperatures in one call.

Show that your function works correctly for a few test cases.

\textbf{YOUR ANSWER GOES HERE:}

\begin{Shaded}
\begin{Highlighting}[]
\NormalTok{to\_kelvin =}\StringTok{ }\ControlFlowTok{function}\NormalTok{(celsius) \{}
\NormalTok{    celsius }\OperatorTok{+}\StringTok{ }\FloatTok{273.15}
\NormalTok{\}}
\KeywordTok{to\_kelvin}\NormalTok{(}\KeywordTok{c}\NormalTok{(}\DecValTok{15}\NormalTok{, }\DecValTok{0}\NormalTok{, }\DecValTok{1}\NormalTok{, }\DecValTok{100}\NormalTok{))}
\end{Highlighting}
\end{Shaded}

\begin{verbatim}
## [1] 288.15 273.15 274.15 373.15
\end{verbatim}

\begin{Shaded}
\begin{Highlighting}[]
\KeywordTok{to\_kelvin}\NormalTok{(}\DecValTok{1234}\NormalTok{)}
\end{Highlighting}
\end{Shaded}

\begin{verbatim}
## [1] 1507.15
\end{verbatim}

\hypertarget{function-example}{%
\section{Function Example}\label{function-example}}

The ``Function Example'' lecture video is optional. It provides a more
realistic example of writing a function.

No exercises for this section.

\hypertarget{variable-scope-lookup}{%
\section{Variable Scope \& Lookup}\label{variable-scope-lookup}}

Watch the ``Variable Scope \& Lookup'' lecture video.

No exercises for this section.

\hypertarget{lazy-evaluation}{%
\section{Lazy Evaluation}\label{lazy-evaluation}}

Watch the ``Lazy Evaluation'' lecture video.

\hypertarget{exercise-2}{%
\subsection{Exercise 2}\label{exercise-2}}

Extend your \texttt{to\_kelvin()} function to be able to convert degrees
Fahrenheit to degrees Kelvin as well. Add a parameter
\texttt{fahrenheit} for the Fahrenheit temperature to convert.

Leave the code in the body of your function unchanged. Instead, provide
a default argument for \texttt{celsius} that converts the
\texttt{fahrenheit} argument.

Show that your function works correctly for a few test cases (for both
Celsius and Fahrenheit temperatures).

What happens if you call your function with arguments to the
\texttt{celsius} parameter and the \texttt{fahrenheit} parameter at the
same time?

\textbf{YOUR ANSWER GOES HERE:}

Passing both the \texttt{celsius} and \texttt{fahrenheit} parameters
will only use the \texttt{celsius} parameter. This is because fahrenheit
only works when it overrides the \texttt{celsius} default value.

\begin{Shaded}
\begin{Highlighting}[]
\NormalTok{to\_kelvin =}\StringTok{ }\ControlFlowTok{function}\NormalTok{(}\DataTypeTok{celsius =}\NormalTok{ (fahrenheit }\OperatorTok{{-}}\StringTok{ }\DecValTok{32}\NormalTok{) }\OperatorTok{*}\StringTok{ }\DecValTok{5} \OperatorTok{/}\StringTok{ }\DecValTok{9}\NormalTok{, fahrenheit) \{}
\NormalTok{    celsius }\OperatorTok{+}\StringTok{ }\FloatTok{273.15}
\NormalTok{\}}
\KeywordTok{to\_kelvin}\NormalTok{(}\DataTypeTok{celsius =} \DecValTok{100}\NormalTok{)}
\end{Highlighting}
\end{Shaded}

\begin{verbatim}
## [1] 373.15
\end{verbatim}

\begin{Shaded}
\begin{Highlighting}[]
\KeywordTok{to\_kelvin}\NormalTok{(}\DataTypeTok{fahrenheit =} \DecValTok{32}\NormalTok{)}
\end{Highlighting}
\end{Shaded}

\begin{verbatim}
## [1] 273.15
\end{verbatim}

\begin{Shaded}
\begin{Highlighting}[]
\KeywordTok{to\_kelvin}\NormalTok{(}\DataTypeTok{fahrenheit =} \KeywordTok{c}\NormalTok{(}\DecValTok{14}\NormalTok{, }\DecValTok{32}\NormalTok{, }\DecValTok{100}\NormalTok{))}
\end{Highlighting}
\end{Shaded}

\begin{verbatim}
## [1] 263.1500 273.1500 310.9278
\end{verbatim}

\begin{Shaded}
\begin{Highlighting}[]
\KeywordTok{to\_kelvin}\NormalTok{(}\DataTypeTok{celsius =} \DecValTok{0}\NormalTok{, }\DataTypeTok{fahrenheit =} \KeywordTok{c}\NormalTok{(}\DecValTok{14}\NormalTok{, }\DecValTok{32}\NormalTok{, }\DecValTok{100}\NormalTok{))}
\end{Highlighting}
\end{Shaded}

\begin{verbatim}
## [1] 273.15
\end{verbatim}

\hypertarget{exercise-3}{%
\subsection{Exercise 3}\label{exercise-3}}

Rather than using a separate parameter in \texttt{to\_kelvin()} for each
source unit, a better design is to have a \texttt{temperature} parameter
and a \texttt{unit} parameter. The function can then check \texttt{unit}
and choose an appropriate conversion for \texttt{temperature}.

The \texttt{unit} parameter should be restricted to supported units,
which in this case are \texttt{"celsius"} and \texttt{"fahrenheit"}. You
can use the special \texttt{match.arg()} function to check that an
argument matches against a set of candidate arguments. For instance,
\texttt{match.arg(x,\ c("red",\ "blue"))} checks that the argument to
\texttt{x} is either \texttt{"red"} or \texttt{"blue"}. The function
also allows for partial matches, so for instance \texttt{"r"} matches
\texttt{"red"}. See the documentation for full details.

Rewrite the \texttt{to\_kelvin()} function with a \texttt{temperature}
and \texttt{unit} parameter. Make sure the function is still vectorized
in the \texttt{temperature} parameter.

As always, show that your function works correctly for a few test cases.

\textbf{YOUR ANSWER GOES HERE:}

\begin{Shaded}
\begin{Highlighting}[]
\NormalTok{to\_kelvin =}\StringTok{ }\ControlFlowTok{function}\NormalTok{(temperature, unit) \{}
\NormalTok{    unit =}\StringTok{ }\KeywordTok{match.arg}\NormalTok{(unit, }\KeywordTok{c}\NormalTok{(}\StringTok{"celsius"}\NormalTok{, }\StringTok{"fahrenheit"}\NormalTok{))}
    \ControlFlowTok{if}\NormalTok{ (unit }\OperatorTok{==}\StringTok{ "celsius"}\NormalTok{) \{}
\NormalTok{        temperature }\OperatorTok{+}\StringTok{ }\FloatTok{273.15}
\NormalTok{    \} }\ControlFlowTok{else} \ControlFlowTok{if}\NormalTok{ (unit }\OperatorTok{==}\StringTok{ "fahrenheit"}\NormalTok{) \{}
\NormalTok{        (temperature }\OperatorTok{{-}}\StringTok{ }\DecValTok{32}\NormalTok{) }\OperatorTok{*}\StringTok{ }\DecValTok{5} \OperatorTok{/}\StringTok{ }\DecValTok{9} \OperatorTok{+}\StringTok{ }\FloatTok{273.15}
\NormalTok{    \}}
\NormalTok{\}}
\KeywordTok{to\_kelvin}\NormalTok{(}\DecValTok{0}\NormalTok{, }\StringTok{"celsius"}\NormalTok{)}
\end{Highlighting}
\end{Shaded}

\begin{verbatim}
## [1] 273.15
\end{verbatim}

\begin{Shaded}
\begin{Highlighting}[]
\KeywordTok{to\_kelvin}\NormalTok{(}\KeywordTok{c}\NormalTok{(}\DecValTok{32}\NormalTok{, }\DecValTok{10}\NormalTok{, }\DecValTok{100}\NormalTok{), }\DataTypeTok{unit =} \StringTok{"f"}\NormalTok{)}
\end{Highlighting}
\end{Shaded}

\begin{verbatim}
## [1] 273.1500 260.9278 310.9278
\end{verbatim}

\begin{Shaded}
\begin{Highlighting}[]
\KeywordTok{to\_kelvin}\NormalTok{(}\KeywordTok{c}\NormalTok{(}\DecValTok{23}\NormalTok{, }\DecValTok{0}\NormalTok{, }\DecValTok{{-}100}\NormalTok{, }\DecValTok{20}\NormalTok{), }\DataTypeTok{unit =} \StringTok{"cel"}\NormalTok{)}
\end{Highlighting}
\end{Shaded}

\begin{verbatim}
## [1] 296.15 273.15 173.15 293.15
\end{verbatim}

\hypertarget{the-dots-parameter}{%
\section{The Dots Parameter}\label{the-dots-parameter}}

Watch the ``The Dots Parameter'' lecture video.

No exercises for this section.

\hypertarget{using-functions}{%
\section{Using Functions}\label{using-functions}}

Watch the ``Using Functions'' lecture video.

No exercises for this section.

\end{document}
