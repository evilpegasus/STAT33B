% Options for packages loaded elsewhere
\PassOptionsToPackage{unicode}{hyperref}
\PassOptionsToPackage{hyphens}{url}
%
\documentclass[
]{article}
\usepackage{lmodern}
\usepackage{amssymb,amsmath}
\usepackage{ifxetex,ifluatex}
\ifnum 0\ifxetex 1\fi\ifluatex 1\fi=0 % if pdftex
  \usepackage[T1]{fontenc}
  \usepackage[utf8]{inputenc}
  \usepackage{textcomp} % provide euro and other symbols
\else % if luatex or xetex
  \usepackage{unicode-math}
  \defaultfontfeatures{Scale=MatchLowercase}
  \defaultfontfeatures[\rmfamily]{Ligatures=TeX,Scale=1}
\fi
% Use upquote if available, for straight quotes in verbatim environments
\IfFileExists{upquote.sty}{\usepackage{upquote}}{}
\IfFileExists{microtype.sty}{% use microtype if available
  \usepackage[]{microtype}
  \UseMicrotypeSet[protrusion]{basicmath} % disable protrusion for tt fonts
}{}
\makeatletter
\@ifundefined{KOMAClassName}{% if non-KOMA class
  \IfFileExists{parskip.sty}{%
    \usepackage{parskip}
  }{% else
    \setlength{\parindent}{0pt}
    \setlength{\parskip}{6pt plus 2pt minus 1pt}}
}{% if KOMA class
  \KOMAoptions{parskip=half}}
\makeatother
\usepackage{xcolor}
\IfFileExists{xurl.sty}{\usepackage{xurl}}{} % add URL line breaks if available
\IfFileExists{bookmark.sty}{\usepackage{bookmark}}{\usepackage{hyperref}}
\hypersetup{
  pdftitle={STAT 33B Workbook 5},
  pdfauthor={Ming Fong (3035619833)},
  hidelinks,
  pdfcreator={LaTeX via pandoc}}
\urlstyle{same} % disable monospaced font for URLs
\usepackage[margin=1in]{geometry}
\usepackage{color}
\usepackage{fancyvrb}
\newcommand{\VerbBar}{|}
\newcommand{\VERB}{\Verb[commandchars=\\\{\}]}
\DefineVerbatimEnvironment{Highlighting}{Verbatim}{commandchars=\\\{\}}
% Add ',fontsize=\small' for more characters per line
\usepackage{framed}
\definecolor{shadecolor}{RGB}{248,248,248}
\newenvironment{Shaded}{\begin{snugshade}}{\end{snugshade}}
\newcommand{\AlertTok}[1]{\textcolor[rgb]{0.94,0.16,0.16}{#1}}
\newcommand{\AnnotationTok}[1]{\textcolor[rgb]{0.56,0.35,0.01}{\textbf{\textit{#1}}}}
\newcommand{\AttributeTok}[1]{\textcolor[rgb]{0.77,0.63,0.00}{#1}}
\newcommand{\BaseNTok}[1]{\textcolor[rgb]{0.00,0.00,0.81}{#1}}
\newcommand{\BuiltInTok}[1]{#1}
\newcommand{\CharTok}[1]{\textcolor[rgb]{0.31,0.60,0.02}{#1}}
\newcommand{\CommentTok}[1]{\textcolor[rgb]{0.56,0.35,0.01}{\textit{#1}}}
\newcommand{\CommentVarTok}[1]{\textcolor[rgb]{0.56,0.35,0.01}{\textbf{\textit{#1}}}}
\newcommand{\ConstantTok}[1]{\textcolor[rgb]{0.00,0.00,0.00}{#1}}
\newcommand{\ControlFlowTok}[1]{\textcolor[rgb]{0.13,0.29,0.53}{\textbf{#1}}}
\newcommand{\DataTypeTok}[1]{\textcolor[rgb]{0.13,0.29,0.53}{#1}}
\newcommand{\DecValTok}[1]{\textcolor[rgb]{0.00,0.00,0.81}{#1}}
\newcommand{\DocumentationTok}[1]{\textcolor[rgb]{0.56,0.35,0.01}{\textbf{\textit{#1}}}}
\newcommand{\ErrorTok}[1]{\textcolor[rgb]{0.64,0.00,0.00}{\textbf{#1}}}
\newcommand{\ExtensionTok}[1]{#1}
\newcommand{\FloatTok}[1]{\textcolor[rgb]{0.00,0.00,0.81}{#1}}
\newcommand{\FunctionTok}[1]{\textcolor[rgb]{0.00,0.00,0.00}{#1}}
\newcommand{\ImportTok}[1]{#1}
\newcommand{\InformationTok}[1]{\textcolor[rgb]{0.56,0.35,0.01}{\textbf{\textit{#1}}}}
\newcommand{\KeywordTok}[1]{\textcolor[rgb]{0.13,0.29,0.53}{\textbf{#1}}}
\newcommand{\NormalTok}[1]{#1}
\newcommand{\OperatorTok}[1]{\textcolor[rgb]{0.81,0.36,0.00}{\textbf{#1}}}
\newcommand{\OtherTok}[1]{\textcolor[rgb]{0.56,0.35,0.01}{#1}}
\newcommand{\PreprocessorTok}[1]{\textcolor[rgb]{0.56,0.35,0.01}{\textit{#1}}}
\newcommand{\RegionMarkerTok}[1]{#1}
\newcommand{\SpecialCharTok}[1]{\textcolor[rgb]{0.00,0.00,0.00}{#1}}
\newcommand{\SpecialStringTok}[1]{\textcolor[rgb]{0.31,0.60,0.02}{#1}}
\newcommand{\StringTok}[1]{\textcolor[rgb]{0.31,0.60,0.02}{#1}}
\newcommand{\VariableTok}[1]{\textcolor[rgb]{0.00,0.00,0.00}{#1}}
\newcommand{\VerbatimStringTok}[1]{\textcolor[rgb]{0.31,0.60,0.02}{#1}}
\newcommand{\WarningTok}[1]{\textcolor[rgb]{0.56,0.35,0.01}{\textbf{\textit{#1}}}}
\usepackage{graphicx}
\makeatletter
\def\maxwidth{\ifdim\Gin@nat@width>\linewidth\linewidth\else\Gin@nat@width\fi}
\def\maxheight{\ifdim\Gin@nat@height>\textheight\textheight\else\Gin@nat@height\fi}
\makeatother
% Scale images if necessary, so that they will not overflow the page
% margins by default, and it is still possible to overwrite the defaults
% using explicit options in \includegraphics[width, height, ...]{}
\setkeys{Gin}{width=\maxwidth,height=\maxheight,keepaspectratio}
% Set default figure placement to htbp
\makeatletter
\def\fps@figure{htbp}
\makeatother
\setlength{\emergencystretch}{3em} % prevent overfull lines
\providecommand{\tightlist}{%
  \setlength{\itemsep}{0pt}\setlength{\parskip}{0pt}}
\setcounter{secnumdepth}{-\maxdimen} % remove section numbering
\ifluatex
  \usepackage{selnolig}  % disable illegal ligatures
\fi

\title{STAT 33B Workbook 5}
\author{Ming Fong (3035619833)}
\date{Oct 1, 2020}

\begin{document}
\maketitle

This workbook is due \textbf{Oct 1, 2020} by 11:59pm PT.

The workbook is organized into sections that correspond to the lecture
videos for the week. Watch a video, then do the corresponding exercises
\emph{before} moving on to the next video.

Workbooks are graded for completeness, so as long as you make a clear
effort to solve each problem, you'll get full credit. That said, make
sure you understand the concepts here, because they're likely to
reappear in homeworks, quizzes, and later lectures.

As you work, write your answers in this notebook. Answer questions with
complete sentences, and put code in code chunks. You can make as many
new code chunks as you like.

In the notebook, you can run the line of code where the cursor is by
pressing \texttt{Ctrl} + \texttt{Enter} on Windows or \texttt{Cmd} +
\texttt{Enter} on Mac OS X. You can run an entire code chunk by clicking
on the green arrow in the upper right corner of the code chunk.

Please do not delete the exercises already in this notebook, because it
may interfere with our grading tools.

You need to submit your work in two places:

\begin{itemize}
\tightlist
\item
  Submit this Rmd file with your edits on bCourses.
\item
  Knit and submit the generated PDF file on Gradescope.
\end{itemize}

If you have any last-minute trouble knitting, \textbf{DON'T PANIC}.
Submit your Rmd file on time and follow up in office hours or on Piazza
to sort out the PDF.

\hypertarget{apply-function-basics}{%
\section{Apply Function Basics}\label{apply-function-basics}}

Watch the ``Apply Function Basics'' lecture video.

For exercises that mention the dogs data, you can use either
\texttt{dogs.rds} or \texttt{dogs\_tibble.rds}. Both are on the bCourse.

\hypertarget{exercise-1}{%
\subsection{Exercise 1}\label{exercise-1}}

\begin{enumerate}
\def\labelenumi{\arabic{enumi}.}
\item
  Suppose you call \texttt{sapply()} with a function that returns
  vectors. What class of object does \texttt{sapply()} return if all of
  the result vectors have the same length?

  For instance, what if the applied function returns a length-3 vector
  for each element?

  \emph{Hint: \texttt{rnorm()} and \texttt{class()} are examples of
  functions that return vectors.}
\item
  Again suppose you call \texttt{sapply()} with a function that returns
  vectors. What class of object does \texttt{sapply()} return if the
  result vectors have the different lengths?
\item
  Suppose you call \texttt{sapply()} with a function that returns
  different types. What happens?

  \emph{Hint: \texttt{all.equal()} is one function that returns
  different types.}
\end{enumerate}

\textbf{YOUR ANSWER GOES HERE:}

\begin{enumerate}
\def\labelenumi{\arabic{enumi}.}
\tightlist
\item
  \texttt{sapply} will return a matrix as the most simplified form.
\end{enumerate}

\begin{Shaded}
\begin{Highlighting}[]
\NormalTok{x =}\StringTok{ }\KeywordTok{c}\NormalTok{(}\DecValTok{3}\NormalTok{, }\DecValTok{3}\NormalTok{, }\DecValTok{3}\NormalTok{, }\DecValTok{3}\NormalTok{)}
\KeywordTok{sapply}\NormalTok{(x, rnorm, }\DataTypeTok{mean =} \DecValTok{0}\NormalTok{, }\DataTypeTok{sd =} \DecValTok{1}\NormalTok{)}
\end{Highlighting}
\end{Shaded}

\begin{verbatim}
##            [,1]       [,2]       [,3]       [,4]
## [1,] -0.5801456 -2.1926861 -0.7840201 -0.6369682
## [2,]  1.0898191 -0.1229504  0.1999118 -0.4676643
## [3,] -1.1039189 -0.3882296 -0.8155480  0.7028213
\end{verbatim}

\begin{Shaded}
\begin{Highlighting}[]
\KeywordTok{class}\NormalTok{(}\KeywordTok{sapply}\NormalTok{(x, rnorm, }\DataTypeTok{mean =} \DecValTok{0}\NormalTok{, }\DataTypeTok{sd =} \DecValTok{1}\NormalTok{))}
\end{Highlighting}
\end{Shaded}

\begin{verbatim}
## [1] "matrix" "array"
\end{verbatim}

\begin{enumerate}
\def\labelenumi{\arabic{enumi}.}
\setcounter{enumi}{1}
\tightlist
\item
  \texttt{sapply} will return a list here because a matrix could not be
  formed with different row lengths.
\end{enumerate}

\begin{Shaded}
\begin{Highlighting}[]
\NormalTok{x =}\StringTok{ }\KeywordTok{c}\NormalTok{(}\DecValTok{1}\NormalTok{, }\DecValTok{2}\NormalTok{, }\DecValTok{3}\NormalTok{, }\DecValTok{4}\NormalTok{)}
\KeywordTok{sapply}\NormalTok{(x, rnorm, }\DataTypeTok{mean =} \DecValTok{0}\NormalTok{, }\DataTypeTok{sd =} \DecValTok{1}\NormalTok{)}
\end{Highlighting}
\end{Shaded}

\begin{verbatim}
## [[1]]
## [1] -1.016675
## 
## [[2]]
## [1] -0.3220106 -0.2346543
## 
## [[3]]
## [1]  0.4279693 -0.7250352 -0.8241151
## 
## [[4]]
## [1] -0.07607897  1.99664888 -0.12693710 -0.66480056
\end{verbatim}

\begin{Shaded}
\begin{Highlighting}[]
\KeywordTok{class}\NormalTok{(}\KeywordTok{sapply}\NormalTok{(x, rnorm, }\DataTypeTok{mean =} \DecValTok{0}\NormalTok{, }\DataTypeTok{sd =} \DecValTok{1}\NormalTok{))}
\end{Highlighting}
\end{Shaded}

\begin{verbatim}
## [1] "list"
\end{verbatim}

\begin{enumerate}
\def\labelenumi{\arabic{enumi}.}
\setcounter{enumi}{2}
\tightlist
\item
  \texttt{sapply} returns a vector of class character. The
  \texttt{TRUE}s from \texttt{all.apply} are coerced into characters.
\end{enumerate}

\begin{Shaded}
\begin{Highlighting}[]
\NormalTok{x =}\StringTok{ }\KeywordTok{c}\NormalTok{(}\DecValTok{1}\NormalTok{, }\DecValTok{2}\NormalTok{, }\DecValTok{1}\NormalTok{, }\DecValTok{4}\NormalTok{, }\DecValTok{1}\NormalTok{, }\DecValTok{6}\NormalTok{)}
\NormalTok{y =}\StringTok{ }\DecValTok{1}
\KeywordTok{sapply}\NormalTok{(x, all.equal, }\DataTypeTok{current =}\NormalTok{ y)}
\end{Highlighting}
\end{Shaded}

\begin{verbatim}
## [1] "TRUE"                                "Mean relative difference: 0.5"      
## [3] "TRUE"                                "Mean relative difference: 0.75"     
## [5] "TRUE"                                "Mean relative difference: 0.8333333"
\end{verbatim}

\begin{Shaded}
\begin{Highlighting}[]
\KeywordTok{class}\NormalTok{(}\KeywordTok{sapply}\NormalTok{(x, all.equal, }\DataTypeTok{current =}\NormalTok{ y))}
\end{Highlighting}
\end{Shaded}

\begin{verbatim}
## [1] "character"
\end{verbatim}

\hypertarget{exercise-2}{%
\subsection{Exercise 2}\label{exercise-2}}

\begin{enumerate}
\def\labelenumi{\arabic{enumi}.}
\item
  Use \texttt{sapply()} and \texttt{is.numeric()} to identify all of the
  numeric columns in the dogs data frame.
\item
  Use \texttt{sapply()} and your result from part 1 to compute the range
  of every numeric column in the dogs data frame.
\end{enumerate}

\textbf{YOUR ANSWER GOES HERE:}

\begin{enumerate}
\def\labelenumi{\arabic{enumi}.}
\tightlist
\item
\end{enumerate}

\begin{Shaded}
\begin{Highlighting}[]
\NormalTok{dogs =}\StringTok{ }\KeywordTok{readRDS}\NormalTok{(}\StringTok{"data/dogs.rds"}\NormalTok{)}
\NormalTok{dogs\_numerics =}\StringTok{ }\KeywordTok{sapply}\NormalTok{(dogs, is.numeric)}
\NormalTok{dogs\_numerics}
\end{Highlighting}
\end{Shaded}

\begin{verbatim}
##             breed             group           datadog    popularity_all 
##             FALSE             FALSE              TRUE              TRUE 
##        popularity     lifetime_cost intelligence_rank         longevity 
##              TRUE              TRUE              TRUE              TRUE 
##          ailments             price         food_cost          grooming 
##              TRUE              TRUE              TRUE             FALSE 
##              kids     megarank_kids          megarank              size 
##             FALSE              TRUE              TRUE             FALSE 
##            weight            height 
##              TRUE              TRUE
\end{verbatim}

\begin{enumerate}
\def\labelenumi{\arabic{enumi}.}
\setcounter{enumi}{1}
\tightlist
\item
\end{enumerate}

\begin{Shaded}
\begin{Highlighting}[]
\KeywordTok{sapply}\NormalTok{(dogs[ , dogs\_numerics], range, }\DataTypeTok{na.rm =} \OtherTok{TRUE}\NormalTok{)}
\end{Highlighting}
\end{Shaded}

\begin{verbatim}
##      datadog popularity_all popularity lifetime_cost intelligence_rank
## [1,]    0.99              1          1         12653                 1
## [2,]    3.64            173         87         26686                80
##      longevity ailments price food_cost megarank_kids megarank weight height
## [1,]      6.29        0   283       270             1        1      5      5
## [2,]     16.50        9  3460      1349            87       87    175     32
\end{verbatim}

\hypertarget{the-split-apply-strategy}{%
\section{The Split-Apply Strategy}\label{the-split-apply-strategy}}

Watch the ``The Split-Apply Strategy'' lecture video.

\hypertarget{exercise-3}{%
\subsection{Exercise 3}\label{exercise-3}}

Use the split-apply strategy to compute the minimum weight (ignoring
missing values) for each size of dog.

\textbf{YOUR ANSWER GOES HERE:}

\begin{Shaded}
\begin{Highlighting}[]
\KeywordTok{sapply}\NormalTok{(}\KeywordTok{split}\NormalTok{(dogs}\OperatorTok{$}\NormalTok{weight, dogs}\OperatorTok{$}\NormalTok{size), min, }\DataTypeTok{na.rm =} \OtherTok{TRUE}\NormalTok{)}
\end{Highlighting}
\end{Shaded}

\begin{verbatim}
##  large medium  small 
##     55     16      5
\end{verbatim}

\hypertarget{exercise-4}{%
\subsection{Exercise 4}\label{exercise-4}}

Use \texttt{tapply()} to compute a \texttt{summary()} of the weight
column for each group (hound, herding, etc) of dog.

\textbf{YOUR ANSWER GOES HERE:}

\begin{Shaded}
\begin{Highlighting}[]
\KeywordTok{tapply}\NormalTok{(dogs}\OperatorTok{$}\NormalTok{weight, dogs}\OperatorTok{$}\NormalTok{group, summary)}
\end{Highlighting}
\end{Shaded}

\begin{verbatim}
## $herding
##    Min. 1st Qu.  Median    Mean 3rd Qu.    Max.    NA's 
##   22.00   27.38   32.25   36.67   42.00   62.50      19 
## 
## $hound
##    Min. 1st Qu.  Median    Mean 3rd Qu.    Max.    NA's 
##   23.00   51.12   63.75   63.83   83.12   97.50      14 
## 
## $`non-sporting`
##    Min. 1st Qu.  Median    Mean 3rd Qu.    Max.    NA's 
##   12.00   17.50   24.00   27.93   36.00   52.50      12 
## 
## $sporting
##    Min. 1st Qu.  Median    Mean 3rd Qu.    Max.    NA's 
##   25.00   37.50   59.50   51.97   63.75   70.00      13 
## 
## $terrier
##    Min. 1st Qu.  Median    Mean 3rd Qu.    Max.    NA's 
##   12.00   16.25   20.00   23.41   28.50   60.00       5 
## 
## $toy
##    Min. 1st Qu.  Median    Mean 3rd Qu.    Max.    NA's 
##   5.000   5.500  10.000   9.818  12.750  16.000       8 
## 
## $working
##    Min. 1st Qu.  Median    Mean 3rd Qu.    Max.    NA's 
##   47.50   79.38  107.50  105.00  126.25  175.00      15
\end{verbatim}

\hypertarget{exercise-5}{%
\subsection{Exercise 5}\label{exercise-5}}

The \texttt{aggregate()} function also implements the split-apply
strategy, but returns the results as a data frame.

Use \texttt{aggregate()} to compute the maximum weight (ignoring missing
values) for each group of dog.

\emph{Hint: The \texttt{by} parameter in \texttt{aggregate()} expects a
list or data frame, so use \texttt{{[}} to select columns for
\texttt{by} rather than \texttt{\$} or \texttt{{[}{[}}.}

\textbf{YOUR ANSWER GOES HERE:}

\begin{Shaded}
\begin{Highlighting}[]
\KeywordTok{aggregate}\NormalTok{(dogs}\OperatorTok{$}\NormalTok{weight, dogs[}\StringTok{"group"}\NormalTok{], max, }\DataTypeTok{na.rm =} \OtherTok{TRUE}\NormalTok{)}
\end{Highlighting}
\end{Shaded}

\begin{verbatim}
##          group     x
## 1      herding  62.5
## 2        hound  97.5
## 3 non-sporting  52.5
## 4     sporting  70.0
## 5      terrier  60.0
## 6          toy  16.0
## 7      working 175.0
\end{verbatim}

\hypertarget{exercise-6}{%
\subsection{Exercise 6}\label{exercise-6}}

Like \texttt{table()}, the \texttt{tapply()} function can use multiple
categorical features to cross-tabulate results.

Use \texttt{tapply()} to compute the median price (ignoring missing
values) for dogs, grouped by both size and grooming.

\emph{Hint: see the \texttt{tapply()} documentation for the
\texttt{INDEX} parameter.}

\textbf{YOUR ANSWER GOES HERE:}

\begin{Shaded}
\begin{Highlighting}[]
\KeywordTok{tapply}\NormalTok{(dogs}\OperatorTok{$}\NormalTok{price, }\KeywordTok{list}\NormalTok{(dogs}\OperatorTok{$}\NormalTok{size, dogs}\OperatorTok{$}\NormalTok{grooming), median, }\DataTypeTok{na.rm =} \OtherTok{TRUE}\NormalTok{)}
\end{Highlighting}
\end{Shaded}

\begin{verbatim}
##        daily weekly monthly
## large  842.5   1040      NA
## medium 832.0    810     650
## small  693.0    740      NA
\end{verbatim}

\hypertarget{even-more-apply-functions}{%
\section{Even More Apply Functions}\label{even-more-apply-functions}}

Watch the ``Even More Apply Functions'' lecture video.

\hypertarget{exercise-7}{%
\subsection{Exercise 7}\label{exercise-7}}

Translate your code from Exercise 1, Part 1 to use \texttt{vapply()}
rather than \texttt{sapply()}.

\textbf{YOUR ANSWER GOES HERE:}

\begin{enumerate}
\def\labelenumi{\arabic{enumi}.}
\tightlist
\item
  The return type here is still a matrix and the example output given
  was a length 3 numeric vector.
\end{enumerate}

\begin{Shaded}
\begin{Highlighting}[]
\NormalTok{x =}\StringTok{ }\KeywordTok{c}\NormalTok{(}\DecValTok{3}\NormalTok{, }\DecValTok{3}\NormalTok{, }\DecValTok{3}\NormalTok{, }\DecValTok{3}\NormalTok{)}
\KeywordTok{vapply}\NormalTok{(x, rnorm, }\KeywordTok{c}\NormalTok{(}\DecValTok{1}\NormalTok{, }\DecValTok{2}\NormalTok{, }\DecValTok{3}\NormalTok{), }\DataTypeTok{mean =} \DecValTok{0}\NormalTok{, }\DataTypeTok{sd =} \DecValTok{1}\NormalTok{)}
\end{Highlighting}
\end{Shaded}

\begin{verbatim}
##            [,1]       [,2]       [,3]       [,4]
## [1,] -0.9512065  0.5884557 1.31216109  1.2166134
## [2,] -0.2599608  0.4521554 1.89926443 -1.2457048
## [3,]  0.1901867 -0.3872492 0.08978934  0.2422749
\end{verbatim}

\begin{Shaded}
\begin{Highlighting}[]
\KeywordTok{class}\NormalTok{(}\KeywordTok{vapply}\NormalTok{(x, rnorm, }\KeywordTok{c}\NormalTok{(}\DecValTok{1}\NormalTok{, }\DecValTok{2}\NormalTok{, }\DecValTok{3}\NormalTok{), }\DataTypeTok{mean =} \DecValTok{0}\NormalTok{, }\DataTypeTok{sd =} \DecValTok{1}\NormalTok{))}
\end{Highlighting}
\end{Shaded}

\begin{verbatim}
## [1] "matrix" "array"
\end{verbatim}

\hypertarget{choosing-an-apply-function}{%
\section{Choosing an Apply Function}\label{choosing-an-apply-function}}

Watch the ``Choosing an Apply Function'' lecture video.

No exercises for this section.

\hypertarget{conditional-expressions}{%
\section{Conditional Expressions}\label{conditional-expressions}}

Watch the ``Choosing an Apply Function'' lecture video.

No exercises for this section.

\hypertarget{the-switch-function}{%
\section{\texorpdfstring{The \texttt{switch()}
Function}{The switch() Function}}\label{the-switch-function}}

Watch the ``The switch() Function'' lecture video.

No exercises for this section.

\hypertarget{the-congruent-vectors-strategy}{%
\section{The Congruent Vectors
Strategy}\label{the-congruent-vectors-strategy}}

Watch the ``The Congruent Vectors Strategy'' lecture video.

No exercises for this section.

\end{document}
