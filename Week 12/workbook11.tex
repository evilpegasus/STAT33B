% Options for packages loaded elsewhere
\PassOptionsToPackage{unicode}{hyperref}
\PassOptionsToPackage{hyphens}{url}
%
\documentclass[
]{article}
\usepackage{lmodern}
\usepackage{amssymb,amsmath}
\usepackage{ifxetex,ifluatex}
\ifnum 0\ifxetex 1\fi\ifluatex 1\fi=0 % if pdftex
  \usepackage[T1]{fontenc}
  \usepackage[utf8]{inputenc}
  \usepackage{textcomp} % provide euro and other symbols
\else % if luatex or xetex
  \usepackage{unicode-math}
  \defaultfontfeatures{Scale=MatchLowercase}
  \defaultfontfeatures[\rmfamily]{Ligatures=TeX,Scale=1}
\fi
% Use upquote if available, for straight quotes in verbatim environments
\IfFileExists{upquote.sty}{\usepackage{upquote}}{}
\IfFileExists{microtype.sty}{% use microtype if available
  \usepackage[]{microtype}
  \UseMicrotypeSet[protrusion]{basicmath} % disable protrusion for tt fonts
}{}
\makeatletter
\@ifundefined{KOMAClassName}{% if non-KOMA class
  \IfFileExists{parskip.sty}{%
    \usepackage{parskip}
  }{% else
    \setlength{\parindent}{0pt}
    \setlength{\parskip}{6pt plus 2pt minus 1pt}}
}{% if KOMA class
  \KOMAoptions{parskip=half}}
\makeatother
\usepackage{xcolor}
\IfFileExists{xurl.sty}{\usepackage{xurl}}{} % add URL line breaks if available
\IfFileExists{bookmark.sty}{\usepackage{bookmark}}{\usepackage{hyperref}}
\hypersetup{
  pdftitle={STAT 33B Workbook 11},
  pdfauthor={Ming Fong (3035619833)},
  hidelinks,
  pdfcreator={LaTeX via pandoc}}
\urlstyle{same} % disable monospaced font for URLs
\usepackage[margin=1in]{geometry}
\usepackage{color}
\usepackage{fancyvrb}
\newcommand{\VerbBar}{|}
\newcommand{\VERB}{\Verb[commandchars=\\\{\}]}
\DefineVerbatimEnvironment{Highlighting}{Verbatim}{commandchars=\\\{\}}
% Add ',fontsize=\small' for more characters per line
\usepackage{framed}
\definecolor{shadecolor}{RGB}{248,248,248}
\newenvironment{Shaded}{\begin{snugshade}}{\end{snugshade}}
\newcommand{\AlertTok}[1]{\textcolor[rgb]{0.94,0.16,0.16}{#1}}
\newcommand{\AnnotationTok}[1]{\textcolor[rgb]{0.56,0.35,0.01}{\textbf{\textit{#1}}}}
\newcommand{\AttributeTok}[1]{\textcolor[rgb]{0.77,0.63,0.00}{#1}}
\newcommand{\BaseNTok}[1]{\textcolor[rgb]{0.00,0.00,0.81}{#1}}
\newcommand{\BuiltInTok}[1]{#1}
\newcommand{\CharTok}[1]{\textcolor[rgb]{0.31,0.60,0.02}{#1}}
\newcommand{\CommentTok}[1]{\textcolor[rgb]{0.56,0.35,0.01}{\textit{#1}}}
\newcommand{\CommentVarTok}[1]{\textcolor[rgb]{0.56,0.35,0.01}{\textbf{\textit{#1}}}}
\newcommand{\ConstantTok}[1]{\textcolor[rgb]{0.00,0.00,0.00}{#1}}
\newcommand{\ControlFlowTok}[1]{\textcolor[rgb]{0.13,0.29,0.53}{\textbf{#1}}}
\newcommand{\DataTypeTok}[1]{\textcolor[rgb]{0.13,0.29,0.53}{#1}}
\newcommand{\DecValTok}[1]{\textcolor[rgb]{0.00,0.00,0.81}{#1}}
\newcommand{\DocumentationTok}[1]{\textcolor[rgb]{0.56,0.35,0.01}{\textbf{\textit{#1}}}}
\newcommand{\ErrorTok}[1]{\textcolor[rgb]{0.64,0.00,0.00}{\textbf{#1}}}
\newcommand{\ExtensionTok}[1]{#1}
\newcommand{\FloatTok}[1]{\textcolor[rgb]{0.00,0.00,0.81}{#1}}
\newcommand{\FunctionTok}[1]{\textcolor[rgb]{0.00,0.00,0.00}{#1}}
\newcommand{\ImportTok}[1]{#1}
\newcommand{\InformationTok}[1]{\textcolor[rgb]{0.56,0.35,0.01}{\textbf{\textit{#1}}}}
\newcommand{\KeywordTok}[1]{\textcolor[rgb]{0.13,0.29,0.53}{\textbf{#1}}}
\newcommand{\NormalTok}[1]{#1}
\newcommand{\OperatorTok}[1]{\textcolor[rgb]{0.81,0.36,0.00}{\textbf{#1}}}
\newcommand{\OtherTok}[1]{\textcolor[rgb]{0.56,0.35,0.01}{#1}}
\newcommand{\PreprocessorTok}[1]{\textcolor[rgb]{0.56,0.35,0.01}{\textit{#1}}}
\newcommand{\RegionMarkerTok}[1]{#1}
\newcommand{\SpecialCharTok}[1]{\textcolor[rgb]{0.00,0.00,0.00}{#1}}
\newcommand{\SpecialStringTok}[1]{\textcolor[rgb]{0.31,0.60,0.02}{#1}}
\newcommand{\StringTok}[1]{\textcolor[rgb]{0.31,0.60,0.02}{#1}}
\newcommand{\VariableTok}[1]{\textcolor[rgb]{0.00,0.00,0.00}{#1}}
\newcommand{\VerbatimStringTok}[1]{\textcolor[rgb]{0.31,0.60,0.02}{#1}}
\newcommand{\WarningTok}[1]{\textcolor[rgb]{0.56,0.35,0.01}{\textbf{\textit{#1}}}}
\usepackage{graphicx}
\makeatletter
\def\maxwidth{\ifdim\Gin@nat@width>\linewidth\linewidth\else\Gin@nat@width\fi}
\def\maxheight{\ifdim\Gin@nat@height>\textheight\textheight\else\Gin@nat@height\fi}
\makeatother
% Scale images if necessary, so that they will not overflow the page
% margins by default, and it is still possible to overwrite the defaults
% using explicit options in \includegraphics[width, height, ...]{}
\setkeys{Gin}{width=\maxwidth,height=\maxheight,keepaspectratio}
% Set default figure placement to htbp
\makeatletter
\def\fps@figure{htbp}
\makeatother
\setlength{\emergencystretch}{3em} % prevent overfull lines
\providecommand{\tightlist}{%
  \setlength{\itemsep}{0pt}\setlength{\parskip}{0pt}}
\setcounter{secnumdepth}{-\maxdimen} % remove section numbering
\ifluatex
  \usepackage{selnolig}  % disable illegal ligatures
\fi

\title{STAT 33B Workbook 11}
\author{Ming Fong (3035619833)}
\date{Nov 12, 2020}

\begin{document}
\maketitle

This workbook is due \textbf{Nov 12, 2020} by 11:59pm PT.

The workbook is organized into sections that correspond to the lecture
videos for the week. Watch a video, then do the corresponding exercises
\emph{before} moving on to the next video.

Workbooks are graded for completeness, so as long as you make a clear
effort to solve each problem, you'll get full credit. That said, make
sure you understand the concepts here, because they're likely to
reappear in homeworks, quizzes, and later lectures.

As you work, write your answers in this notebook. Answer questions with
complete sentences, and put code in code chunks. You can make as many
new code chunks as you like.

In the notebook, you can run the line of code where the cursor is by
pressing \texttt{Ctrl} + \texttt{Enter} on Windows or \texttt{Cmd} +
\texttt{Enter} on Mac OS X. You can run an entire code chunk by clicking
on the green arrow in the upper right corner of the code chunk.

Please do not delete the exercises already in this notebook, because it
may interfere with our grading tools.

You need to submit your work in two places:

\begin{itemize}
\tightlist
\item
  Submit this Rmd file with your edits on bCourses.
\item
  Knit and submit the generated PDF file on Gradescope.
\end{itemize}

If you have any last-minute trouble knitting, \textbf{DON'T PANIC}.
Submit your Rmd file on time and follow up in office hours or on Piazza
to sort out the PDF.

\hypertarget{code-performance}{%
\section{Code Performance}\label{code-performance}}

Watch the ``Code Performance'' lecture video.

No exercises for this section.

\hypertarget{profiling}{%
\section{Profiling}\label{profiling}}

Watch the ``Profiling'' lecture video.

No exercises for this section.

\hypertarget{profvis}{%
\section{Profvis}\label{profvis}}

Watch the ``Profvis'' lecture video.

\hypertarget{exercise-1}{%
\subsection{Exercise 1}\label{exercise-1}}

Read chapter 23 of \href{https://adv-r.hadley.nz}{Advanced R}.

The flame graph for the function \texttt{rhand} function in the lecture
has entries for a call to \texttt{\textless{}GC\textgreater{}}. What do
these entries mean? Why are the listed as being ``called'' by the
\texttt{c} function? Explain in 3-5 sentences.

\textbf{YOUR ANSWER GOES HERE:}

Calls to \texttt{\textless{}GC\textgreater{}} are for the garbage
collector, which clears unused memory. The \texttt{c} function calls
this because of its copy-on-write. Everytime you replace the value of c,
a new copy is created and the old one needs to be removed to free up
memory. This is inefficient and should be avoided if possible.

\hypertarget{profiling-case-study}{%
\section{Profiling Case Study}\label{profiling-case-study}}

Watch the ``Profiling Case Study'' lecture video.

The next two exercises are related to the ``Code Performance'' lecture,
but I've put them here at the end so the assigned readings are in order.

\hypertarget{exercise-2}{%
\subsection{Exercise 2}\label{exercise-2}}

Read chapter 24 of \href{https://adv-r.hadley.nz}{Advanced R}.

What's the difference between \texttt{colSums} and \texttt{.colSums}?
Explain.

\textbf{YOUR ANSWER GOES HERE:}

\texttt{.colSums} can only be used with numeric matrices and is the
bare-bones version of \texttt{colSums}. \texttt{.colSums} requires
specifying the number of rows and columns in the matrix.

\hypertarget{exercise-3}{%
\subsection{Exercise 3}\label{exercise-3}}

Create several test matrices of different sizes and use the
microbenchmark or bench package to compare the speeds of:

\begin{itemize}
\tightlist
\item
  \texttt{colSums(m)}
\item
  \texttt{.colSums(m)}
\item
  \texttt{apply(m,\ 2,\ sum)}
\end{itemize}

Here \texttt{m} denotes a matrix. Use a plot to present your results. Is
one of these consistently faster than the others? Does the size of the
matrix have an effect on which is faster?

\textbf{YOUR ANSWER GOES HERE:}

\begin{Shaded}
\begin{Highlighting}[]
\KeywordTok{library}\NormalTok{(microbenchmark)}
\NormalTok{n =}\StringTok{ }\DecValTok{1000}
\NormalTok{mats =}\StringTok{ }\KeywordTok{list}\NormalTok{()}
\ControlFlowTok{for}\NormalTok{(x }\ControlFlowTok{in} \DecValTok{1}\OperatorTok{:}\NormalTok{n) \{}
\NormalTok{    mats[[x]] =}\StringTok{ }\KeywordTok{matrix}\NormalTok{(}\DecValTok{1}\OperatorTok{:}\NormalTok{(x}\OperatorTok{\^{}}\DecValTok{2}\NormalTok{), }\DataTypeTok{nrow =}\NormalTok{ x, }\DataTypeTok{ncol =}\NormalTok{ x)}
\NormalTok{\}}

\NormalTok{times =}\StringTok{ }\KeywordTok{data.frame}\NormalTok{(}
    \DataTypeTok{n =} \KeywordTok{rep}\NormalTok{(}\OtherTok{NA}\NormalTok{, n),}
    \DataTypeTok{colSums =} \KeywordTok{rep}\NormalTok{(}\OtherTok{NA}\NormalTok{, n),}
    \DataTypeTok{.colSums =} \KeywordTok{rep}\NormalTok{(}\OtherTok{NA}\NormalTok{, n),}
    \DataTypeTok{apply =} \KeywordTok{rep}\NormalTok{(}\OtherTok{NA}\NormalTok{, n)}
\NormalTok{)}
\ControlFlowTok{for}\NormalTok{(x }\ControlFlowTok{in} \DecValTok{1}\OperatorTok{:}\NormalTok{n) \{}
\NormalTok{    times[x, }\StringTok{"n"}\NormalTok{] =}\StringTok{ }\NormalTok{x}
\NormalTok{    times[x, }\StringTok{"colSums"}\NormalTok{] =}\StringTok{ }\KeywordTok{summary}\NormalTok{(}\KeywordTok{microbenchmark}\NormalTok{(}\KeywordTok{colSums}\NormalTok{(mats[[x]]), }\DataTypeTok{unit =} \StringTok{"us"}\NormalTok{))[[}\StringTok{"mean"}\NormalTok{]]}
\NormalTok{    times[x, }\StringTok{".colSums"}\NormalTok{] =}\StringTok{ }\KeywordTok{summary}\NormalTok{(}\KeywordTok{microbenchmark}\NormalTok{(}\KeywordTok{.colSums}\NormalTok{(mats[[x]], x, x), }\DataTypeTok{unit =} \StringTok{"us"}\NormalTok{))[[}\StringTok{"mean"}\NormalTok{]]}
\NormalTok{    times[x, }\StringTok{"apply"}\NormalTok{] =}\StringTok{ }\KeywordTok{summary}\NormalTok{(}\KeywordTok{microbenchmark}\NormalTok{(}\KeywordTok{apply}\NormalTok{(mats[[x]], }\DecValTok{2}\NormalTok{, sum), }\DataTypeTok{unit =} \StringTok{"us"}\NormalTok{))[[}\StringTok{"mean"}\NormalTok{]]}
\NormalTok{\}}
\end{Highlighting}
\end{Shaded}

It seems that \texttt{.colSums} is the fastest method of the three. The
size of the matrix has little effect on the order of speeds
(\texttt{colSums} and \texttt{.colSums} occasionally switch).
\texttt{apply} is definitely slower than the other two.

\begin{Shaded}
\begin{Highlighting}[]
\KeywordTok{library}\NormalTok{(ggplot2)}
\KeywordTok{ggplot}\NormalTok{(times, }\KeywordTok{aes}\NormalTok{(}\DataTypeTok{x =}\NormalTok{ n)) }\OperatorTok{+}
\StringTok{    }\KeywordTok{geom\_line}\NormalTok{(}\KeywordTok{aes}\NormalTok{(}\DataTypeTok{y =}\NormalTok{ colSums, }\DataTypeTok{color =} \StringTok{"colSums"}\NormalTok{)) }\OperatorTok{+}
\StringTok{    }\KeywordTok{geom\_line}\NormalTok{(}\KeywordTok{aes}\NormalTok{(}\DataTypeTok{y =}\NormalTok{ .colSums, }\DataTypeTok{color =} \StringTok{".colSums"}\NormalTok{)) }\OperatorTok{+}
\StringTok{    }\KeywordTok{geom\_line}\NormalTok{(}\KeywordTok{aes}\NormalTok{(}\DataTypeTok{y =}\NormalTok{ apply, }\DataTypeTok{color =} \StringTok{"apply"}\NormalTok{)) }\OperatorTok{+}
\StringTok{    }\KeywordTok{labs}\NormalTok{(}\DataTypeTok{title =} \StringTok{"Time vs n"}\NormalTok{, }\DataTypeTok{color =} \StringTok{"Function"}\NormalTok{) }\OperatorTok{+}
\StringTok{    }\KeywordTok{xlab}\NormalTok{(}\StringTok{"n"}\NormalTok{) }\OperatorTok{+}
\StringTok{    }\KeywordTok{ylab}\NormalTok{(}\StringTok{"Time (us)"}\NormalTok{) }\OperatorTok{+}
\StringTok{    }\KeywordTok{theme}\NormalTok{()}
\end{Highlighting}
\end{Shaded}

\includegraphics{workbook11_files/figure-latex/unnamed-chunk-2-1.pdf}

\end{document}
