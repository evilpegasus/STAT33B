% Options for packages loaded elsewhere
\PassOptionsToPackage{unicode}{hyperref}
\PassOptionsToPackage{hyphens}{url}
%
\documentclass[
]{article}
\usepackage{lmodern}
\usepackage{amssymb,amsmath}
\usepackage{ifxetex,ifluatex}
\ifnum 0\ifxetex 1\fi\ifluatex 1\fi=0 % if pdftex
  \usepackage[T1]{fontenc}
  \usepackage[utf8]{inputenc}
  \usepackage{textcomp} % provide euro and other symbols
\else % if luatex or xetex
  \usepackage{unicode-math}
  \defaultfontfeatures{Scale=MatchLowercase}
  \defaultfontfeatures[\rmfamily]{Ligatures=TeX,Scale=1}
\fi
% Use upquote if available, for straight quotes in verbatim environments
\IfFileExists{upquote.sty}{\usepackage{upquote}}{}
\IfFileExists{microtype.sty}{% use microtype if available
  \usepackage[]{microtype}
  \UseMicrotypeSet[protrusion]{basicmath} % disable protrusion for tt fonts
}{}
\makeatletter
\@ifundefined{KOMAClassName}{% if non-KOMA class
  \IfFileExists{parskip.sty}{%
    \usepackage{parskip}
  }{% else
    \setlength{\parindent}{0pt}
    \setlength{\parskip}{6pt plus 2pt minus 1pt}}
}{% if KOMA class
  \KOMAoptions{parskip=half}}
\makeatother
\usepackage{xcolor}
\IfFileExists{xurl.sty}{\usepackage{xurl}}{} % add URL line breaks if available
\IfFileExists{bookmark.sty}{\usepackage{bookmark}}{\usepackage{hyperref}}
\hypersetup{
  pdftitle={STAT 33B Workbook 1},
  pdfauthor={Ming Fong (3035619833)},
  hidelinks,
  pdfcreator={LaTeX via pandoc}}
\urlstyle{same} % disable monospaced font for URLs
\usepackage[margin=1in]{geometry}
\usepackage{color}
\usepackage{fancyvrb}
\newcommand{\VerbBar}{|}
\newcommand{\VERB}{\Verb[commandchars=\\\{\}]}
\DefineVerbatimEnvironment{Highlighting}{Verbatim}{commandchars=\\\{\}}
% Add ',fontsize=\small' for more characters per line
\usepackage{framed}
\definecolor{shadecolor}{RGB}{248,248,248}
\newenvironment{Shaded}{\begin{snugshade}}{\end{snugshade}}
\newcommand{\AlertTok}[1]{\textcolor[rgb]{0.94,0.16,0.16}{#1}}
\newcommand{\AnnotationTok}[1]{\textcolor[rgb]{0.56,0.35,0.01}{\textbf{\textit{#1}}}}
\newcommand{\AttributeTok}[1]{\textcolor[rgb]{0.77,0.63,0.00}{#1}}
\newcommand{\BaseNTok}[1]{\textcolor[rgb]{0.00,0.00,0.81}{#1}}
\newcommand{\BuiltInTok}[1]{#1}
\newcommand{\CharTok}[1]{\textcolor[rgb]{0.31,0.60,0.02}{#1}}
\newcommand{\CommentTok}[1]{\textcolor[rgb]{0.56,0.35,0.01}{\textit{#1}}}
\newcommand{\CommentVarTok}[1]{\textcolor[rgb]{0.56,0.35,0.01}{\textbf{\textit{#1}}}}
\newcommand{\ConstantTok}[1]{\textcolor[rgb]{0.00,0.00,0.00}{#1}}
\newcommand{\ControlFlowTok}[1]{\textcolor[rgb]{0.13,0.29,0.53}{\textbf{#1}}}
\newcommand{\DataTypeTok}[1]{\textcolor[rgb]{0.13,0.29,0.53}{#1}}
\newcommand{\DecValTok}[1]{\textcolor[rgb]{0.00,0.00,0.81}{#1}}
\newcommand{\DocumentationTok}[1]{\textcolor[rgb]{0.56,0.35,0.01}{\textbf{\textit{#1}}}}
\newcommand{\ErrorTok}[1]{\textcolor[rgb]{0.64,0.00,0.00}{\textbf{#1}}}
\newcommand{\ExtensionTok}[1]{#1}
\newcommand{\FloatTok}[1]{\textcolor[rgb]{0.00,0.00,0.81}{#1}}
\newcommand{\FunctionTok}[1]{\textcolor[rgb]{0.00,0.00,0.00}{#1}}
\newcommand{\ImportTok}[1]{#1}
\newcommand{\InformationTok}[1]{\textcolor[rgb]{0.56,0.35,0.01}{\textbf{\textit{#1}}}}
\newcommand{\KeywordTok}[1]{\textcolor[rgb]{0.13,0.29,0.53}{\textbf{#1}}}
\newcommand{\NormalTok}[1]{#1}
\newcommand{\OperatorTok}[1]{\textcolor[rgb]{0.81,0.36,0.00}{\textbf{#1}}}
\newcommand{\OtherTok}[1]{\textcolor[rgb]{0.56,0.35,0.01}{#1}}
\newcommand{\PreprocessorTok}[1]{\textcolor[rgb]{0.56,0.35,0.01}{\textit{#1}}}
\newcommand{\RegionMarkerTok}[1]{#1}
\newcommand{\SpecialCharTok}[1]{\textcolor[rgb]{0.00,0.00,0.00}{#1}}
\newcommand{\SpecialStringTok}[1]{\textcolor[rgb]{0.31,0.60,0.02}{#1}}
\newcommand{\StringTok}[1]{\textcolor[rgb]{0.31,0.60,0.02}{#1}}
\newcommand{\VariableTok}[1]{\textcolor[rgb]{0.00,0.00,0.00}{#1}}
\newcommand{\VerbatimStringTok}[1]{\textcolor[rgb]{0.31,0.60,0.02}{#1}}
\newcommand{\WarningTok}[1]{\textcolor[rgb]{0.56,0.35,0.01}{\textbf{\textit{#1}}}}
\usepackage{graphicx}
\makeatletter
\def\maxwidth{\ifdim\Gin@nat@width>\linewidth\linewidth\else\Gin@nat@width\fi}
\def\maxheight{\ifdim\Gin@nat@height>\textheight\textheight\else\Gin@nat@height\fi}
\makeatother
% Scale images if necessary, so that they will not overflow the page
% margins by default, and it is still possible to overwrite the defaults
% using explicit options in \includegraphics[width, height, ...]{}
\setkeys{Gin}{width=\maxwidth,height=\maxheight,keepaspectratio}
% Set default figure placement to htbp
\makeatletter
\def\fps@figure{htbp}
\makeatother
\setlength{\emergencystretch}{3em} % prevent overfull lines
\providecommand{\tightlist}{%
  \setlength{\itemsep}{0pt}\setlength{\parskip}{0pt}}
\setcounter{secnumdepth}{-\maxdimen} % remove section numbering
\ifluatex
  \usepackage{selnolig}  % disable illegal ligatures
\fi

\title{STAT 33B Workbook 1}
\author{Ming Fong (3035619833)}
\date{Sep 3, 2020}

\begin{document}
\maketitle

This workbook is due \textbf{Sep 3, 2020} by 11:59pm PT.

The workbook is organized into sections that correspond to the lecture
videos for the week. Watch a video, then do the corresponding exercises
\emph{before} moving on to the next video.

Workbooks are graded for completeness, so as long as you make a clear
effort to solve each problem, you'll get full credit. That said, make
sure you understand the concepts here, because they're likely to
reappear in homeworks, quizzes, and later lectures.

As you work, write your answers in this notebook. Answer questions with
complete sentences, and put code in code chunks. You can make as many
new code chunks as you like.

In the notebook, you can run the line of code where the cursor is by
pressing \texttt{Ctrl} + \texttt{Enter} on Windows or \texttt{Cmd} +
\texttt{Enter} on Mac OS X. You can run an entire code chunk by clicking
on the green arrow in the upper right corner of the code chunk.

Please do not delete the exercises already in this notebook, because it
may interfere with our grading tools.

\hypertarget{data-types}{%
\section{Data Types}\label{data-types}}

Watch the ``Data Types'' lecture video.

\hypertarget{exercise-1}{%
\subsection{Exercise 1}\label{exercise-1}}

In R, if you pass vectors with different lengths to a binary operator,
the shorter one will be \textbf{recycled}. This means the elements of
the shorter vector will be repeated to match the length of the longer
vector.

Use the recycling rule to explain what's happening in each of these
lines of code:

\begin{Shaded}
\begin{Highlighting}[]
\KeywordTok{c}\NormalTok{(}\DecValTok{1}\NormalTok{, }\DecValTok{2}\NormalTok{) }\OperatorTok{+}\StringTok{ }\KeywordTok{c}\NormalTok{(}\DecValTok{3}\NormalTok{, }\DecValTok{4}\NormalTok{)}
\end{Highlighting}
\end{Shaded}

\begin{verbatim}
## [1] 4 6
\end{verbatim}

\begin{Shaded}
\begin{Highlighting}[]
\KeywordTok{c}\NormalTok{(}\DecValTok{20}\NormalTok{, }\DecValTok{30}\NormalTok{, }\DecValTok{40}\NormalTok{) }\OperatorTok{/}\StringTok{ }\DecValTok{10}
\end{Highlighting}
\end{Shaded}

\begin{verbatim}
## [1] 2 3 4
\end{verbatim}

\begin{Shaded}
\begin{Highlighting}[]
\KeywordTok{c}\NormalTok{(}\DecValTok{1}\NormalTok{, }\DecValTok{3}\NormalTok{) }\OperatorTok{+}\StringTok{ }\KeywordTok{c}\NormalTok{(}\DecValTok{0}\NormalTok{, }\DecValTok{0}\NormalTok{, }\DecValTok{0}\NormalTok{, }\DecValTok{0}\NormalTok{, }\DecValTok{0}\NormalTok{)}
\end{Highlighting}
\end{Shaded}

\begin{verbatim}
## Warning in c(1, 3) + c(0, 0, 0, 0, 0): longer object length is not a multiple of
## shorter object length
\end{verbatim}

\begin{verbatim}
## [1] 1 3 1 3 1
\end{verbatim}

YOUR ANSWER GOES HERE:

\begin{enumerate}
\def\labelenumi{\arabic{enumi}.}
\tightlist
\item
  Two vectors of equal length are added. The value at the nth index of
  the first array is added to the nth index of the second array.
\item
  The value of each index is divided by 10.
\item
  The values of the first vector are copied until there are the same
  number of elements in each vector. Then the vectors are added like in
  the first example.
\end{enumerate}

\hypertarget{exercise-2}{%
\subsection{Exercise 2}\label{exercise-2}}

Run each line in the following code chunk and inspect the result. For
each one, state the type and class of the result, and explain why the
result has that type.

\begin{Shaded}
\begin{Highlighting}[]
\KeywordTok{c}\NormalTok{(}\OtherTok{TRUE}\NormalTok{, }\StringTok{"hello"}\NormalTok{, }\DecValTok{3}\NormalTok{, }\DecValTok{6}\NormalTok{)}
\end{Highlighting}
\end{Shaded}

\begin{verbatim}
## [1] "TRUE"  "hello" "3"     "6"
\end{verbatim}

\begin{Shaded}
\begin{Highlighting}[]
\NormalTok{3L }\OperatorTok{+}\StringTok{ }\NormalTok{3i}
\end{Highlighting}
\end{Shaded}

\begin{verbatim}
## [1] 3+3i
\end{verbatim}

\begin{Shaded}
\begin{Highlighting}[]
\KeywordTok{c}\NormalTok{(3L, 4L, 5L) }\OperatorTok{/}\StringTok{ }\OtherTok{TRUE}
\end{Highlighting}
\end{Shaded}

\begin{verbatim}
## [1] 3 4 5
\end{verbatim}

YOUR ANSWER GOES HERE:

\hypertarget{exercise-3}{%
\subsection{Exercise 3}\label{exercise-3}}

Another way to create vectors is with the \texttt{rep()} function. The
\texttt{rep()} function creates a vector by replicating a value or
vector of values.

\begin{enumerate}
\def\labelenumi{\arabic{enumi}.}
\item
  The first parameter of \texttt{rep()} is the thing to replicate. The
  second parameter, \texttt{times}, is the number of times to to
  replicate. Use \texttt{rep()} to make a vector with 10 elements, all
  equal to 78.
\item
  What happens if you pass a vector as the first argument to
  \texttt{rep()}? Give some examples.
\item
  Skim the help file \texttt{?rep}. What happens if you pass a vector as
  the second argument to \texttt{rep()}? The help file might seem a bit
  cryptic, so you'll also need to experiment. Give some examples.
\end{enumerate}

YOUR ANSWER GOES HERE:

\hypertarget{exercise-4}{%
\subsection{Exercise 4}\label{exercise-4}}

Yet another way to create vectors is with the \texttt{seq()} function.
The \texttt{seq()} function creates a vector that contains a sequence of
numbers.

Skim the help file \texttt{?seq}. Give some examples of creating vectors
with the \texttt{seq()} function.

YOUR ANSWER GOES HERE:

\hypertarget{exercise-5}{%
\subsection{Exercise 5}\label{exercise-5}}

In R, \texttt{T} and \texttt{F} are shortcuts for \texttt{TRUE} and
\texttt{FALSE}.

\begin{enumerate}
\def\labelenumi{\arabic{enumi}.}
\item
  What happens if you try to assign a value to \texttt{TRUE}?
\item
  What happens if you try to assign a value to \texttt{T}?
\item
  Check that what you observed in \#1 and \#2 is also true for
  \texttt{FALSE} and \texttt{F}. Why might it be safer to use
  \texttt{TRUE} and \texttt{FALSE} rather than \texttt{T} and \texttt{F}
  in code?
\end{enumerate}

YOUR ANSWER GOES HERE:

\hypertarget{matrices-arrays-lists}{%
\section{Matrices, Arrays, \& Lists}\label{matrices-arrays-lists}}

Watch the ``Matrices, Arrays, \& Lists'' lecture video.

\hypertarget{exercise-6}{%
\subsection{Exercise 6}\label{exercise-6}}

Recall that many of R's functions are vectorized, which means they are
applied element-by-element to vectors.

\begin{enumerate}
\def\labelenumi{\arabic{enumi}.}
\item
  What happens if you call a vectorized function on a matrix?
\item
  What happens if you call a vectorized function on an array?
\end{enumerate}

Give examples to support your answer.

YOUR ANSWER GOES HERE:

\hypertarget{exercise-7}{%
\subsection{Exercise 7}\label{exercise-7}}

Suppose we want to multiply a length-2 vector with a 2-by-2 matrix.

What happens if you use \texttt{*} to multiply them? What happens if you
use \texttt{\%*\%}?

Give some examples that show the difference, including for vectors and
matrices of other sizes.

YOUR ANSWER GOES HERE:

\hypertarget{exercise-8}{%
\subsection{Exercise 8}\label{exercise-8}}

The \texttt{c()} function combines vectors, but it can also combine
lists. Use \texttt{list()} to create two lists, and show that
\texttt{c()} can be used to combine them.

YOUR ANSWER GOES HERE:

\hypertarget{special-values}{%
\section{Special Values}\label{special-values}}

Watch the ``Special Values'' lecture video.

\hypertarget{exercise-9}{%
\subsection{Exercise 9}\label{exercise-9}}

Skim the help file for the \texttt{mean()} function.

\begin{enumerate}
\def\labelenumi{\arabic{enumi}.}
\item
  What happens if you call the mean function on a vector that contains
  missing values? Is there a way to override this behavior?
\item
  What happens if you call the mean function on a vector that contains
  \texttt{NaN} values or infinite values?
\end{enumerate}

In each case, provide examples to suport your answers.

YOUR ANSWER GOES HERE:

\hypertarget{making-comparisons}{%
\section{Making Comparisons}\label{making-comparisons}}

Watch the ``Making Comparisons'' lecture video.

\hypertarget{exercise-10}{%
\subsection{Exercise 10}\label{exercise-10}}

Each of the following lines of code produces a result that, at a glance,
you might not expect. Explain the reason for each result.

\begin{Shaded}
\begin{Highlighting}[]
\DecValTok{3} \OperatorTok{==}\StringTok{ "3"}
\end{Highlighting}
\end{Shaded}

\begin{verbatim}
## [1] TRUE
\end{verbatim}

\begin{Shaded}
\begin{Highlighting}[]
\DecValTok{50} \OperatorTok{\textless{}}\StringTok{ \textquotesingle{}6\textquotesingle{}}
\end{Highlighting}
\end{Shaded}

\begin{verbatim}
## [1] TRUE
\end{verbatim}

\begin{Shaded}
\begin{Highlighting}[]
\KeywordTok{isTRUE}\NormalTok{(}\StringTok{"TRUE"}\NormalTok{)}
\end{Highlighting}
\end{Shaded}

\begin{verbatim}
## [1] FALSE
\end{verbatim}

YOUR ANSWER GOES HERE:

\hypertarget{exercise-11}{%
\subsection{Exercise 11}\label{exercise-11}}

Suppose you want to check whether any of the values in
\texttt{c(1,\ 2,\ 3)} appear in the vector \texttt{c(4,\ 1,\ 3,\ 1)}.

Novice R users often expect they can check with the code:

\begin{Shaded}
\begin{Highlighting}[]
\KeywordTok{c}\NormalTok{(}\DecValTok{1}\NormalTok{, }\DecValTok{2}\NormalTok{, }\DecValTok{3}\NormalTok{) }\OperatorTok{==}\StringTok{ }\KeywordTok{c}\NormalTok{(}\DecValTok{4}\NormalTok{, }\DecValTok{1}\NormalTok{, }\DecValTok{3}\NormalTok{, }\DecValTok{1}\NormalTok{)}
\end{Highlighting}
\end{Shaded}

\begin{verbatim}
## Warning in c(1, 2, 3) == c(4, 1, 3, 1): longer object length is not a multiple
## of shorter object length
\end{verbatim}

\begin{verbatim}
## [1] FALSE FALSE  TRUE  TRUE
\end{verbatim}

\begin{enumerate}
\def\labelenumi{\arabic{enumi}.}
\item
  Explain why the code above is not correct, and what's actually
  happening.
\item
  The correct way is to use the \texttt{\%in\%} operator. Give some
  examples of using the \texttt{\%in\%} operator. Recall that you can
  access its help page with \texttt{?"\%in\%"}.
\end{enumerate}

YOUR ANSWER GOES HERE:

\hypertarget{submitting-your-work}{%
\section{Submitting Your Work}\label{submitting-your-work}}

Congratulations, you made it through the first workbook!

You need to submit your work in two places:

\begin{itemize}
\tightlist
\item
  Submit this Rmd file with your edits on bCourses.
\item
  Knit and submit the generated PDF file on Gradescope.
\end{itemize}

\end{document}
