% Options for packages loaded elsewhere
\PassOptionsToPackage{unicode}{hyperref}
\PassOptionsToPackage{hyphens}{url}
%
\documentclass[
]{article}
\usepackage{lmodern}
\usepackage{amssymb,amsmath}
\usepackage{ifxetex,ifluatex}
\ifnum 0\ifxetex 1\fi\ifluatex 1\fi=0 % if pdftex
  \usepackage[T1]{fontenc}
  \usepackage[utf8]{inputenc}
  \usepackage{textcomp} % provide euro and other symbols
\else % if luatex or xetex
  \usepackage{unicode-math}
  \defaultfontfeatures{Scale=MatchLowercase}
  \defaultfontfeatures[\rmfamily]{Ligatures=TeX,Scale=1}
\fi
% Use upquote if available, for straight quotes in verbatim environments
\IfFileExists{upquote.sty}{\usepackage{upquote}}{}
\IfFileExists{microtype.sty}{% use microtype if available
  \usepackage[]{microtype}
  \UseMicrotypeSet[protrusion]{basicmath} % disable protrusion for tt fonts
}{}
\makeatletter
\@ifundefined{KOMAClassName}{% if non-KOMA class
  \IfFileExists{parskip.sty}{%
    \usepackage{parskip}
  }{% else
    \setlength{\parindent}{0pt}
    \setlength{\parskip}{6pt plus 2pt minus 1pt}}
}{% if KOMA class
  \KOMAoptions{parskip=half}}
\makeatother
\usepackage{xcolor}
\IfFileExists{xurl.sty}{\usepackage{xurl}}{} % add URL line breaks if available
\IfFileExists{bookmark.sty}{\usepackage{bookmark}}{\usepackage{hyperref}}
\hypersetup{
  pdftitle={STAT 33B Workbook 7},
  pdfauthor={Ming Fong (3035619833)},
  hidelinks,
  pdfcreator={LaTeX via pandoc}}
\urlstyle{same} % disable monospaced font for URLs
\usepackage[margin=1in]{geometry}
\usepackage{color}
\usepackage{fancyvrb}
\newcommand{\VerbBar}{|}
\newcommand{\VERB}{\Verb[commandchars=\\\{\}]}
\DefineVerbatimEnvironment{Highlighting}{Verbatim}{commandchars=\\\{\}}
% Add ',fontsize=\small' for more characters per line
\usepackage{framed}
\definecolor{shadecolor}{RGB}{248,248,248}
\newenvironment{Shaded}{\begin{snugshade}}{\end{snugshade}}
\newcommand{\AlertTok}[1]{\textcolor[rgb]{0.94,0.16,0.16}{#1}}
\newcommand{\AnnotationTok}[1]{\textcolor[rgb]{0.56,0.35,0.01}{\textbf{\textit{#1}}}}
\newcommand{\AttributeTok}[1]{\textcolor[rgb]{0.77,0.63,0.00}{#1}}
\newcommand{\BaseNTok}[1]{\textcolor[rgb]{0.00,0.00,0.81}{#1}}
\newcommand{\BuiltInTok}[1]{#1}
\newcommand{\CharTok}[1]{\textcolor[rgb]{0.31,0.60,0.02}{#1}}
\newcommand{\CommentTok}[1]{\textcolor[rgb]{0.56,0.35,0.01}{\textit{#1}}}
\newcommand{\CommentVarTok}[1]{\textcolor[rgb]{0.56,0.35,0.01}{\textbf{\textit{#1}}}}
\newcommand{\ConstantTok}[1]{\textcolor[rgb]{0.00,0.00,0.00}{#1}}
\newcommand{\ControlFlowTok}[1]{\textcolor[rgb]{0.13,0.29,0.53}{\textbf{#1}}}
\newcommand{\DataTypeTok}[1]{\textcolor[rgb]{0.13,0.29,0.53}{#1}}
\newcommand{\DecValTok}[1]{\textcolor[rgb]{0.00,0.00,0.81}{#1}}
\newcommand{\DocumentationTok}[1]{\textcolor[rgb]{0.56,0.35,0.01}{\textbf{\textit{#1}}}}
\newcommand{\ErrorTok}[1]{\textcolor[rgb]{0.64,0.00,0.00}{\textbf{#1}}}
\newcommand{\ExtensionTok}[1]{#1}
\newcommand{\FloatTok}[1]{\textcolor[rgb]{0.00,0.00,0.81}{#1}}
\newcommand{\FunctionTok}[1]{\textcolor[rgb]{0.00,0.00,0.00}{#1}}
\newcommand{\ImportTok}[1]{#1}
\newcommand{\InformationTok}[1]{\textcolor[rgb]{0.56,0.35,0.01}{\textbf{\textit{#1}}}}
\newcommand{\KeywordTok}[1]{\textcolor[rgb]{0.13,0.29,0.53}{\textbf{#1}}}
\newcommand{\NormalTok}[1]{#1}
\newcommand{\OperatorTok}[1]{\textcolor[rgb]{0.81,0.36,0.00}{\textbf{#1}}}
\newcommand{\OtherTok}[1]{\textcolor[rgb]{0.56,0.35,0.01}{#1}}
\newcommand{\PreprocessorTok}[1]{\textcolor[rgb]{0.56,0.35,0.01}{\textit{#1}}}
\newcommand{\RegionMarkerTok}[1]{#1}
\newcommand{\SpecialCharTok}[1]{\textcolor[rgb]{0.00,0.00,0.00}{#1}}
\newcommand{\SpecialStringTok}[1]{\textcolor[rgb]{0.31,0.60,0.02}{#1}}
\newcommand{\StringTok}[1]{\textcolor[rgb]{0.31,0.60,0.02}{#1}}
\newcommand{\VariableTok}[1]{\textcolor[rgb]{0.00,0.00,0.00}{#1}}
\newcommand{\VerbatimStringTok}[1]{\textcolor[rgb]{0.31,0.60,0.02}{#1}}
\newcommand{\WarningTok}[1]{\textcolor[rgb]{0.56,0.35,0.01}{\textbf{\textit{#1}}}}
\usepackage{graphicx}
\makeatletter
\def\maxwidth{\ifdim\Gin@nat@width>\linewidth\linewidth\else\Gin@nat@width\fi}
\def\maxheight{\ifdim\Gin@nat@height>\textheight\textheight\else\Gin@nat@height\fi}
\makeatother
% Scale images if necessary, so that they will not overflow the page
% margins by default, and it is still possible to overwrite the defaults
% using explicit options in \includegraphics[width, height, ...]{}
\setkeys{Gin}{width=\maxwidth,height=\maxheight,keepaspectratio}
% Set default figure placement to htbp
\makeatletter
\def\fps@figure{htbp}
\makeatother
\setlength{\emergencystretch}{3em} % prevent overfull lines
\providecommand{\tightlist}{%
  \setlength{\itemsep}{0pt}\setlength{\parskip}{0pt}}
\setcounter{secnumdepth}{-\maxdimen} % remove section numbering
\ifluatex
  \usepackage{selnolig}  % disable illegal ligatures
\fi

\title{STAT 33B Workbook 7}
\author{Ming Fong (3035619833)}
\date{Oct 15, 2020}

\begin{document}
\maketitle

This workbook is due \textbf{Oct 15, 2020} by 11:59pm PT.

The workbook is organized into sections that correspond to the lecture
videos for the week. Watch a video, then do the corresponding exercises
\emph{before} moving on to the next video.

Workbooks are graded for completeness, so as long as you make a clear
effort to solve each problem, you'll get full credit. That said, make
sure you understand the concepts here, because they're likely to
reappear in homeworks, quizzes, and later lectures.

As you work, write your answers in this notebook. Answer questions with
complete sentences, and put code in code chunks. You can make as many
new code chunks as you like.

In the notebook, you can run the line of code where the cursor is by
pressing \texttt{Ctrl} + \texttt{Enter} on Windows or \texttt{Cmd} +
\texttt{Enter} on Mac OS X. You can run an entire code chunk by clicking
on the green arrow in the upper right corner of the code chunk.

Please do not delete the exercises already in this notebook, because it
may interfere with our grading tools.

You need to submit your work in two places:

\begin{itemize}
\tightlist
\item
  Submit this Rmd file with your edits on bCourses.
\item
  Knit and submit the generated PDF file on Gradescope.
\end{itemize}

If you have any last-minute trouble knitting, \textbf{DON'T PANIC}.
Submit your Rmd file on time and follow up in office hours or on Piazza
to sort out the PDF.

\hypertarget{for-loops}{%
\section{For-loops}\label{for-loops}}

Watch the ``For-loops'' lecture video.

No exercises for this section.

\hypertarget{loop-indices}{%
\section{Loop Indices}\label{loop-indices}}

Watch the ``Loop Indices'' lecture video.

No exercises for this section.

\hypertarget{while-loops}{%
\section{While-loops}\label{while-loops}}

Watch the ``While-loops'' lecture video.

No exercises for this section.

\hypertarget{preallocation}{%
\section{Preallocation}\label{preallocation}}

Watch the ``Preallocation'' lecture video.

\hypertarget{exercise-1}{%
\subsection{Exercise 1}\label{exercise-1}}

Use the microbenchmark package to benchmark the ``BAD'' and ``GOOD''
example from the lecture video.

Benchmark with three different values of \texttt{n} (testing both the
``BAD'' and ``GOOD'' example for each value). About how much faster is
the ``GOOD'' example?

\textbf{YOUR ANSWER GOES HERE:}

The ``GOOD'' example runs about 10-100 times faster than the ``BAD''
example.

\begin{Shaded}
\begin{Highlighting}[]
\KeywordTok{library}\NormalTok{(microbenchmark)}

\CommentTok{\# BAD, NO PREALLOCATION:}
\NormalTok{bad =}\StringTok{ }\ControlFlowTok{function}\NormalTok{(n) \{}
\NormalTok{   x =}\StringTok{ }\KeywordTok{c}\NormalTok{()}

   \ControlFlowTok{for}\NormalTok{ (i }\ControlFlowTok{in} \DecValTok{1}\OperatorTok{:}\NormalTok{n) \{}
\NormalTok{   x =}\StringTok{ }\KeywordTok{c}\NormalTok{(x, i }\OperatorTok{*}\StringTok{ }\DecValTok{2}\NormalTok{)}
\NormalTok{   \}}
\NormalTok{   x}
\NormalTok{\}}

\CommentTok{\# GOOD:}
\NormalTok{good =}\StringTok{ }\ControlFlowTok{function}\NormalTok{(n) \{}
\NormalTok{   x =}\StringTok{ }\KeywordTok{numeric}\NormalTok{(n)}
   \ControlFlowTok{for}\NormalTok{ (i }\ControlFlowTok{in} \KeywordTok{seq\_len}\NormalTok{(n)) \{}
\NormalTok{   x[i] =}\StringTok{ }\NormalTok{i }\OperatorTok{*}\StringTok{ }\DecValTok{2}
\NormalTok{   \}}
\NormalTok{\}}
\KeywordTok{microbenchmark}\NormalTok{(}\KeywordTok{bad}\NormalTok{(}\FloatTok{5e2}\NormalTok{))}
\end{Highlighting}
\end{Shaded}

\begin{verbatim}
## Unit: microseconds
##      expr   min    lq    mean median    uq    max neval
##  bad(500) 345.1 372.7 532.069  389.8 423.5 5704.6   100
\end{verbatim}

\begin{Shaded}
\begin{Highlighting}[]
\KeywordTok{microbenchmark}\NormalTok{(}\KeywordTok{good}\NormalTok{(}\FloatTok{5e2}\NormalTok{))}
\end{Highlighting}
\end{Shaded}

\begin{verbatim}
## Unit: microseconds
##       expr  min   lq   mean median   uq    max neval
##  good(500) 32.7 35.1 83.694   35.4 35.9 4720.9   100
\end{verbatim}

\begin{Shaded}
\begin{Highlighting}[]
\KeywordTok{microbenchmark}\NormalTok{(}\KeywordTok{bad}\NormalTok{(}\FloatTok{1e3}\NormalTok{))}
\end{Highlighting}
\end{Shaded}

\begin{verbatim}
## Unit: milliseconds
##       expr    min      lq     mean  median      uq    max neval
##  bad(1000) 1.1613 1.23485 1.590201 1.29205 1.36555 5.8847   100
\end{verbatim}

\begin{Shaded}
\begin{Highlighting}[]
\KeywordTok{microbenchmark}\NormalTok{(}\KeywordTok{good}\NormalTok{(}\FloatTok{1e3}\NormalTok{))}
\end{Highlighting}
\end{Shaded}

\begin{verbatim}
## Unit: microseconds
##        expr  min   lq   mean median   uq   max neval
##  good(1000) 61.4 61.9 66.334   62.7 68.2 101.2   100
\end{verbatim}

\begin{Shaded}
\begin{Highlighting}[]
\KeywordTok{microbenchmark}\NormalTok{(}\KeywordTok{bad}\NormalTok{(}\FloatTok{5e3}\NormalTok{))}
\end{Highlighting}
\end{Shaded}

\begin{verbatim}
## Unit: milliseconds
##       expr     min       lq     mean  median       uq     max neval
##  bad(5000) 29.0487 32.18795 33.87132 32.8926 33.96935 93.6655   100
\end{verbatim}

\begin{Shaded}
\begin{Highlighting}[]
\KeywordTok{microbenchmark}\NormalTok{(}\KeywordTok{good}\NormalTok{(}\FloatTok{5e3}\NormalTok{))}
\end{Highlighting}
\end{Shaded}

\begin{verbatim}
## Unit: microseconds
##        expr   min    lq    mean median    uq   max neval
##  good(5000) 309.9 330.7 352.415 340.95 349.8 632.5   100
\end{verbatim}

\hypertarget{loops-example}{%
\section{Loops Example}\label{loops-example}}

Watch the ``Loops Example'' lecture video.

\hypertarget{exercise-2}{%
\subsection{Exercise 2}\label{exercise-2}}

Write a function that returns the first \texttt{n\ +\ 1} positions of a
3-dimensional discrete random walk. Return the \texttt{x}, \texttt{y},
and \texttt{z} coordinates in a data frame with columns \texttt{x},
\texttt{y}, and \texttt{z}. Your function should have a parameter
\texttt{n} that controls the number of steps.

\emph{Hint: For efficiency, use vectors for \texttt{x}, \texttt{y}, and
\texttt{z}. Wait to combine them into a data frame until the very last
line of your function.}

\textbf{YOUR ANSWER GOES HERE:}

\begin{Shaded}
\begin{Highlighting}[]
\NormalTok{random\_walk =}\StringTok{ }\ControlFlowTok{function}\NormalTok{(n) \{}
\NormalTok{   x =}\StringTok{ }\KeywordTok{numeric}\NormalTok{(n }\OperatorTok{+}\StringTok{ }\DecValTok{1}\NormalTok{)}
\NormalTok{   y =}\StringTok{ }\KeywordTok{numeric}\NormalTok{(n }\OperatorTok{+}\StringTok{ }\DecValTok{1}\NormalTok{)}
\NormalTok{   z =}\StringTok{ }\KeywordTok{numeric}\NormalTok{(n }\OperatorTok{+}\StringTok{ }\DecValTok{1}\NormalTok{)}

\NormalTok{   xyz =}\StringTok{ }\KeywordTok{sample}\NormalTok{(}\KeywordTok{c}\NormalTok{(}\DecValTok{0}\NormalTok{, }\DecValTok{1}\NormalTok{, }\DecValTok{2}\NormalTok{), n, }\DataTypeTok{replace =} \OtherTok{TRUE}\NormalTok{)}
\NormalTok{   move =}\StringTok{ }\KeywordTok{sample}\NormalTok{(}\KeywordTok{c}\NormalTok{(}\OperatorTok{{-}}\DecValTok{1}\NormalTok{, }\DecValTok{1}\NormalTok{), n, }\DataTypeTok{replace =} \OtherTok{TRUE}\NormalTok{)}

   \ControlFlowTok{for}\NormalTok{ (i }\ControlFlowTok{in} \KeywordTok{seq\_len}\NormalTok{(n)) \{}
      \ControlFlowTok{if}\NormalTok{ (xyz[i] }\OperatorTok{==}\StringTok{ }\DecValTok{0}\NormalTok{) \{ }\CommentTok{\# x}
\NormalTok{         x[i }\OperatorTok{+}\StringTok{ }\DecValTok{1}\NormalTok{] =}\StringTok{ }\NormalTok{x[i] }\OperatorTok{+}\StringTok{ }\NormalTok{move[i]}
\NormalTok{         y[i }\OperatorTok{+}\StringTok{ }\DecValTok{1}\NormalTok{] =}\StringTok{ }\NormalTok{y[i]}
\NormalTok{         z[i }\OperatorTok{+}\StringTok{ }\DecValTok{1}\NormalTok{] =}\StringTok{ }\NormalTok{z[i]}
\NormalTok{   \} }\ControlFlowTok{else} \ControlFlowTok{if}\NormalTok{ (xyz[i] }\OperatorTok{==}\StringTok{ }\DecValTok{1}\NormalTok{)\{ }\CommentTok{\# y}
\NormalTok{         x[i }\OperatorTok{+}\StringTok{ }\DecValTok{1}\NormalTok{] =}\StringTok{ }\NormalTok{x[i]}
\NormalTok{         y[i }\OperatorTok{+}\StringTok{ }\DecValTok{1}\NormalTok{] =}\StringTok{ }\NormalTok{y[i] }\OperatorTok{+}\StringTok{ }\NormalTok{move[i]}
\NormalTok{         z[i }\OperatorTok{+}\StringTok{ }\DecValTok{1}\NormalTok{] =}\StringTok{ }\NormalTok{z[i]}
\NormalTok{   \} }\ControlFlowTok{else}\NormalTok{ \{ }\CommentTok{\# z}
\NormalTok{         x[i }\OperatorTok{+}\StringTok{ }\DecValTok{1}\NormalTok{] =}\StringTok{ }\NormalTok{x[i]}
\NormalTok{         y[i }\OperatorTok{+}\StringTok{ }\DecValTok{1}\NormalTok{] =}\StringTok{ }\NormalTok{y[i]}
\NormalTok{         z[i }\OperatorTok{+}\StringTok{ }\DecValTok{1}\NormalTok{] =}\StringTok{ }\NormalTok{z[i] }\OperatorTok{+}\StringTok{ }\NormalTok{move[i]}
\NormalTok{   \}}
\NormalTok{   \}}
   \KeywordTok{data.frame}\NormalTok{(x, y, z)}
\NormalTok{\}}
\KeywordTok{random\_walk}\NormalTok{(}\DecValTok{10}\NormalTok{)}
\end{Highlighting}
\end{Shaded}

\begin{verbatim}
##     x  y  z
## 1   0  0  0
## 2   0  0  1
## 3   0 -1  1
## 4   0 -1  0
## 5   0 -2  0
## 6   0 -1  0
## 7  -1 -1  0
## 8  -1 -1 -1
## 9  -1 -2 -1
## 10 -2 -2 -1
## 11 -2 -1 -1
\end{verbatim}

\hypertarget{recursion}{%
\section{Recursion}\label{recursion}}

Watch the ``Recursion'' lecture video.

\hypertarget{exercise-3}{%
\subsection{Exercise 3}\label{exercise-3}}

\begin{enumerate}
\def\labelenumi{\arabic{enumi}.}
\item
  Use the microbenchmark package to benchmark \texttt{find\_fib()} and
  \texttt{find\_fib2()} for \texttt{n} equal to 1 through 30.
\item
  Collect the median timings for each into a data frame with a columns
  \texttt{time}, \texttt{n}, and \texttt{function}. The data frame
  should have 60 rows (30 for each function).
\item
  Use ggplot2 to make a line plot of \texttt{n} versus \texttt{time},
  with a separate line for each \texttt{function}.
\item
  Comment on the the shapes of the lines. Does the computation time grow
  at the same rate (as \texttt{n} increases) for both functions?
\end{enumerate}

\textbf{YOUR ANSWER GOES HERE:}

\begin{enumerate}
\def\labelenumi{\arabic{enumi}.}
\tightlist
\item
  Combined with part 2 below
\item
\end{enumerate}

\begin{Shaded}
\begin{Highlighting}[]
\NormalTok{find\_fib =}\StringTok{ }\ControlFlowTok{function}\NormalTok{(n) \{}
  \ControlFlowTok{if}\NormalTok{ (n }\OperatorTok{==}\StringTok{ }\DecValTok{1} \OperatorTok{|}\StringTok{ }\NormalTok{n }\OperatorTok{==}\StringTok{ }\DecValTok{2}\NormalTok{)}
    \KeywordTok{return}\NormalTok{ (}\DecValTok{1}\NormalTok{)}

  \KeywordTok{find\_fib}\NormalTok{(n }\OperatorTok{{-}}\StringTok{ }\DecValTok{2}\NormalTok{) }\OperatorTok{+}\StringTok{ }\KeywordTok{find\_fib}\NormalTok{(n }\OperatorTok{{-}}\StringTok{ }\DecValTok{1}\NormalTok{)}
\NormalTok{\}}

\NormalTok{find\_fib2 =}\StringTok{ }\ControlFlowTok{function}\NormalTok{(n, }\DataTypeTok{fib =} \KeywordTok{c}\NormalTok{(}\DecValTok{1}\NormalTok{, }\DecValTok{1}\NormalTok{)) \{}
\NormalTok{  len =}\StringTok{ }\KeywordTok{length}\NormalTok{(fib)}
  \ControlFlowTok{if}\NormalTok{ (n }\OperatorTok{\textless{}=}\StringTok{ }\NormalTok{len)}
    \KeywordTok{return}\NormalTok{ (fib[n])}

\NormalTok{  fib =}\StringTok{ }\KeywordTok{c}\NormalTok{(fib, fib[len }\OperatorTok{{-}}\StringTok{ }\DecValTok{1}\NormalTok{] }\OperatorTok{+}\StringTok{ }\NormalTok{fib[len])}
  \KeywordTok{find\_fib2}\NormalTok{(n, fib)}
\NormalTok{\}}

\NormalTok{times =}\StringTok{ }\KeywordTok{numeric}\NormalTok{(}\DecValTok{60}\NormalTok{)}
\NormalTok{n =}\StringTok{ }\KeywordTok{numeric}\NormalTok{(}\DecValTok{60}\NormalTok{)}
\NormalTok{fun =}\StringTok{ }\KeywordTok{character}\NormalTok{(}\DecValTok{60}\NormalTok{)}
\ControlFlowTok{for}\NormalTok{ (i }\ControlFlowTok{in} \DecValTok{1}\OperatorTok{:}\DecValTok{30}\NormalTok{)\{}
\NormalTok{   m1 =}\StringTok{ }\KeywordTok{microbenchmark}\NormalTok{(}\KeywordTok{find\_fib}\NormalTok{(i), }\DataTypeTok{unit =} \StringTok{"us"}\NormalTok{, }\DataTypeTok{times =}\NormalTok{ 30L)}
\NormalTok{   times[i] =}\StringTok{ }\KeywordTok{summary}\NormalTok{(m1)[}\StringTok{"median"}\NormalTok{]}
\NormalTok{   n[i] =}\StringTok{ }\NormalTok{i}
\NormalTok{   fun[i] =}\StringTok{ "find\_fib"}

\NormalTok{   m2 =}\StringTok{ }\KeywordTok{microbenchmark}\NormalTok{(}\KeywordTok{find\_fib2}\NormalTok{(i), }\DataTypeTok{unit =} \StringTok{"us"}\NormalTok{, }\DataTypeTok{times =}\NormalTok{ 30L)}
\NormalTok{   times[i }\OperatorTok{+}\StringTok{ }\DecValTok{30}\NormalTok{] =}\StringTok{ }\KeywordTok{summary}\NormalTok{(m2)[}\StringTok{"median"}\NormalTok{]}
\NormalTok{   n[i }\OperatorTok{+}\StringTok{ }\DecValTok{30}\NormalTok{] =}\StringTok{ }\NormalTok{i}
\NormalTok{   fun[i }\OperatorTok{+}\StringTok{ }\DecValTok{30}\NormalTok{] =}\StringTok{ "find\_fib2"}
\NormalTok{\}}
\NormalTok{times =}\StringTok{ }\KeywordTok{unlist}\NormalTok{(times)}
\NormalTok{df =}\StringTok{ }\KeywordTok{data.frame}\NormalTok{(times, n, fun)}
\end{Highlighting}
\end{Shaded}

\begin{enumerate}
\def\labelenumi{\arabic{enumi}.}
\setcounter{enumi}{2}
\tightlist
\item
\end{enumerate}

\begin{Shaded}
\begin{Highlighting}[]
\KeywordTok{library}\NormalTok{(ggplot2)}
\KeywordTok{ggplot}\NormalTok{(df, }\KeywordTok{aes}\NormalTok{(}\DataTypeTok{x =}\NormalTok{ n, }\DataTypeTok{y =}\NormalTok{ times, }\DataTypeTok{color =}\NormalTok{ fun)) }\OperatorTok{+}\StringTok{ }\KeywordTok{geom\_point}\NormalTok{() }\OperatorTok{+}\StringTok{ }\KeywordTok{geom\_line}\NormalTok{() }\OperatorTok{+}
\StringTok{   }\KeywordTok{labs}\NormalTok{(}\DataTypeTok{title =} \StringTok{"Times vs n"}\NormalTok{, }\DataTypeTok{color =} \StringTok{"Function"}\NormalTok{) }\OperatorTok{+}
\StringTok{   }\KeywordTok{xlab}\NormalTok{(}\StringTok{"n"}\NormalTok{) }\OperatorTok{+}\StringTok{ }\KeywordTok{ylab}\NormalTok{(}\StringTok{"Times (us)"}\NormalTok{)}
\end{Highlighting}
\end{Shaded}

\includegraphics{workbook07_files/figure-latex/unnamed-chunk-4-1.pdf}

\begin{enumerate}
\def\labelenumi{\arabic{enumi}.}
\setcounter{enumi}{3}
\tightlist
\item
  The recursive approach is \textbf{much} slower than the second
  appraoch. With recursion, the time to run grew exponentially with
  \texttt{n}, while the second method seemed linear.
\end{enumerate}

\hypertarget{developing-iterative-code}{%
\section{Developing Iterative Code}\label{developing-iterative-code}}

Watch the ``Developing Iterative Code'' lecture video.

No exercises for this section. All done!

\end{document}
