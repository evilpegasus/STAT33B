% Options for packages loaded elsewhere
\PassOptionsToPackage{unicode}{hyperref}
\PassOptionsToPackage{hyphens}{url}
%
\documentclass[
]{article}
\usepackage{lmodern}
\usepackage{amssymb,amsmath}
\usepackage{ifxetex,ifluatex}
\ifnum 0\ifxetex 1\fi\ifluatex 1\fi=0 % if pdftex
  \usepackage[T1]{fontenc}
  \usepackage[utf8]{inputenc}
  \usepackage{textcomp} % provide euro and other symbols
\else % if luatex or xetex
  \usepackage{unicode-math}
  \defaultfontfeatures{Scale=MatchLowercase}
  \defaultfontfeatures[\rmfamily]{Ligatures=TeX,Scale=1}
\fi
% Use upquote if available, for straight quotes in verbatim environments
\IfFileExists{upquote.sty}{\usepackage{upquote}}{}
\IfFileExists{microtype.sty}{% use microtype if available
  \usepackage[]{microtype}
  \UseMicrotypeSet[protrusion]{basicmath} % disable protrusion for tt fonts
}{}
\makeatletter
\@ifundefined{KOMAClassName}{% if non-KOMA class
  \IfFileExists{parskip.sty}{%
    \usepackage{parskip}
  }{% else
    \setlength{\parindent}{0pt}
    \setlength{\parskip}{6pt plus 2pt minus 1pt}}
}{% if KOMA class
  \KOMAoptions{parskip=half}}
\makeatother
\usepackage{xcolor}
\IfFileExists{xurl.sty}{\usepackage{xurl}}{} % add URL line breaks if available
\IfFileExists{bookmark.sty}{\usepackage{bookmark}}{\usepackage{hyperref}}
\hypersetup{
  pdftitle={Stat 33B - Lecture Notes 1},
  hidelinks,
  pdfcreator={LaTeX via pandoc}}
\urlstyle{same} % disable monospaced font for URLs
\usepackage[margin=1in]{geometry}
\usepackage{color}
\usepackage{fancyvrb}
\newcommand{\VerbBar}{|}
\newcommand{\VERB}{\Verb[commandchars=\\\{\}]}
\DefineVerbatimEnvironment{Highlighting}{Verbatim}{commandchars=\\\{\}}
% Add ',fontsize=\small' for more characters per line
\usepackage{framed}
\definecolor{shadecolor}{RGB}{248,248,248}
\newenvironment{Shaded}{\begin{snugshade}}{\end{snugshade}}
\newcommand{\AlertTok}[1]{\textcolor[rgb]{0.94,0.16,0.16}{#1}}
\newcommand{\AnnotationTok}[1]{\textcolor[rgb]{0.56,0.35,0.01}{\textbf{\textit{#1}}}}
\newcommand{\AttributeTok}[1]{\textcolor[rgb]{0.77,0.63,0.00}{#1}}
\newcommand{\BaseNTok}[1]{\textcolor[rgb]{0.00,0.00,0.81}{#1}}
\newcommand{\BuiltInTok}[1]{#1}
\newcommand{\CharTok}[1]{\textcolor[rgb]{0.31,0.60,0.02}{#1}}
\newcommand{\CommentTok}[1]{\textcolor[rgb]{0.56,0.35,0.01}{\textit{#1}}}
\newcommand{\CommentVarTok}[1]{\textcolor[rgb]{0.56,0.35,0.01}{\textbf{\textit{#1}}}}
\newcommand{\ConstantTok}[1]{\textcolor[rgb]{0.00,0.00,0.00}{#1}}
\newcommand{\ControlFlowTok}[1]{\textcolor[rgb]{0.13,0.29,0.53}{\textbf{#1}}}
\newcommand{\DataTypeTok}[1]{\textcolor[rgb]{0.13,0.29,0.53}{#1}}
\newcommand{\DecValTok}[1]{\textcolor[rgb]{0.00,0.00,0.81}{#1}}
\newcommand{\DocumentationTok}[1]{\textcolor[rgb]{0.56,0.35,0.01}{\textbf{\textit{#1}}}}
\newcommand{\ErrorTok}[1]{\textcolor[rgb]{0.64,0.00,0.00}{\textbf{#1}}}
\newcommand{\ExtensionTok}[1]{#1}
\newcommand{\FloatTok}[1]{\textcolor[rgb]{0.00,0.00,0.81}{#1}}
\newcommand{\FunctionTok}[1]{\textcolor[rgb]{0.00,0.00,0.00}{#1}}
\newcommand{\ImportTok}[1]{#1}
\newcommand{\InformationTok}[1]{\textcolor[rgb]{0.56,0.35,0.01}{\textbf{\textit{#1}}}}
\newcommand{\KeywordTok}[1]{\textcolor[rgb]{0.13,0.29,0.53}{\textbf{#1}}}
\newcommand{\NormalTok}[1]{#1}
\newcommand{\OperatorTok}[1]{\textcolor[rgb]{0.81,0.36,0.00}{\textbf{#1}}}
\newcommand{\OtherTok}[1]{\textcolor[rgb]{0.56,0.35,0.01}{#1}}
\newcommand{\PreprocessorTok}[1]{\textcolor[rgb]{0.56,0.35,0.01}{\textit{#1}}}
\newcommand{\RegionMarkerTok}[1]{#1}
\newcommand{\SpecialCharTok}[1]{\textcolor[rgb]{0.00,0.00,0.00}{#1}}
\newcommand{\SpecialStringTok}[1]{\textcolor[rgb]{0.31,0.60,0.02}{#1}}
\newcommand{\StringTok}[1]{\textcolor[rgb]{0.31,0.60,0.02}{#1}}
\newcommand{\VariableTok}[1]{\textcolor[rgb]{0.00,0.00,0.00}{#1}}
\newcommand{\VerbatimStringTok}[1]{\textcolor[rgb]{0.31,0.60,0.02}{#1}}
\newcommand{\WarningTok}[1]{\textcolor[rgb]{0.56,0.35,0.01}{\textbf{\textit{#1}}}}
\usepackage{graphicx}
\makeatletter
\def\maxwidth{\ifdim\Gin@nat@width>\linewidth\linewidth\else\Gin@nat@width\fi}
\def\maxheight{\ifdim\Gin@nat@height>\textheight\textheight\else\Gin@nat@height\fi}
\makeatother
% Scale images if necessary, so that they will not overflow the page
% margins by default, and it is still possible to overwrite the defaults
% using explicit options in \includegraphics[width, height, ...]{}
\setkeys{Gin}{width=\maxwidth,height=\maxheight,keepaspectratio}
% Set default figure placement to htbp
\makeatletter
\def\fps@figure{htbp}
\makeatother
\setlength{\emergencystretch}{3em} % prevent overfull lines
\providecommand{\tightlist}{%
  \setlength{\itemsep}{0pt}\setlength{\parskip}{0pt}}
\setcounter{secnumdepth}{-\maxdimen} % remove section numbering
\ifluatex
  \usepackage{selnolig}  % disable illegal ligatures
\fi

\title{Stat 33B - Lecture Notes 1}
\author{}
\date{\vspace{-2.5em}August 23, 2020}

\begin{document}
\maketitle

\hypertarget{r-basics}{%
\section{R Basics}\label{r-basics}}

R has a Read-Eval-Print Loop (REPL):

\begin{enumerate}
\def\labelenumi{\arabic{enumi}.}
\tightlist
\item
  Type an expression at the R prompt and hit the enter key.
\item
  R reads the expression.
\item
  R evaluates the expression to compute a result.
\item
  R prints the result in the console.
\item
  R loops back to waiting for you to enter an expression.
\end{enumerate}

This is similar to Python, Julia, Lisp, etc.

R has many built-in functions for doing math and stats.

\hypertarget{getting-help}{%
\section{Getting Help}\label{getting-help}}

R has built-in documentation.

You can use the \texttt{?} command to get help with a specific function:

\begin{Shaded}
\begin{Highlighting}[]
\NormalTok{?sin}
\end{Highlighting}
\end{Shaded}

You can use the \texttt{??} command to search the documentation:

\begin{Shaded}
\begin{Highlighting}[]
\NormalTok{??graphics}
\end{Highlighting}
\end{Shaded}

Strings use single or double quotes (there's no difference).

\begin{Shaded}
\begin{Highlighting}[]
\StringTok{"hi"}
\end{Highlighting}
\end{Shaded}

\begin{verbatim}
## [1] "hi"
\end{verbatim}

\begin{Shaded}
\begin{Highlighting}[]
\StringTok{\textquotesingle{}hi\textquotesingle{}}
\end{Highlighting}
\end{Shaded}

\begin{verbatim}
## [1] "hi"
\end{verbatim}

The help commands work with strings or unquoted names.

\begin{Shaded}
\begin{Highlighting}[]
\NormalTok{?}\StringTok{"+"}
\end{Highlighting}
\end{Shaded}

The \texttt{sessionInfo()} function prints info about your R session:

\begin{Shaded}
\begin{Highlighting}[]
\KeywordTok{sessionInfo}\NormalTok{()}
\end{Highlighting}
\end{Shaded}

\begin{verbatim}
## R version 4.0.2 (2020-06-22)
## Platform: x86_64-w64-mingw32/x64 (64-bit)
## Running under: Windows 10 x64 (build 19041)
## 
## Matrix products: default
## 
## locale:
## [1] LC_COLLATE=English_United States.1252 
## [2] LC_CTYPE=English_United States.1252   
## [3] LC_MONETARY=English_United States.1252
## [4] LC_NUMERIC=C                          
## [5] LC_TIME=English_United States.1252    
## 
## attached base packages:
## [1] stats     graphics  grDevices utils     datasets  methods   base     
## 
## other attached packages:
## [1] ggplot2_3.3.2
## 
## loaded via a namespace (and not attached):
##  [1] knitr_1.29       magrittr_1.5     munsell_0.5.0    colorspace_1.4-1
##  [5] R6_2.4.1         rlang_0.4.7      stringr_1.4.0    tools_4.0.2     
##  [9] grid_4.0.2       gtable_0.3.0     xfun_0.16        tinytex_0.25    
## [13] withr_2.2.0      htmltools_0.5.0  ellipsis_0.3.1   yaml_2.2.1      
## [17] digest_0.6.25    tibble_3.0.3     lifecycle_0.2.0  crayon_1.3.4    
## [21] vctrs_0.3.2      glue_1.4.1       evaluate_0.14    rmarkdown_2.3   
## [25] stringi_1.4.6    compiler_4.0.2   pillar_1.4.6     scales_1.1.1    
## [29] pkgconfig_2.0.3
\end{verbatim}

\hypertarget{order-of-operations}{%
\subsection{Order of Operations}\label{order-of-operations}}

The order of operations in R is similar to math and most programming
languages.

To see the entire order:

\begin{Shaded}
\begin{Highlighting}[]
\NormalTok{?Syntax}
\end{Highlighting}
\end{Shaded}

\hypertarget{functions-calls}{%
\subsection{Functions \& Calls}\label{functions-calls}}

Recall that:

\begin{itemize}
\tightlist
\item
  \textbf{Parameters} are the inputs a function accepts.
\item
  \textbf{Arguments} are the values assigned to parameters in a call.
\end{itemize}

You can set arguments by position or by name:

\begin{Shaded}
\begin{Highlighting}[]
\KeywordTok{log}\NormalTok{(}\DecValTok{10}\NormalTok{)}
\end{Highlighting}
\end{Shaded}

\begin{verbatim}
## [1] 2.302585
\end{verbatim}

\begin{Shaded}
\begin{Highlighting}[]
\NormalTok{?log}

\KeywordTok{log}\NormalTok{(}\DecValTok{10}\NormalTok{, }\DecValTok{2}\NormalTok{)}
\end{Highlighting}
\end{Shaded}

\begin{verbatim}
## [1] 3.321928
\end{verbatim}

\begin{Shaded}
\begin{Highlighting}[]
\KeywordTok{log}\NormalTok{(}\DataTypeTok{base =} \DecValTok{2}\NormalTok{, }\DataTypeTok{x =} \DecValTok{10}\NormalTok{)}
\end{Highlighting}
\end{Shaded}

\begin{verbatim}
## [1] 3.321928
\end{verbatim}

\begin{Shaded}
\begin{Highlighting}[]
\KeywordTok{log}\NormalTok{(}\DataTypeTok{base =} \DecValTok{2}\NormalTok{, }\DecValTok{10}\NormalTok{)}
\end{Highlighting}
\end{Shaded}

\begin{verbatim}
## [1] 3.321928
\end{verbatim}

\hypertarget{copy-on-write-part-1}{%
\subsection{Copy-on-write, Part 1}\label{copy-on-write-part-1}}

In R, most objects are \textbf{copy-on-write}.

That is, if we assign \texttt{x} to \texttt{y}:

\begin{Shaded}
\begin{Highlighting}[]
\NormalTok{x =}\StringTok{ }\DecValTok{3}
\NormalTok{y =}\StringTok{ }\NormalTok{x}
\end{Highlighting}
\end{Shaded}

And then change \texttt{x}:

\begin{Shaded}
\begin{Highlighting}[]
\NormalTok{x =}\StringTok{ }\DecValTok{5}
\end{Highlighting}
\end{Shaded}

Then \texttt{y} remains unchanged:

\begin{Shaded}
\begin{Highlighting}[]
\NormalTok{y}
\end{Highlighting}
\end{Shaded}

\begin{verbatim}
## [1] 3
\end{verbatim}

Originally, \texttt{x} and \texttt{y} referred to the same value in
memory.

When we changed \texttt{x} (a ``write''), R automatically copied the
original value so that \texttt{y} remained the same.

\hypertarget{packages-notebooks}{%
\section{Packages \& Notebooks}\label{packages-notebooks}}

A \textbf{package} is collection of functions and/or data for use in R.

The Comprehensive R Archive Network (\textbf{CRAN}) stores most
user-contributed packages.

You can install packages from CRAN with \texttt{install.packages()}.

For example:

\begin{Shaded}
\begin{Highlighting}[]
\CommentTok{\#install.packages("ggplot2")}
\end{Highlighting}
\end{Shaded}

A package only needs to be installed once.

For maintaining your packages, there are also the functions:

\begin{itemize}
\tightlist
\item
  \texttt{installed.packages()} to list installed packages
\item
  \texttt{remove.packages()} to remove a package
\item
  \texttt{update.packages()} to update ALL packages
\end{itemize}

\hypertarget{loading-packages}{%
\subsection{Loading Packages}\label{loading-packages}}

The \texttt{library()} function loads an installed package:

\begin{Shaded}
\begin{Highlighting}[]
\KeywordTok{library}\NormalTok{(ggplot2)}
\end{Highlighting}
\end{Shaded}

Only load the packages you actually need.

You'll have to reload the packages each time you restart R.

\hypertarget{notebooks}{%
\subsection{Notebooks}\label{notebooks}}

Two typical ways to save R code:

\begin{itemize}
\tightlist
\item
  R script (.R file)
\item
  R notebook (.Rmd file)
\end{itemize}

R scripts are simpler:

\begin{itemize}
\tightlist
\item
  No extra packages required
\item
  Ideal for developing software
\end{itemize}

R notebooks are richer:

\begin{itemize}
\tightlist
\item
  Can store formatted text and code
\item
  Can be converted to HTML, DOCX, and PDF
\item
  Ideal for data analyses and presentations
\end{itemize}

R notebooks require the \texttt{rmarkdown} package:

\begin{Shaded}
\begin{Highlighting}[]
\KeywordTok{install.packages}\NormalTok{(}\StringTok{"rmarkdown"}\NormalTok{)}
\end{Highlighting}
\end{Shaded}

Generating a report from an R notebook is called \textbf{knitting}.

\hypertarget{tinytex}{%
\subsection{TinyTeX}\label{tinytex}}

If you want to knit PDFs from R notebooks, you also need LaTeX.

LaTeX is programming language for typesetting books.

The \texttt{tinytex} package aims to make installing LaTeX easy.

First, install \texttt{tinytex}:

\begin{Shaded}
\begin{Highlighting}[]
\KeywordTok{install.packages}\NormalTok{(}\StringTok{"tinytex"}\NormalTok{)}
\end{Highlighting}
\end{Shaded}

Second, tell \texttt{tinytex} to install LaTeX:

\begin{Shaded}
\begin{Highlighting}[]
\KeywordTok{library}\NormalTok{(tinytex)}
\KeywordTok{install\_tinytex}\NormalTok{()}
\end{Highlighting}
\end{Shaded}

This may take a while, and you may need administrator permissions.

Finally, restart R and try knitting an R notebook.

Remember that the output type must be \texttt{pdf\_document}.

\hypertarget{vectors}{%
\section{Vectors}\label{vectors}}

R has no concept of scalars or arrays.

R's atomic data type is the \textbf{vector}, an ordered container for 0
or more elements.

Vector elements must all have the same data type.

The \texttt{c()} function combines vectors:

\begin{Shaded}
\begin{Highlighting}[]
\NormalTok{x =}\StringTok{ }\KeywordTok{c}\NormalTok{(}\DecValTok{5}\NormalTok{, }\DecValTok{7}\NormalTok{, }\DecValTok{1}\NormalTok{)}
\NormalTok{x}
\end{Highlighting}
\end{Shaded}

\begin{verbatim}
## [1] 5 7 1
\end{verbatim}

\begin{Shaded}
\begin{Highlighting}[]
\KeywordTok{c}\NormalTok{(x, }\DecValTok{1}\NormalTok{)}
\end{Highlighting}
\end{Shaded}

\begin{verbatim}
## [1] 5 7 1 1
\end{verbatim}

\begin{Shaded}
\begin{Highlighting}[]
\KeywordTok{c}\NormalTok{(}\StringTok{"hi"}\NormalTok{, }\StringTok{"hello"}\NormalTok{)}
\end{Highlighting}
\end{Shaded}

\begin{verbatim}
## [1] "hi"    "hello"
\end{verbatim}

\begin{Shaded}
\begin{Highlighting}[]
\StringTok{"hi"}
\end{Highlighting}
\end{Shaded}

\begin{verbatim}
## [1] "hi"
\end{verbatim}

\begin{Shaded}
\begin{Highlighting}[]
\KeywordTok{c}\NormalTok{(}\StringTok{"hi"}\NormalTok{, }\DecValTok{1}\NormalTok{)}
\end{Highlighting}
\end{Shaded}

\begin{verbatim}
## [1] "hi" "1"
\end{verbatim}

\hypertarget{vectorization}{%
\subsection{Vectorization}\label{vectorization}}

A \textbf{vectorized} function is one that is applied element-by-element
when passed a vector argument.

Many R functions are vectorized:

\begin{Shaded}
\begin{Highlighting}[]
\KeywordTok{c}\NormalTok{(}\KeywordTok{sin}\NormalTok{(}\DecValTok{0}\NormalTok{), }\KeywordTok{sin}\NormalTok{(}\DecValTok{1}\NormalTok{), }\KeywordTok{sin}\NormalTok{(}\DecValTok{2}\NormalTok{))}
\end{Highlighting}
\end{Shaded}

\begin{verbatim}
## [1] 0.0000000 0.8414710 0.9092974
\end{verbatim}

\begin{Shaded}
\begin{Highlighting}[]
\NormalTok{x =}\StringTok{ }\KeywordTok{c}\NormalTok{(}\DecValTok{0}\NormalTok{, }\DecValTok{1}\NormalTok{, }\DecValTok{2}\NormalTok{)}
\KeywordTok{sin}\NormalTok{(x)}
\end{Highlighting}
\end{Shaded}

\begin{verbatim}
## [1] 0.0000000 0.8414710 0.9092974
\end{verbatim}

\begin{Shaded}
\begin{Highlighting}[]
\CommentTok{\# NOT VECTORIZED:}
\KeywordTok{mean}\NormalTok{(x)}
\end{Highlighting}
\end{Shaded}

\begin{verbatim}
## [1] 1
\end{verbatim}

Vectorization is the fastest kind of iteration in R.

\hypertarget{indexing}{%
\subsection{Indexing}\label{indexing}}

In R, indexes start at 1.

Use the square bracket \texttt{{[}} to access elements of a vector:

\begin{Shaded}
\begin{Highlighting}[]
\NormalTok{x =}\StringTok{ }\KeywordTok{c}\NormalTok{(}\DecValTok{1}\NormalTok{, }\DecValTok{3}\NormalTok{, }\DecValTok{7}\NormalTok{)}
\NormalTok{x[}\DecValTok{2}\NormalTok{]}
\end{Highlighting}
\end{Shaded}

\begin{verbatim}
## [1] 3
\end{verbatim}

\begin{Shaded}
\begin{Highlighting}[]
\NormalTok{x[}\DecValTok{6}\NormalTok{]}
\end{Highlighting}
\end{Shaded}

\begin{verbatim}
## [1] NA
\end{verbatim}

You can use a vector as an index:

\begin{Shaded}
\begin{Highlighting}[]
\NormalTok{x[}\KeywordTok{c}\NormalTok{(}\DecValTok{1}\NormalTok{, }\DecValTok{1}\NormalTok{, }\DecValTok{2}\NormalTok{)]}
\end{Highlighting}
\end{Shaded}

\begin{verbatim}
## [1] 1 1 3
\end{verbatim}

\hypertarget{copy-on-write-part-2}{%
\subsection{Copy-on-write, Part 2}\label{copy-on-write-part-2}}

The copy-on-write rule applies to vectors.

For example:

\begin{Shaded}
\begin{Highlighting}[]
\NormalTok{x =}\StringTok{ }\KeywordTok{c}\NormalTok{(}\DecValTok{10}\NormalTok{, }\DecValTok{20}\NormalTok{, }\DecValTok{30}\NormalTok{)}
\NormalTok{y =}\StringTok{ }\NormalTok{x}
\NormalTok{x}
\end{Highlighting}
\end{Shaded}

\begin{verbatim}
## [1] 10 20 30
\end{verbatim}

\begin{Shaded}
\begin{Highlighting}[]
\NormalTok{y}
\end{Highlighting}
\end{Shaded}

\begin{verbatim}
## [1] 10 20 30
\end{verbatim}

\begin{Shaded}
\begin{Highlighting}[]
\NormalTok{x[}\DecValTok{1}\NormalTok{] =}\StringTok{ }\DecValTok{15}
\NormalTok{x}
\end{Highlighting}
\end{Shaded}

\begin{verbatim}
## [1] 15 20 30
\end{verbatim}

\begin{Shaded}
\begin{Highlighting}[]
\NormalTok{y}
\end{Highlighting}
\end{Shaded}

\begin{verbatim}
## [1] 10 20 30
\end{verbatim}

This is different from languages like C and Python.

\end{document}
