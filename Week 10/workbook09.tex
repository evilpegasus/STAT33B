% Options for packages loaded elsewhere
\PassOptionsToPackage{unicode}{hyperref}
\PassOptionsToPackage{hyphens}{url}
%
\documentclass[
]{article}
\usepackage{lmodern}
\usepackage{amssymb,amsmath}
\usepackage{ifxetex,ifluatex}
\ifnum 0\ifxetex 1\fi\ifluatex 1\fi=0 % if pdftex
  \usepackage[T1]{fontenc}
  \usepackage[utf8]{inputenc}
  \usepackage{textcomp} % provide euro and other symbols
\else % if luatex or xetex
  \usepackage{unicode-math}
  \defaultfontfeatures{Scale=MatchLowercase}
  \defaultfontfeatures[\rmfamily]{Ligatures=TeX,Scale=1}
\fi
% Use upquote if available, for straight quotes in verbatim environments
\IfFileExists{upquote.sty}{\usepackage{upquote}}{}
\IfFileExists{microtype.sty}{% use microtype if available
  \usepackage[]{microtype}
  \UseMicrotypeSet[protrusion]{basicmath} % disable protrusion for tt fonts
}{}
\makeatletter
\@ifundefined{KOMAClassName}{% if non-KOMA class
  \IfFileExists{parskip.sty}{%
    \usepackage{parskip}
  }{% else
    \setlength{\parindent}{0pt}
    \setlength{\parskip}{6pt plus 2pt minus 1pt}}
}{% if KOMA class
  \KOMAoptions{parskip=half}}
\makeatother
\usepackage{xcolor}
\IfFileExists{xurl.sty}{\usepackage{xurl}}{} % add URL line breaks if available
\IfFileExists{bookmark.sty}{\usepackage{bookmark}}{\usepackage{hyperref}}
\hypersetup{
  pdftitle={STAT 33B Workbook 9},
  pdfauthor={Ming Fong (3035619833)},
  hidelinks,
  pdfcreator={LaTeX via pandoc}}
\urlstyle{same} % disable monospaced font for URLs
\usepackage[margin=1in]{geometry}
\usepackage{color}
\usepackage{fancyvrb}
\newcommand{\VerbBar}{|}
\newcommand{\VERB}{\Verb[commandchars=\\\{\}]}
\DefineVerbatimEnvironment{Highlighting}{Verbatim}{commandchars=\\\{\}}
% Add ',fontsize=\small' for more characters per line
\usepackage{framed}
\definecolor{shadecolor}{RGB}{248,248,248}
\newenvironment{Shaded}{\begin{snugshade}}{\end{snugshade}}
\newcommand{\AlertTok}[1]{\textcolor[rgb]{0.94,0.16,0.16}{#1}}
\newcommand{\AnnotationTok}[1]{\textcolor[rgb]{0.56,0.35,0.01}{\textbf{\textit{#1}}}}
\newcommand{\AttributeTok}[1]{\textcolor[rgb]{0.77,0.63,0.00}{#1}}
\newcommand{\BaseNTok}[1]{\textcolor[rgb]{0.00,0.00,0.81}{#1}}
\newcommand{\BuiltInTok}[1]{#1}
\newcommand{\CharTok}[1]{\textcolor[rgb]{0.31,0.60,0.02}{#1}}
\newcommand{\CommentTok}[1]{\textcolor[rgb]{0.56,0.35,0.01}{\textit{#1}}}
\newcommand{\CommentVarTok}[1]{\textcolor[rgb]{0.56,0.35,0.01}{\textbf{\textit{#1}}}}
\newcommand{\ConstantTok}[1]{\textcolor[rgb]{0.00,0.00,0.00}{#1}}
\newcommand{\ControlFlowTok}[1]{\textcolor[rgb]{0.13,0.29,0.53}{\textbf{#1}}}
\newcommand{\DataTypeTok}[1]{\textcolor[rgb]{0.13,0.29,0.53}{#1}}
\newcommand{\DecValTok}[1]{\textcolor[rgb]{0.00,0.00,0.81}{#1}}
\newcommand{\DocumentationTok}[1]{\textcolor[rgb]{0.56,0.35,0.01}{\textbf{\textit{#1}}}}
\newcommand{\ErrorTok}[1]{\textcolor[rgb]{0.64,0.00,0.00}{\textbf{#1}}}
\newcommand{\ExtensionTok}[1]{#1}
\newcommand{\FloatTok}[1]{\textcolor[rgb]{0.00,0.00,0.81}{#1}}
\newcommand{\FunctionTok}[1]{\textcolor[rgb]{0.00,0.00,0.00}{#1}}
\newcommand{\ImportTok}[1]{#1}
\newcommand{\InformationTok}[1]{\textcolor[rgb]{0.56,0.35,0.01}{\textbf{\textit{#1}}}}
\newcommand{\KeywordTok}[1]{\textcolor[rgb]{0.13,0.29,0.53}{\textbf{#1}}}
\newcommand{\NormalTok}[1]{#1}
\newcommand{\OperatorTok}[1]{\textcolor[rgb]{0.81,0.36,0.00}{\textbf{#1}}}
\newcommand{\OtherTok}[1]{\textcolor[rgb]{0.56,0.35,0.01}{#1}}
\newcommand{\PreprocessorTok}[1]{\textcolor[rgb]{0.56,0.35,0.01}{\textit{#1}}}
\newcommand{\RegionMarkerTok}[1]{#1}
\newcommand{\SpecialCharTok}[1]{\textcolor[rgb]{0.00,0.00,0.00}{#1}}
\newcommand{\SpecialStringTok}[1]{\textcolor[rgb]{0.31,0.60,0.02}{#1}}
\newcommand{\StringTok}[1]{\textcolor[rgb]{0.31,0.60,0.02}{#1}}
\newcommand{\VariableTok}[1]{\textcolor[rgb]{0.00,0.00,0.00}{#1}}
\newcommand{\VerbatimStringTok}[1]{\textcolor[rgb]{0.31,0.60,0.02}{#1}}
\newcommand{\WarningTok}[1]{\textcolor[rgb]{0.56,0.35,0.01}{\textbf{\textit{#1}}}}
\usepackage{graphicx}
\makeatletter
\def\maxwidth{\ifdim\Gin@nat@width>\linewidth\linewidth\else\Gin@nat@width\fi}
\def\maxheight{\ifdim\Gin@nat@height>\textheight\textheight\else\Gin@nat@height\fi}
\makeatother
% Scale images if necessary, so that they will not overflow the page
% margins by default, and it is still possible to overwrite the defaults
% using explicit options in \includegraphics[width, height, ...]{}
\setkeys{Gin}{width=\maxwidth,height=\maxheight,keepaspectratio}
% Set default figure placement to htbp
\makeatletter
\def\fps@figure{htbp}
\makeatother
\setlength{\emergencystretch}{3em} % prevent overfull lines
\providecommand{\tightlist}{%
  \setlength{\itemsep}{0pt}\setlength{\parskip}{0pt}}
\setcounter{secnumdepth}{-\maxdimen} % remove section numbering
\ifluatex
  \usepackage{selnolig}  % disable illegal ligatures
\fi

\title{STAT 33B Workbook 9}
\author{Ming Fong (3035619833)}
\date{Oct 29, 2020}

\begin{document}
\maketitle

This workbook is due \textbf{Oct 29, 2020} by 11:59pm PT.

The workbook is organized into sections that correspond to the lecture
videos for the week. Watch a video, then do the corresponding exercises
\emph{before} moving on to the next video.

Workbooks are graded for completeness, so as long as you make a clear
effort to solve each problem, you'll get full credit. That said, make
sure you understand the concepts here, because they're likely to
reappear in homeworks, quizzes, and later lectures.

As you work, write your answers in this notebook. Answer questions with
complete sentences, and put code in code chunks. You can make as many
new code chunks as you like.

In the notebook, you can run the line of code where the cursor is by
pressing \texttt{Ctrl} + \texttt{Enter} on Windows or \texttt{Cmd} +
\texttt{Enter} on Mac OS X. You can run an entire code chunk by clicking
on the green arrow in the upper right corner of the code chunk.

Please do not delete the exercises already in this notebook, because it
may interfere with our grading tools.

You need to submit your work in two places:

\begin{itemize}
\tightlist
\item
  Submit this Rmd file with your edits on bCourses.
\item
  Knit and submit the generated PDF file on Gradescope.
\end{itemize}

If you have any last-minute trouble knitting, \textbf{DON'T PANIC}.
Submit your Rmd file on time and follow up in office hours or on Piazza
to sort out the PDF.

\hypertarget{environments}{%
\section{Environments}\label{environments}}

Watch the ``Environments'' lecture video.

No exercises for this section.

\hypertarget{variable-lookup-part-2}{%
\section{Variable Lookup, Part 2}\label{variable-lookup-part-2}}

Watch the ``Variable Lookup, Part 2'' lecture video.

No exercises for this section.

\hypertarget{the-search-path}{%
\section{The Search Path}\label{the-search-path}}

Watch the ``The Search Path'' lecture video.

\hypertarget{exercise-1}{%
\subsection{Exercise 1}\label{exercise-1}}

Create a function called \texttt{locate} that finds and returns the
first environment (in a chain of environments) that contains a given
variable name. Your function should have a parameter \texttt{name} for
the quoted variable name and a parameter \texttt{env} for the initial
environment to search.

If the variable is not present in any of the environments in the chain,
your function should return the empty environment.

\emph{Hint 1: Use \texttt{exists} to check whether the variable exists
in \texttt{env} (and not its ancestors). If it does, return
\texttt{env}. If it does not, set \texttt{env} to be its own parent and
repeat this process.}

\emph{Hint 2: You can use \texttt{identical} to check if two
environments are equal.}

\textbf{YOUR ANSWER GOES HERE:}

\begin{Shaded}
\begin{Highlighting}[]
\NormalTok{locate =}\StringTok{ }\ControlFlowTok{function}\NormalTok{(name, env) \{}
\NormalTok{   e =}\StringTok{ }\NormalTok{env}
   \ControlFlowTok{while}\NormalTok{(}\OperatorTok{!}\KeywordTok{exists}\NormalTok{(name, e, }\DataTypeTok{inherits =} \OtherTok{FALSE}\NormalTok{)) \{}
      \ControlFlowTok{if}\NormalTok{(}\KeywordTok{identical}\NormalTok{(e, }\KeywordTok{emptyenv}\NormalTok{())) \{}
         \KeywordTok{return}\NormalTok{(e)}
\NormalTok{      \}}
\NormalTok{      e =}\StringTok{ }\KeywordTok{parent.env}\NormalTok{(e)}
\NormalTok{   \}}
\NormalTok{   e}
\NormalTok{\}}
\end{Highlighting}
\end{Shaded}

\hypertarget{exercise-2}{%
\subsection{Exercise 2}\label{exercise-2}}

This code produces an environment \texttt{e} with several ancestors:

\begin{Shaded}
\begin{Highlighting}[]
\NormalTok{e =}\StringTok{ }\KeywordTok{new.env}\NormalTok{()}
\NormalTok{e}\OperatorTok{$}\NormalTok{c =}\StringTok{ }\DecValTok{42}
\NormalTok{e =}\StringTok{ }\KeywordTok{new.env}\NormalTok{(}\DataTypeTok{parent =}\NormalTok{ e)}
\NormalTok{e}\OperatorTok{$}\NormalTok{a =}\StringTok{ "33a"}
\NormalTok{e}\OperatorTok{$}\NormalTok{b =}\StringTok{ "33b"}
\NormalTok{e}\OperatorTok{$}\NormalTok{c =}\StringTok{ "33ab"}
\NormalTok{e =}\StringTok{ }\KeywordTok{new.env}\NormalTok{(}\DataTypeTok{parent =}\NormalTok{ e)}
\NormalTok{e}\OperatorTok{$}\NormalTok{x =}\StringTok{ }\DecValTok{8}
\end{Highlighting}
\end{Shaded}

Test your \texttt{locate} function on \texttt{e} by:

\begin{enumerate}
\def\labelenumi{\arabic{enumi}.}
\item
  Locating \texttt{"c"}. Your result \texttt{result} should have
  \texttt{result\$c} equal to \texttt{"33ab"}.
\item
  Locating \texttt{"zzz"}. Your result \texttt{result} should be the
  empty environment.
\item
  Locating \texttt{"e"}. Your result \texttt{result} should be the
  global environment.
\item
  Locating \texttt{"show"}. Which built-in package provides this
  function?
\end{enumerate}

\textbf{YOUR ANSWER GOES HERE:}

\hypertarget{part-1}{%
\subsubsection{Part 1}\label{part-1}}

\begin{Shaded}
\begin{Highlighting}[]
\KeywordTok{locate}\NormalTok{(}\StringTok{"c"}\NormalTok{, e)}\OperatorTok{$}\NormalTok{c}
\end{Highlighting}
\end{Shaded}

\begin{verbatim}
## [1] "33ab"
\end{verbatim}

\hypertarget{part-2}{%
\subsubsection{Part 2}\label{part-2}}

\begin{Shaded}
\begin{Highlighting}[]
\KeywordTok{locate}\NormalTok{(}\StringTok{"zzz"}\NormalTok{, e)}
\end{Highlighting}
\end{Shaded}

\begin{verbatim}
## <environment: R_EmptyEnv>
\end{verbatim}

\hypertarget{part-3}{%
\subsubsection{Part 3}\label{part-3}}

\begin{Shaded}
\begin{Highlighting}[]
\KeywordTok{locate}\NormalTok{(}\StringTok{"e"}\NormalTok{, e)}
\end{Highlighting}
\end{Shaded}

\begin{verbatim}
## <environment: R_GlobalEnv>
\end{verbatim}

\hypertarget{part-4}{%
\subsubsection{Part 4}\label{part-4}}

The \texttt{methods} package provides \texttt{show()}

\begin{Shaded}
\begin{Highlighting}[]
\KeywordTok{locate}\NormalTok{(}\StringTok{"show"}\NormalTok{, e)}
\end{Highlighting}
\end{Shaded}

\begin{verbatim}
## <environment: package:methods>
## attr(,"name")
## [1] "package:methods"
## attr(,"path")
## [1] "C:/Program Files/R/R-4.0.2/library/methods"
\end{verbatim}

\hypertarget{the-colon-operators}{%
\section{The Colon Operators}\label{the-colon-operators}}

Watch the ``The Colon Operators'' lecture video.

No exercises for this section.

\hypertarget{closures}{%
\section{Closures}\label{closures}}

Watch the ``Closures'' lecture video.

\hypertarget{exercise-3}{%
\subsection{Exercise 3}\label{exercise-3}}

Recall the \texttt{find\_fib2} function (from week 8) for computing
Fibonacci numbers:

\begin{Shaded}
\begin{Highlighting}[]
\NormalTok{find\_fib2 =}\StringTok{ }\ControlFlowTok{function}\NormalTok{(n, }\DataTypeTok{fib =} \KeywordTok{c}\NormalTok{(}\DecValTok{1}\NormalTok{, }\DecValTok{1}\NormalTok{)) \{}
\NormalTok{  len =}\StringTok{ }\KeywordTok{length}\NormalTok{(fib)}
  \ControlFlowTok{if}\NormalTok{ (n }\OperatorTok{\textless{}=}\StringTok{ }\NormalTok{len)}
    \KeywordTok{return}\NormalTok{ (fib[n])}
  
\NormalTok{  fib =}\StringTok{ }\KeywordTok{c}\NormalTok{(fib, fib[len }\OperatorTok{{-}}\StringTok{ }\DecValTok{1}\NormalTok{] }\OperatorTok{+}\StringTok{ }\NormalTok{fib[len])}
  \KeywordTok{Recall}\NormalTok{(n, fib)}
\NormalTok{\}}
\end{Highlighting}
\end{Shaded}

The key to computing Fibonacci numbers efficiently is to keep a record
of the numbers that have already been computed. The \texttt{find\_fib2}
function does this by passing the record of computed numbers on through
the \texttt{fib} parameter.

An alternative to passing the record through a parameter is to store the
record in the function's enclosing environment. Write a function
\texttt{find\_fib3} that does this.

Your function should still have a parameter \texttt{n} and return the
\texttt{n}-th Fibonacci number. Your function should \textbf{NOT} have a
parameter \texttt{fib}.

Test your function by computing \texttt{fib(40)}. If your function is
working correctly, it should be able to compute this number in less than
5 seconds, and the number should be \texttt{102334155}.

\emph{Hint 1: Create a factory function \texttt{make\_find\_fib3} to
provide the enclosing environment. The factory function should not have
any parameters, and should return your \texttt{find\_fib3} function.}

\emph{Note: Using the enclosing environment to store values that have
been computed is called ``memoization''. Memoization is a useful
strategy for improving efficiency in many programming problems.}

\textbf{YOUR ANSWER GOES HERE:}

\begin{Shaded}
\begin{Highlighting}[]
\NormalTok{make\_find\_fib3 =}\StringTok{ }\ControlFlowTok{function}\NormalTok{(n) \{}
\NormalTok{   fib\_}\DecValTok{1}\NormalTok{ =}\StringTok{ }\DecValTok{1}   \CommentTok{\# fib(n {-} 1)}
\NormalTok{   fib\_}\DecValTok{2}\NormalTok{ =}\StringTok{ }\DecValTok{1}   \CommentTok{\# fib(n {-} 2)}
   \ControlFlowTok{function}\NormalTok{(n) \{}
      \ControlFlowTok{if}\NormalTok{ (n }\OperatorTok{==}\StringTok{ }\DecValTok{1} \OperatorTok{||}\StringTok{ }\NormalTok{n }\OperatorTok{==}\StringTok{ }\DecValTok{2}\NormalTok{) \{}
         \KeywordTok{return}\NormalTok{(}\DecValTok{1}\NormalTok{)}
\NormalTok{      \}}
      \ControlFlowTok{for}\NormalTok{ (i }\ControlFlowTok{in} \KeywordTok{seq\_len}\NormalTok{(n }\OperatorTok{{-}}\StringTok{ }\DecValTok{2}\NormalTok{)) \{}
\NormalTok{         temp =}\StringTok{ }\NormalTok{fib\_}\DecValTok{1} \OperatorTok{+}\StringTok{ }\NormalTok{fib\_}\DecValTok{2}
\NormalTok{         fib\_}\DecValTok{2}\NormalTok{ =}\StringTok{ }\NormalTok{fib\_}\DecValTok{1}
\NormalTok{         fib\_}\DecValTok{1}\NormalTok{ =}\StringTok{ }\NormalTok{temp}
\NormalTok{      \}}
\NormalTok{      temp}
\NormalTok{   \}}
\NormalTok{\}}

\NormalTok{find\_fib3 =}\StringTok{ }\KeywordTok{make\_find\_fib3}\NormalTok{()}
\KeywordTok{find\_fib3}\NormalTok{(}\DecValTok{40}\NormalTok{)}
\end{Highlighting}
\end{Shaded}

\begin{verbatim}
## [1] 102334155
\end{verbatim}

\hypertarget{exercise-4}{%
\subsection{Exercise 4}\label{exercise-4}}

\begin{enumerate}
\def\labelenumi{\arabic{enumi}.}
\item
  Enclosing environments persist between calls. Explain how you think
  this will affect the speed and memory usage of \texttt{find\_fib3}
  compared to \texttt{find\_fib2}. What are the advantages and
  disadvantages of the two different functions?

  \emph{Hint: How long does it take to compute \texttt{find\_fib3(40)}
  the first time? The second time? The third time?}
\item
  Use the microbenchmark package to benchmark \texttt{find\_fib2} and
  \texttt{find\_fib3} for \texttt{n} equal to 30 and \texttt{n} equal to
  40. Which function is faster?
\end{enumerate}

\textbf{YOUR ANSWER GOES HERE:}

\hypertarget{part-1-1}{%
\subsubsection{Part 1}\label{part-1-1}}

\texttt{find\_fib3} will use more memory than \texttt{find\_fib2} but
will be much faster. Usually speed is preferred, so \texttt{find\_fib3}
would be better.

\hypertarget{part-2-1}{%
\subsubsection{Part 2}\label{part-2-1}}

\texttt{find\_fib3} is much faster (about 20x faster in these two
tests).

\begin{Shaded}
\begin{Highlighting}[]
\KeywordTok{library}\NormalTok{(microbenchmark)}
\NormalTok{n =}\StringTok{ }\DecValTok{30}
\KeywordTok{microbenchmark}\NormalTok{(}\KeywordTok{find\_fib2}\NormalTok{(n))}
\end{Highlighting}
\end{Shaded}

\begin{verbatim}
## Unit: microseconds
##          expr  min   lq   mean median    uq    max neval
##  find_fib2(n) 51.9 53.5 117.36  54.35 55.25 6288.7   100
\end{verbatim}

\begin{Shaded}
\begin{Highlighting}[]
\KeywordTok{microbenchmark}\NormalTok{(}\KeywordTok{find\_fib3}\NormalTok{(n))}
\end{Highlighting}
\end{Shaded}

\begin{verbatim}
## Unit: microseconds
##          expr min  lq  mean median  uq  max neval
##  find_fib3(n) 2.5 2.6 2.818    2.6 2.7 12.9   100
\end{verbatim}

\begin{Shaded}
\begin{Highlighting}[]
\NormalTok{n =}\StringTok{ }\DecValTok{40}
\KeywordTok{microbenchmark}\NormalTok{(}\KeywordTok{find\_fib2}\NormalTok{(n))}
\end{Highlighting}
\end{Shaded}

\begin{verbatim}
## Unit: microseconds
##          expr  min   lq   mean median    uq   max neval
##  find_fib2(n) 69.5 73.7 80.048   76.8 79.45 245.4   100
\end{verbatim}

\begin{Shaded}
\begin{Highlighting}[]
\KeywordTok{microbenchmark}\NormalTok{(}\KeywordTok{find\_fib3}\NormalTok{(n))}
\end{Highlighting}
\end{Shaded}

\begin{verbatim}
## Unit: microseconds
##          expr min  lq  mean median  uq max neval
##  find_fib3(n) 3.4 3.5 3.763    3.5 3.6  21   100
\end{verbatim}

\end{document}
