% Options for packages loaded elsewhere
\PassOptionsToPackage{unicode}{hyperref}
\PassOptionsToPackage{hyphens}{url}
%
\documentclass[
]{article}
\usepackage{lmodern}
\usepackage{amssymb,amsmath}
\usepackage{ifxetex,ifluatex}
\ifnum 0\ifxetex 1\fi\ifluatex 1\fi=0 % if pdftex
  \usepackage[T1]{fontenc}
  \usepackage[utf8]{inputenc}
  \usepackage{textcomp} % provide euro and other symbols
\else % if luatex or xetex
  \usepackage{unicode-math}
  \defaultfontfeatures{Scale=MatchLowercase}
  \defaultfontfeatures[\rmfamily]{Ligatures=TeX,Scale=1}
\fi
% Use upquote if available, for straight quotes in verbatim environments
\IfFileExists{upquote.sty}{\usepackage{upquote}}{}
\IfFileExists{microtype.sty}{% use microtype if available
  \usepackage[]{microtype}
  \UseMicrotypeSet[protrusion]{basicmath} % disable protrusion for tt fonts
}{}
\makeatletter
\@ifundefined{KOMAClassName}{% if non-KOMA class
  \IfFileExists{parskip.sty}{%
    \usepackage{parskip}
  }{% else
    \setlength{\parindent}{0pt}
    \setlength{\parskip}{6pt plus 2pt minus 1pt}}
}{% if KOMA class
  \KOMAoptions{parskip=half}}
\makeatother
\usepackage{xcolor}
\IfFileExists{xurl.sty}{\usepackage{xurl}}{} % add URL line breaks if available
\IfFileExists{bookmark.sty}{\usepackage{bookmark}}{\usepackage{hyperref}}
\hypersetup{
  pdftitle={STAT 33B Workbook 3},
  pdfauthor={Ming Fong (3035619833)},
  hidelinks,
  pdfcreator={LaTeX via pandoc}}
\urlstyle{same} % disable monospaced font for URLs
\usepackage[margin=1in]{geometry}
\usepackage{color}
\usepackage{fancyvrb}
\newcommand{\VerbBar}{|}
\newcommand{\VERB}{\Verb[commandchars=\\\{\}]}
\DefineVerbatimEnvironment{Highlighting}{Verbatim}{commandchars=\\\{\}}
% Add ',fontsize=\small' for more characters per line
\usepackage{framed}
\definecolor{shadecolor}{RGB}{248,248,248}
\newenvironment{Shaded}{\begin{snugshade}}{\end{snugshade}}
\newcommand{\AlertTok}[1]{\textcolor[rgb]{0.94,0.16,0.16}{#1}}
\newcommand{\AnnotationTok}[1]{\textcolor[rgb]{0.56,0.35,0.01}{\textbf{\textit{#1}}}}
\newcommand{\AttributeTok}[1]{\textcolor[rgb]{0.77,0.63,0.00}{#1}}
\newcommand{\BaseNTok}[1]{\textcolor[rgb]{0.00,0.00,0.81}{#1}}
\newcommand{\BuiltInTok}[1]{#1}
\newcommand{\CharTok}[1]{\textcolor[rgb]{0.31,0.60,0.02}{#1}}
\newcommand{\CommentTok}[1]{\textcolor[rgb]{0.56,0.35,0.01}{\textit{#1}}}
\newcommand{\CommentVarTok}[1]{\textcolor[rgb]{0.56,0.35,0.01}{\textbf{\textit{#1}}}}
\newcommand{\ConstantTok}[1]{\textcolor[rgb]{0.00,0.00,0.00}{#1}}
\newcommand{\ControlFlowTok}[1]{\textcolor[rgb]{0.13,0.29,0.53}{\textbf{#1}}}
\newcommand{\DataTypeTok}[1]{\textcolor[rgb]{0.13,0.29,0.53}{#1}}
\newcommand{\DecValTok}[1]{\textcolor[rgb]{0.00,0.00,0.81}{#1}}
\newcommand{\DocumentationTok}[1]{\textcolor[rgb]{0.56,0.35,0.01}{\textbf{\textit{#1}}}}
\newcommand{\ErrorTok}[1]{\textcolor[rgb]{0.64,0.00,0.00}{\textbf{#1}}}
\newcommand{\ExtensionTok}[1]{#1}
\newcommand{\FloatTok}[1]{\textcolor[rgb]{0.00,0.00,0.81}{#1}}
\newcommand{\FunctionTok}[1]{\textcolor[rgb]{0.00,0.00,0.00}{#1}}
\newcommand{\ImportTok}[1]{#1}
\newcommand{\InformationTok}[1]{\textcolor[rgb]{0.56,0.35,0.01}{\textbf{\textit{#1}}}}
\newcommand{\KeywordTok}[1]{\textcolor[rgb]{0.13,0.29,0.53}{\textbf{#1}}}
\newcommand{\NormalTok}[1]{#1}
\newcommand{\OperatorTok}[1]{\textcolor[rgb]{0.81,0.36,0.00}{\textbf{#1}}}
\newcommand{\OtherTok}[1]{\textcolor[rgb]{0.56,0.35,0.01}{#1}}
\newcommand{\PreprocessorTok}[1]{\textcolor[rgb]{0.56,0.35,0.01}{\textit{#1}}}
\newcommand{\RegionMarkerTok}[1]{#1}
\newcommand{\SpecialCharTok}[1]{\textcolor[rgb]{0.00,0.00,0.00}{#1}}
\newcommand{\SpecialStringTok}[1]{\textcolor[rgb]{0.31,0.60,0.02}{#1}}
\newcommand{\StringTok}[1]{\textcolor[rgb]{0.31,0.60,0.02}{#1}}
\newcommand{\VariableTok}[1]{\textcolor[rgb]{0.00,0.00,0.00}{#1}}
\newcommand{\VerbatimStringTok}[1]{\textcolor[rgb]{0.31,0.60,0.02}{#1}}
\newcommand{\WarningTok}[1]{\textcolor[rgb]{0.56,0.35,0.01}{\textbf{\textit{#1}}}}
\usepackage{graphicx}
\makeatletter
\def\maxwidth{\ifdim\Gin@nat@width>\linewidth\linewidth\else\Gin@nat@width\fi}
\def\maxheight{\ifdim\Gin@nat@height>\textheight\textheight\else\Gin@nat@height\fi}
\makeatother
% Scale images if necessary, so that they will not overflow the page
% margins by default, and it is still possible to overwrite the defaults
% using explicit options in \includegraphics[width, height, ...]{}
\setkeys{Gin}{width=\maxwidth,height=\maxheight,keepaspectratio}
% Set default figure placement to htbp
\makeatletter
\def\fps@figure{htbp}
\makeatother
\setlength{\emergencystretch}{3em} % prevent overfull lines
\providecommand{\tightlist}{%
  \setlength{\itemsep}{0pt}\setlength{\parskip}{0pt}}
\setcounter{secnumdepth}{-\maxdimen} % remove section numbering
\ifluatex
  \usepackage{selnolig}  % disable illegal ligatures
\fi

\title{STAT 33B Workbook 3}
\author{Ming Fong (3035619833)}
\date{Sep 17, 2020}

\begin{document}
\maketitle

This workbook is due \textbf{Sep 17, 2020} by 11:59pm PT.

The workbook is organized into sections that correspond to the lecture
videos for the week. Watch a video, then do the corresponding exercises
\emph{before} moving on to the next video.

Workbooks are graded for completeness, so as long as you make a clear
effort to solve each problem, you'll get full credit. That said, make
sure you understand the concepts here, because they're likely to
reappear in homeworks, quizzes, and later lectures.

As you work, write your answers in this notebook. Answer questions with
complete sentences, and put code in code chunks. You can make as many
new code chunks as you like.

In the notebook, you can run the line of code where the cursor is by
pressing \texttt{Ctrl} + \texttt{Enter} on Windows or \texttt{Cmd} +
\texttt{Enter} on Mac OS X. You can run an entire code chunk by clicking
on the green arrow in the upper right corner of the code chunk.

Please do not delete the exercises already in this notebook, because it
may interfere with our grading tools.

You need to submit your work in two places:

\begin{itemize}
\tightlist
\item
  Submit this Rmd file with your edits on bCourses.
\item
  Knit and submit the generated PDF file on Gradescope.
\end{itemize}

\hypertarget{three-ways-to-subset}{%
\section{Three Ways to Subset}\label{three-ways-to-subset}}

Watch the ``Three Ways to Subset'' lecture video.

\hypertarget{exercise-1}{%
\subsection{Exercise 1}\label{exercise-1}}

Create a variable \texttt{count} that contains the integers from 1 to
100 (inclusive).

The \texttt{as.character()} function coerces its argument into a
character vector. Coerce \texttt{count} into a character vector and
assign the result to a variable called \texttt{fizzy}. Now you have
congruent vectors \texttt{count} and \texttt{fizzy}.

Use subset assignment to replace every number in \texttt{fizzy} that's:

\begin{itemize}
\tightlist
\item
  Divisible by 3 with \texttt{"Fizz"}
\item
  Divisible by 5 with \texttt{"Buzz"}
\item
  Divisible by 15 with \texttt{"FizzBuzz"}
\end{itemize}

Leave all other numbers in \texttt{fizzy} as-is.

Print out the final version of \texttt{fizzy}. It should begin:

\begin{verbatim}
 [1] "1"        "2"        "Fizz"     "4"        "Buzz"     "Fizz"
 [7] "7"        "8"        "Fizz"     "Buzz"     "11"       "Fizz"
[13] "13"       "14"       "FizzBuzz" "16"       "17"       "Fizz"
\end{verbatim}

\emph{Hint 1: Take advantage of the fact that \texttt{count} and
\texttt{fizzy} are congruent.}

\emph{Hint 2: The modulo operator \texttt{\%\%} returns the remainder
after dividing its first argument by its second argument. You can use
the modulo operator to test whether a number is divisible by some other
number (that is, the remainder is zero after divsion).}

YOUR ANSWER GOES HERE:

\begin{Shaded}
\begin{Highlighting}[]
\NormalTok{count =}\StringTok{ }\KeywordTok{c}\NormalTok{(}\DecValTok{1}\OperatorTok{:}\DecValTok{100}\NormalTok{)}

\NormalTok{fizzy =}\StringTok{ }\KeywordTok{as.character}\NormalTok{(count)}

\NormalTok{fizzy[count }\OperatorTok{\%\%}\StringTok{ }\DecValTok{3} \OperatorTok{==}\StringTok{ }\DecValTok{0}\NormalTok{] =}\StringTok{ "Fizz"}
\NormalTok{fizzy[count }\OperatorTok{\%\%}\StringTok{ }\DecValTok{5} \OperatorTok{==}\StringTok{ }\DecValTok{0}\NormalTok{] =}\StringTok{ "Buzz"}
\NormalTok{fizzy[count }\OperatorTok{\%\%}\StringTok{ }\DecValTok{15} \OperatorTok{==}\StringTok{ }\DecValTok{0}\NormalTok{] =}\StringTok{ "FizzBuzz"}
\NormalTok{fizzy}
\end{Highlighting}
\end{Shaded}

\begin{verbatim}
##   [1] "1"        "2"        "Fizz"     "4"        "Buzz"     "Fizz"    
##   [7] "7"        "8"        "Fizz"     "Buzz"     "11"       "Fizz"    
##  [13] "13"       "14"       "FizzBuzz" "16"       "17"       "Fizz"    
##  [19] "19"       "Buzz"     "Fizz"     "22"       "23"       "Fizz"    
##  [25] "Buzz"     "26"       "Fizz"     "28"       "29"       "FizzBuzz"
##  [31] "31"       "32"       "Fizz"     "34"       "Buzz"     "Fizz"    
##  [37] "37"       "38"       "Fizz"     "Buzz"     "41"       "Fizz"    
##  [43] "43"       "44"       "FizzBuzz" "46"       "47"       "Fizz"    
##  [49] "49"       "Buzz"     "Fizz"     "52"       "53"       "Fizz"    
##  [55] "Buzz"     "56"       "Fizz"     "58"       "59"       "FizzBuzz"
##  [61] "61"       "62"       "Fizz"     "64"       "Buzz"     "Fizz"    
##  [67] "67"       "68"       "Fizz"     "Buzz"     "71"       "Fizz"    
##  [73] "73"       "74"       "FizzBuzz" "76"       "77"       "Fizz"    
##  [79] "79"       "Buzz"     "Fizz"     "82"       "83"       "Fizz"    
##  [85] "Buzz"     "86"       "Fizz"     "88"       "89"       "FizzBuzz"
##  [91] "91"       "92"       "Fizz"     "94"       "Buzz"     "Fizz"    
##  [97] "97"       "98"       "Fizz"     "Buzz"
\end{verbatim}

\hypertarget{logic}{%
\section{Logic}\label{logic}}

Watch the ``Logic'' lecture video.

\hypertarget{exercise-2}{%
\subsection{Exercise 2}\label{exercise-2}}

Suppose you conduct a survey and store the results in the following
congruent vectors:

\begin{Shaded}
\begin{Highlighting}[]
\CommentTok{\# Q: What\textquotesingle{}s your favorite color?}
\NormalTok{color =}\StringTok{ }\KeywordTok{c}\NormalTok{(}\StringTok{"red"}\NormalTok{, }\StringTok{"blue"}\NormalTok{, }\StringTok{"blue"}\NormalTok{, }\StringTok{"green"}\NormalTok{, }\StringTok{"yellow"}\NormalTok{, }\StringTok{"green"}\NormalTok{)}
\NormalTok{color =}\StringTok{ }\KeywordTok{factor}\NormalTok{(color)}

\CommentTok{\# Q: Name a dessert you like?}
\NormalTok{sweet =}\StringTok{ }\KeywordTok{c}\NormalTok{(}\StringTok{"egg tart"}\NormalTok{, }\StringTok{"brownie"}\NormalTok{, }\StringTok{"ice cream"}\NormalTok{, }\StringTok{"ice cream"}\NormalTok{, }\StringTok{"fruit"}\NormalTok{, }\StringTok{"egg tart"}\NormalTok{)}
\NormalTok{sweet =}\StringTok{ }\KeywordTok{factor}\NormalTok{(sweet)}

\CommentTok{\# Q: Name a desert (not dessert) you like?}
\NormalTok{dry =}\StringTok{ }\KeywordTok{c}\NormalTok{(}\StringTok{"Kalahari"}\NormalTok{, }\StringTok{"Atacama"}\NormalTok{, }\StringTok{"Taklamakan"}\NormalTok{, }\StringTok{"Sonoran"}\NormalTok{, }\StringTok{"Atacama"}\NormalTok{, }\StringTok{"Atacama"}\NormalTok{)}
\NormalTok{dry =}\StringTok{ }\KeywordTok{factor}\NormalTok{(dry)}

\CommentTok{\# Q: How old are you?}
\NormalTok{age =}\StringTok{ }\KeywordTok{c}\NormalTok{(}\DecValTok{23}\NormalTok{, }\DecValTok{15}\NormalTok{, }\DecValTok{92}\NormalTok{, }\DecValTok{21}\NormalTok{, }\DecValTok{28}\NormalTok{, }\DecValTok{45}\NormalTok{)}

\CommentTok{\# Q: How many UFOs have you seen since 2010?}
\NormalTok{ufo =}\StringTok{ }\KeywordTok{c}\NormalTok{(}\DecValTok{0}\NormalTok{, }\DecValTok{3}\NormalTok{, }\DecValTok{122}\NormalTok{, }\DecValTok{0}\NormalTok{, }\DecValTok{0}\NormalTok{, }\DecValTok{1}\NormalTok{)}
\end{Highlighting}
\end{Shaded}

Use the vectors above, comparison operators, and logical operators to
compute a logical vector that corresponds to each of the following
conditions.

\begin{enumerate}
\def\labelenumi{\arabic{enumi}.}
\item
  People who have seen a UFO.
\item
  People who have seen a UFO but aren't over 50 years old.
\item
  People who didn't choose ice cream.
\item
  People who like both ice cream and the color green.
\item
  People who like the color red or the color green.
\end{enumerate}

YOUR ANSWER GOES HERE:

\begin{enumerate}
\def\labelenumi{\arabic{enumi}.}
\tightlist
\item
\end{enumerate}

\begin{Shaded}
\begin{Highlighting}[]
\NormalTok{ufo }\OperatorTok{\textgreater{}}\StringTok{ }\DecValTok{0}
\end{Highlighting}
\end{Shaded}

\begin{verbatim}
## [1] FALSE  TRUE  TRUE FALSE FALSE  TRUE
\end{verbatim}

\begin{enumerate}
\def\labelenumi{\arabic{enumi}.}
\setcounter{enumi}{1}
\tightlist
\item
\end{enumerate}

\begin{Shaded}
\begin{Highlighting}[]
\NormalTok{ufo }\OperatorTok{\textgreater{}}\StringTok{ }\DecValTok{0} \OperatorTok{\&}\StringTok{ }\NormalTok{age }\OperatorTok{\textless{}}\StringTok{ }\DecValTok{50}
\end{Highlighting}
\end{Shaded}

\begin{verbatim}
## [1] FALSE  TRUE FALSE FALSE FALSE  TRUE
\end{verbatim}

\begin{enumerate}
\def\labelenumi{\arabic{enumi}.}
\setcounter{enumi}{2}
\tightlist
\item
\end{enumerate}

\begin{Shaded}
\begin{Highlighting}[]
\OperatorTok{!}\NormalTok{(sweet }\OperatorTok{==}\StringTok{ "ice cream"}\NormalTok{)}
\end{Highlighting}
\end{Shaded}

\begin{verbatim}
## [1]  TRUE  TRUE FALSE FALSE  TRUE  TRUE
\end{verbatim}

\begin{enumerate}
\def\labelenumi{\arabic{enumi}.}
\setcounter{enumi}{3}
\tightlist
\item
\end{enumerate}

\begin{Shaded}
\begin{Highlighting}[]
\NormalTok{sweet }\OperatorTok{==}\StringTok{ "ice cream"} \OperatorTok{\&}\StringTok{ }\NormalTok{color }\OperatorTok{==}\StringTok{ "green"}
\end{Highlighting}
\end{Shaded}

\begin{verbatim}
## [1] FALSE FALSE FALSE  TRUE FALSE FALSE
\end{verbatim}

\begin{enumerate}
\def\labelenumi{\arabic{enumi}.}
\setcounter{enumi}{4}
\tightlist
\item
\end{enumerate}

\begin{Shaded}
\begin{Highlighting}[]
\NormalTok{color }\OperatorTok{==}\StringTok{ "red"} \OperatorTok{|}\StringTok{ }\NormalTok{color }\OperatorTok{==}\StringTok{ "green"}
\end{Highlighting}
\end{Shaded}

\begin{verbatim}
## [1]  TRUE FALSE FALSE  TRUE FALSE  TRUE
\end{verbatim}

\hypertarget{exercise-3}{%
\subsection{Exercise 3}\label{exercise-3}}

In the expression \texttt{(x\ \textless{}\ 5)\ ==\ TRUE}, explain why
\texttt{==\ TRUE} is redundant.

YOUR ANSWER GOES HERE:

The \texttt{==} operator is a short-circuited \textbf{AND} operator. If
the left side of the operator is false, then the entire experssion will
be false. There is no need to check the right side of an \textbf{AND}
operator when the left is already false.

\hypertarget{logical-summaries}{%
\section{Logical Summaries}\label{logical-summaries}}

Watch the ``Logical Summaries'' lecture video.

No exercises for this section. You're halfway finished!

\hypertarget{subset-vs.-extract}{%
\section{Subset vs.~Extract}\label{subset-vs.-extract}}

Watch the ``Subset vs.~Extract'' lecture video.

\hypertarget{exercise-4}{%
\subsection{Exercise 4}\label{exercise-4}}

A \textbf{recursive} list is a list with elements that are also lists.

Here's an example of a recursive list:

\begin{Shaded}
\begin{Highlighting}[]
\NormalTok{mylist =}\StringTok{ }\KeywordTok{list}\NormalTok{(}\KeywordTok{list}\NormalTok{(1i, }\DecValTok{2}\NormalTok{, 3i), }\KeywordTok{list}\NormalTok{(}\KeywordTok{c}\NormalTok{(}\StringTok{"hello"}\NormalTok{, }\StringTok{"hi"}\NormalTok{), }\DecValTok{42}\NormalTok{))}
\end{Highlighting}
\end{Shaded}

Use the recursive list above to answer the following:

\begin{enumerate}
\def\labelenumi{\arabic{enumi}.}
\item
  What's the first element? What's the second element?
\item
  Use the extraction operator \texttt{{[}{[}} to get the value
  \texttt{3i}.
\item
  What does \texttt{mylist{[}{[}c(1,\ 3){]}{]}} do? What does the index
  \texttt{c(1,\ 3)} mean here? Experiment with using other vectors in
  \texttt{{[}{[}} to get elements from the recursive list. Then explain
  what the extraction operator \texttt{{[}{[}} does for recursive lists
  when the index is a vector.
\end{enumerate}

YOUR ANSWER GOES HERE:

\begin{enumerate}
\def\labelenumi{\arabic{enumi}.}
\tightlist
\item
  The first element is the list with \texttt{(1i,\ 2,\ 3i)} and the
  second is a list with \texttt{("hello",\ "hi"),\ 42}.
\end{enumerate}

\begin{Shaded}
\begin{Highlighting}[]
\NormalTok{mylist[[}\DecValTok{1}\NormalTok{]]}
\end{Highlighting}
\end{Shaded}

\begin{verbatim}
## [[1]]
## [1] 0+1i
## 
## [[2]]
## [1] 2
## 
## [[3]]
## [1] 0+3i
\end{verbatim}

\begin{Shaded}
\begin{Highlighting}[]
\NormalTok{mylist[[}\DecValTok{2}\NormalTok{]]}
\end{Highlighting}
\end{Shaded}

\begin{verbatim}
## [[1]]
## [1] "hello" "hi"   
## 
## [[2]]
## [1] 42
\end{verbatim}

\begin{enumerate}
\def\labelenumi{\arabic{enumi}.}
\setcounter{enumi}{1}
\tightlist
\item
\end{enumerate}

\begin{Shaded}
\begin{Highlighting}[]
\NormalTok{mylist[[}\DecValTok{1}\NormalTok{]][[}\DecValTok{3}\NormalTok{]]}
\end{Highlighting}
\end{Shaded}

\begin{verbatim}
## [1] 0+3i
\end{verbatim}

\begin{enumerate}
\def\labelenumi{\arabic{enumi}.}
\setcounter{enumi}{2}
\tightlist
\item
  The statement \texttt{mylist{[}{[}c(1,\ 3){]}{]}} does same as in part
  2: it returns the third element of the first list.
\end{enumerate}

\begin{Shaded}
\begin{Highlighting}[]
\NormalTok{mylist[[}\KeywordTok{c}\NormalTok{(}\DecValTok{1}\NormalTok{, }\DecValTok{3}\NormalTok{)]]}
\end{Highlighting}
\end{Shaded}

\begin{verbatim}
## [1] 0+3i
\end{verbatim}

\begin{Shaded}
\begin{Highlighting}[]
\NormalTok{mylist[[}\KeywordTok{c}\NormalTok{(}\DecValTok{2}\NormalTok{, }\DecValTok{1}\NormalTok{, }\DecValTok{2}\NormalTok{)]]}
\end{Highlighting}
\end{Shaded}

\begin{verbatim}
## [1] "hi"
\end{verbatim}

Passing a vector into the index of \texttt{{[}{[}} for recursive lists
gets elements of the nested list. The \texttt{{[}{[}} operator will use
the \texttt{nth} value of a passed vector as the index of the
\texttt{nth} nested list.

\hypertarget{exercise-5}{%
\subsection{Exercise 5}\label{exercise-5}}

For the list
\texttt{cool\_list\ =\ list("Hope",\ "springs",\ "eternal")}, why is
\texttt{cool\_list{[}1{]}} the same as
\texttt{cool\_list{[}1{]}{[}1{]}{[}1{]}{[}1{]}{[}1{]}}? Is this property
unique to \texttt{cool\_list}, or is it a property of all lists? Explain
your answer.

YOUR ANSWER GOES HERE:

Because \texttt{{[}} does not remove containers, it returns a list. Thus
endlessly adding \texttt{{[}1{]}} just returns more identical lists.

\begin{Shaded}
\begin{Highlighting}[]
\NormalTok{cool\_list =}\StringTok{ }\KeywordTok{list}\NormalTok{(}\StringTok{"Hope"}\NormalTok{, }\StringTok{"springs"}\NormalTok{, }\StringTok{"eternal"}\NormalTok{)}
\NormalTok{cool\_list[}\DecValTok{1}\NormalTok{]}
\end{Highlighting}
\end{Shaded}

\begin{verbatim}
## [[1]]
## [1] "Hope"
\end{verbatim}

\begin{Shaded}
\begin{Highlighting}[]
\NormalTok{cool\_list[}\DecValTok{1}\NormalTok{][}\DecValTok{1}\NormalTok{][}\DecValTok{1}\NormalTok{][}\DecValTok{1}\NormalTok{][}\DecValTok{1}\NormalTok{]}
\end{Highlighting}
\end{Shaded}

\begin{verbatim}
## [[1]]
## [1] "Hope"
\end{verbatim}

\hypertarget{subsets-of-data-frames}{%
\section{Subsets of Data Frames}\label{subsets-of-data-frames}}

Watch the ``Subsets of Data Frames'' lecture video.

\hypertarget{exercise-6}{%
\subsection{Exercise 6}\label{exercise-6}}

For the dogs data, compute:

\begin{enumerate}
\def\labelenumi{\arabic{enumi}.}
\item
  The subset that contains rows 10-20 of the height, weight, and
  longevity columns.
\item
  The mean and median of the longevity column (ignoring missing values).
\item
  The number of dog breeds whose average weight is greater than 42.
  \emph{Note: the \texttt{weight} column is the average weight of each
  row's breed.}
\item
  The subset of large dogs that require daily grooming.
\end{enumerate}

YOUR ANSWER GOES HERE:

\begin{enumerate}
\def\labelenumi{\arabic{enumi}.}
\tightlist
\item
\end{enumerate}

\begin{Shaded}
\begin{Highlighting}[]
\NormalTok{dogs =}\StringTok{ }\KeywordTok{readRDS}\NormalTok{(}\StringTok{"C:}\CharTok{\textbackslash{}\textbackslash{}}\StringTok{Users}\CharTok{\textbackslash{}\textbackslash{}}\StringTok{mingf}\CharTok{\textbackslash{}\textbackslash{}}\StringTok{Desktop}\CharTok{\textbackslash{}\textbackslash{}}\StringTok{git}\CharTok{\textbackslash{}\textbackslash{}}\StringTok{STAT33B}\CharTok{\textbackslash{}\textbackslash{}}\StringTok{Week 3}\CharTok{\textbackslash{}\textbackslash{}}\StringTok{data}\CharTok{\textbackslash{}\textbackslash{}}\StringTok{dogs.rds"}\NormalTok{)}
\NormalTok{dogs[}\DecValTok{10}\OperatorTok{:}\DecValTok{20}\NormalTok{, }\KeywordTok{c}\NormalTok{(}\StringTok{"height"}\NormalTok{, }\StringTok{"weight"}\NormalTok{, }\StringTok{"longevity"}\NormalTok{)]}
\end{Highlighting}
\end{Shaded}

\begin{verbatim}
##    height weight longevity
## 10  14.50   22.0     12.53
## 11  21.75   47.5     12.58
## 12  10.50   15.0     13.92
## 13  10.25     NA     11.42
## 14     NA   24.0     12.63
## 15  13.00   15.5     11.81
## 16   5.00    5.5     16.50
## 17  10.50     NA     11.05
## 18  20.00     NA     12.87
## 19  19.50   45.0     12.54
## 20  10.50     NA     12.80
\end{verbatim}

\begin{enumerate}
\def\labelenumi{\arabic{enumi}.}
\setcounter{enumi}{1}
\tightlist
\item
  Mean: 10.95674, Median: 11.29
\end{enumerate}

\begin{Shaded}
\begin{Highlighting}[]
\KeywordTok{mean}\NormalTok{(dogs[, }\StringTok{"longevity"}\NormalTok{], }\DataTypeTok{na.rm =} \OtherTok{TRUE}\NormalTok{)}
\end{Highlighting}
\end{Shaded}

\begin{verbatim}
## [1] 10.95674
\end{verbatim}

\begin{Shaded}
\begin{Highlighting}[]
\KeywordTok{median}\NormalTok{(dogs[, }\StringTok{"longevity"}\NormalTok{], }\DataTypeTok{na.rm =} \OtherTok{TRUE}\NormalTok{)}
\end{Highlighting}
\end{Shaded}

\begin{verbatim}
## [1] 11.29
\end{verbatim}

\begin{enumerate}
\def\labelenumi{\arabic{enumi}.}
\setcounter{enumi}{2}
\tightlist
\item
  37 breeds have an average weight greater than 42.
\end{enumerate}

\begin{Shaded}
\begin{Highlighting}[]
\KeywordTok{sum}\NormalTok{(dogs[, }\StringTok{"weight"}\NormalTok{] }\OperatorTok{\textgreater{}}\StringTok{ }\DecValTok{42}\NormalTok{, }\DataTypeTok{na.rm =} \OtherTok{TRUE}\NormalTok{)}
\end{Highlighting}
\end{Shaded}

\begin{verbatim}
## [1] 37
\end{verbatim}

\begin{enumerate}
\def\labelenumi{\arabic{enumi}.}
\setcounter{enumi}{3}
\tightlist
\item
\end{enumerate}

\begin{Shaded}
\begin{Highlighting}[]
\KeywordTok{subset}\NormalTok{(dogs, size }\OperatorTok{==}\StringTok{ "large"} \OperatorTok{\&}\StringTok{ }\NormalTok{grooming }\OperatorTok{==}\StringTok{ "daily"}\NormalTok{)}
\end{Highlighting}
\end{Shaded}

\begin{verbatim}
##               breed   group datadog popularity_all popularity lifetime_cost
## 44           Briard herding    2.71            125         79         19673
## 62  Giant Schnauzer working    2.38             95         70         26686
## 67     Afghan Hound   hound    2.08             88         66         24077
## 75           Borzoi   hound    1.89            102         71         16176
## 79 Alaskan Malamute working    1.82             58         47         21986
## 86    Saint Bernard working    1.42             49         43         20022
##    intelligence_rank longevity ailments price food_cost grooming   kids
## 44                30     11.17        1   650       466    daily   high
## 62                28     10.00        1   810      1349    daily medium
## 67                80     11.92        0   890       710    daily   high
## 75                76      9.08        0   675       466    daily medium
## 79                50     10.67        2  1210       710    daily medium
## 86                65      7.78        3   875      1217    daily   high
##    megarank_kids megarank  size weight height
## 44            44       33 large     NA   24.5
## 62            62       67 large   77.5   25.5
## 67            67       60 large   55.0   26.0
## 75            75       82 large   82.5   28.0
## 79            79       83 large   80.0   24.0
## 86            86       81 large  155.0   26.5
\end{verbatim}

\hypertarget{exercise-7}{%
\subsection{Exercise 7}\label{exercise-7}}

The \texttt{sort()} function sorts the elements of a vector. For
instance:

\begin{Shaded}
\begin{Highlighting}[]
\NormalTok{x =}\StringTok{ }\KeywordTok{c}\NormalTok{(}\DecValTok{4}\NormalTok{, }\DecValTok{5}\NormalTok{, }\DecValTok{1}\NormalTok{)}
\KeywordTok{sort}\NormalTok{(x)}
\end{Highlighting}
\end{Shaded}

\begin{verbatim}
## [1] 1 4 5
\end{verbatim}

The \texttt{order()} function is a more flexible alternative to
\texttt{sort()}. Instead of returning the sorted vector, the
\texttt{order()} function returns the index that sorts the vector. To
actually sort the vector, you have to pass this index to the subset
operator \texttt{{[}}:

\begin{Shaded}
\begin{Highlighting}[]
\NormalTok{x =}\StringTok{ }\KeywordTok{c}\NormalTok{(}\DecValTok{4}\NormalTok{, }\DecValTok{5}\NormalTok{, }\DecValTok{1}\NormalTok{)}
\NormalTok{x[}\KeywordTok{order}\NormalTok{(x)]}
\end{Highlighting}
\end{Shaded}

\begin{verbatim}
## [1] 1 4 5
\end{verbatim}

The advantage of \texttt{order()} over \texttt{sort()} is that you can
use \texttt{order()} to sort one vector based on the elements of some
other congruent vector.

Use the \texttt{order()} function to sort the rows of the dogs data set
based on height. What are the 3 tallest breeds of dog?

YOUR ANSWER GOES HERE:

\begin{enumerate}
\def\labelenumi{\arabic{enumi}.}
\tightlist
\item
  The 3 tallest breeds are: ``Irish Wolfhound'', ``Mastiff'', and
  ``Great Dane''.
\end{enumerate}

\begin{Shaded}
\begin{Highlighting}[]
\NormalTok{dogs\_by\_height =}\StringTok{ }\NormalTok{dogs[}\KeywordTok{order}\NormalTok{(}\OperatorTok{{-}}\NormalTok{dogs[, }\StringTok{"height"}\NormalTok{]), ]}
\KeywordTok{head}\NormalTok{(dogs\_by\_height)}
\end{Highlighting}
\end{Shaded}

\begin{verbatim}
##               breed   group datadog popularity_all popularity lifetime_cost
## 82  Irish Wolfhound   hound    1.66             79         60         18435
## 84          Mastiff working    1.57             28         28         13581
## 85       Great Dane working    1.53             19         19         14662
## 119  Great Pyrenees working      NA             71         NA            NA
## 133      Leonberger working      NA            103         NA         15141
## 75           Borzoi   hound    1.89            102         71         16176
##     intelligence_rank longevity ailments price food_cost grooming   kids
## 82                 41      6.94        3  1333      1217   weekly   high
## 84                 72      6.50        2   900       701   weekly   high
## 85                 48      6.96        4  1040       710   weekly   high
## 119                64     10.00        1   503        NA     <NA>   <NA>
## 133                NA      6.98       NA  1480        NA   weekly   high
## 75                 76      9.08        0   675       466    daily medium
##     megarank_kids megarank  size weight height
## 82             82       70 large     NA   32.0
## 84             84       73 large  175.0   30.0
## 85             85       75 large     NA   30.0
## 119            NA       NA large     NA   28.5
## 133            NA       NA large     NA   28.5
## 75             75       82 large   82.5   28.0
\end{verbatim}

\begin{Shaded}
\begin{Highlighting}[]
\NormalTok{dogs\_by\_height[}\DecValTok{1}\OperatorTok{:}\DecValTok{3}\NormalTok{, }\StringTok{"breed"}\NormalTok{]}
\end{Highlighting}
\end{Shaded}

\begin{verbatim}
## [1] "Irish Wolfhound" "Mastiff"         "Great Dane"
\end{verbatim}

\end{document}
