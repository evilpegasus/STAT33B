% Options for packages loaded elsewhere
\PassOptionsToPackage{unicode}{hyperref}
\PassOptionsToPackage{hyphens}{url}
%
\documentclass[
]{article}
\usepackage{lmodern}
\usepackage{amssymb,amsmath}
\usepackage{ifxetex,ifluatex}
\ifnum 0\ifxetex 1\fi\ifluatex 1\fi=0 % if pdftex
  \usepackage[T1]{fontenc}
  \usepackage[utf8]{inputenc}
  \usepackage{textcomp} % provide euro and other symbols
\else % if luatex or xetex
  \usepackage{unicode-math}
  \defaultfontfeatures{Scale=MatchLowercase}
  \defaultfontfeatures[\rmfamily]{Ligatures=TeX,Scale=1}
\fi
% Use upquote if available, for straight quotes in verbatim environments
\IfFileExists{upquote.sty}{\usepackage{upquote}}{}
\IfFileExists{microtype.sty}{% use microtype if available
  \usepackage[]{microtype}
  \UseMicrotypeSet[protrusion]{basicmath} % disable protrusion for tt fonts
}{}
\makeatletter
\@ifundefined{KOMAClassName}{% if non-KOMA class
  \IfFileExists{parskip.sty}{%
    \usepackage{parskip}
  }{% else
    \setlength{\parindent}{0pt}
    \setlength{\parskip}{6pt plus 2pt minus 1pt}}
}{% if KOMA class
  \KOMAoptions{parskip=half}}
\makeatother
\usepackage{xcolor}
\IfFileExists{xurl.sty}{\usepackage{xurl}}{} % add URL line breaks if available
\IfFileExists{bookmark.sty}{\usepackage{bookmark}}{\usepackage{hyperref}}
\hypersetup{
  pdftitle={STAT 33B Workbook 2},
  pdfauthor={Ming Fong (3035619833)},
  hidelinks,
  pdfcreator={LaTeX via pandoc}}
\urlstyle{same} % disable monospaced font for URLs
\usepackage[margin=1in]{geometry}
\usepackage{color}
\usepackage{fancyvrb}
\newcommand{\VerbBar}{|}
\newcommand{\VERB}{\Verb[commandchars=\\\{\}]}
\DefineVerbatimEnvironment{Highlighting}{Verbatim}{commandchars=\\\{\}}
% Add ',fontsize=\small' for more characters per line
\usepackage{framed}
\definecolor{shadecolor}{RGB}{248,248,248}
\newenvironment{Shaded}{\begin{snugshade}}{\end{snugshade}}
\newcommand{\AlertTok}[1]{\textcolor[rgb]{0.94,0.16,0.16}{#1}}
\newcommand{\AnnotationTok}[1]{\textcolor[rgb]{0.56,0.35,0.01}{\textbf{\textit{#1}}}}
\newcommand{\AttributeTok}[1]{\textcolor[rgb]{0.77,0.63,0.00}{#1}}
\newcommand{\BaseNTok}[1]{\textcolor[rgb]{0.00,0.00,0.81}{#1}}
\newcommand{\BuiltInTok}[1]{#1}
\newcommand{\CharTok}[1]{\textcolor[rgb]{0.31,0.60,0.02}{#1}}
\newcommand{\CommentTok}[1]{\textcolor[rgb]{0.56,0.35,0.01}{\textit{#1}}}
\newcommand{\CommentVarTok}[1]{\textcolor[rgb]{0.56,0.35,0.01}{\textbf{\textit{#1}}}}
\newcommand{\ConstantTok}[1]{\textcolor[rgb]{0.00,0.00,0.00}{#1}}
\newcommand{\ControlFlowTok}[1]{\textcolor[rgb]{0.13,0.29,0.53}{\textbf{#1}}}
\newcommand{\DataTypeTok}[1]{\textcolor[rgb]{0.13,0.29,0.53}{#1}}
\newcommand{\DecValTok}[1]{\textcolor[rgb]{0.00,0.00,0.81}{#1}}
\newcommand{\DocumentationTok}[1]{\textcolor[rgb]{0.56,0.35,0.01}{\textbf{\textit{#1}}}}
\newcommand{\ErrorTok}[1]{\textcolor[rgb]{0.64,0.00,0.00}{\textbf{#1}}}
\newcommand{\ExtensionTok}[1]{#1}
\newcommand{\FloatTok}[1]{\textcolor[rgb]{0.00,0.00,0.81}{#1}}
\newcommand{\FunctionTok}[1]{\textcolor[rgb]{0.00,0.00,0.00}{#1}}
\newcommand{\ImportTok}[1]{#1}
\newcommand{\InformationTok}[1]{\textcolor[rgb]{0.56,0.35,0.01}{\textbf{\textit{#1}}}}
\newcommand{\KeywordTok}[1]{\textcolor[rgb]{0.13,0.29,0.53}{\textbf{#1}}}
\newcommand{\NormalTok}[1]{#1}
\newcommand{\OperatorTok}[1]{\textcolor[rgb]{0.81,0.36,0.00}{\textbf{#1}}}
\newcommand{\OtherTok}[1]{\textcolor[rgb]{0.56,0.35,0.01}{#1}}
\newcommand{\PreprocessorTok}[1]{\textcolor[rgb]{0.56,0.35,0.01}{\textit{#1}}}
\newcommand{\RegionMarkerTok}[1]{#1}
\newcommand{\SpecialCharTok}[1]{\textcolor[rgb]{0.00,0.00,0.00}{#1}}
\newcommand{\SpecialStringTok}[1]{\textcolor[rgb]{0.31,0.60,0.02}{#1}}
\newcommand{\StringTok}[1]{\textcolor[rgb]{0.31,0.60,0.02}{#1}}
\newcommand{\VariableTok}[1]{\textcolor[rgb]{0.00,0.00,0.00}{#1}}
\newcommand{\VerbatimStringTok}[1]{\textcolor[rgb]{0.31,0.60,0.02}{#1}}
\newcommand{\WarningTok}[1]{\textcolor[rgb]{0.56,0.35,0.01}{\textbf{\textit{#1}}}}
\usepackage{graphicx}
\makeatletter
\def\maxwidth{\ifdim\Gin@nat@width>\linewidth\linewidth\else\Gin@nat@width\fi}
\def\maxheight{\ifdim\Gin@nat@height>\textheight\textheight\else\Gin@nat@height\fi}
\makeatother
% Scale images if necessary, so that they will not overflow the page
% margins by default, and it is still possible to overwrite the defaults
% using explicit options in \includegraphics[width, height, ...]{}
\setkeys{Gin}{width=\maxwidth,height=\maxheight,keepaspectratio}
% Set default figure placement to htbp
\makeatletter
\def\fps@figure{htbp}
\makeatother
\setlength{\emergencystretch}{3em} % prevent overfull lines
\providecommand{\tightlist}{%
  \setlength{\itemsep}{0pt}\setlength{\parskip}{0pt}}
\setcounter{secnumdepth}{-\maxdimen} % remove section numbering
\ifluatex
  \usepackage{selnolig}  % disable illegal ligatures
\fi

\title{STAT 33B Workbook 2}
\author{Ming Fong (3035619833)}
\date{Sep 10, 2020}

\begin{document}
\maketitle

This workbook is due \textbf{Sep 10, 2020} by 11:59pm PT.

The workbook is organized into sections that correspond to the lecture
videos for the week. Watch a video, then do the corresponding exercises
\emph{before} moving on to the next video.

Workbooks are graded for completeness, so as long as you make a clear
effort to solve each problem, you'll get full credit. That said, make
sure you understand the concepts here, because they're likely to
reappear in homeworks, quizzes, and later lectures.

As you work, write your answers in this notebook. Answer questions with
complete sentences, and put code in code chunks. You can make as many
new code chunks as you like.

In the notebook, you can run the line of code where the cursor is by
pressing \texttt{Ctrl} + \texttt{Enter} on Windows or \texttt{Cmd} +
\texttt{Enter} on Mac OS X. You can run an entire code chunk by clicking
on the green arrow in the upper right corner of the code chunk.

Please do not delete the exercises already in this notebook, because it
may interfere with our grading tools.

You need to submit your work in two places:

\begin{itemize}
\tightlist
\item
  Submit this Rmd file with your edits on bCourses.
\item
  Knit and submit the generated PDF file on Gradescope.
\end{itemize}

\hypertarget{file-systems}{%
\section{File Systems}\label{file-systems}}

Watch the ``File Systems'' lecture video.

\hypertarget{exercise-1}{%
\subsection{Exercise 1}\label{exercise-1}}

For each of the following paths, say whether the path is absolute or
relative, and explain how you can tell.

\begin{enumerate}
\def\labelenumi{\arabic{enumi}.}
\tightlist
\item
  \texttt{"./documents"}
\item
  \texttt{"/Users/Jun/doggos\_to\_pet.md"}
\item
  \texttt{"TODO.md"}
\end{enumerate}

YOUR ANSWER GOES HERE:

\begin{enumerate}
\def\labelenumi{\arabic{enumi}.}
\item
  Relative. The path starts with a \texttt{./} which searches for
  \texttt{"documents"} from the working directory.
\item
  Absolute. The path starts at the root \texttt{/}.
\item
  Relative. The path searches for \texttt{"TODO.md"} from the current
  working directory.
\end{enumerate}

\hypertarget{the-r-working-directory}{%
\section{The R Working Directory}\label{the-r-working-directory}}

Watch the ``The R Working Directory'' lecture video.

\hypertarget{exercise-2}{%
\subsection{Exercise 2}\label{exercise-2}}

\begin{enumerate}
\def\labelenumi{\arabic{enumi}.}
\tightlist
\item
  What's the root directory called on your computer?
\item
  What's the absolute path to the home directory on your computer?
\item
  Use R to count the total number of files in your home directory. Your
  code should return the result as a number.
\end{enumerate}

YOUR ANSWER GOES HERE:

\begin{enumerate}
\def\labelenumi{\arabic{enumi}.}
\item
  \texttt{"C:/"}
\item
  \texttt{"C:\textbackslash{}\textbackslash{}Users\textbackslash{}\textbackslash{}mingf\textbackslash{}\textbackslash{}Documents"}
\end{enumerate}

\begin{Shaded}
\begin{Highlighting}[]
\KeywordTok{normalizePath}\NormalTok{(}\StringTok{"\textasciitilde{}"}\NormalTok{)}
\end{Highlighting}
\end{Shaded}

\begin{verbatim}
## [1] "C:\\Users\\mingf\\Documents"
\end{verbatim}

\begin{enumerate}
\def\labelenumi{\arabic{enumi}.}
\setcounter{enumi}{2}
\tightlist
\item
  There are \(14\) files in my home directory. See below:
\end{enumerate}

\begin{Shaded}
\begin{Highlighting}[]
\KeywordTok{length}\NormalTok{(}\KeywordTok{list.files}\NormalTok{(}\StringTok{"\textasciitilde{}"}\NormalTok{))}
\end{Highlighting}
\end{Shaded}

\begin{verbatim}
## [1] 14
\end{verbatim}

\hypertarget{data-frames}{%
\section{Data Frames}\label{data-frames}}

Watch the ``Data Frames'' lecture video.

\hypertarget{exercise-3}{%
\subsection{Exercise 3}\label{exercise-3}}

\begin{enumerate}
\def\labelenumi{\arabic{enumi}.}
\tightlist
\item
  Load the dogs data set \texttt{dogs.rds} into R (this one has more
  than 10 rows).
\item
  What's the mean weight of the dogs? You can use the \texttt{na.rm}
  parameter in the \texttt{mean()} function to make the function ignore
  missing values.
\item
  The \texttt{which.min()} function returns the index of the minimum
  element of a vector. Which breed of dog has the shortest height?
\item
  Which breed of dog has the longest lifespan?
\end{enumerate}

YOUR ANSWER GOES HERE:

\begin{enumerate}
\def\labelenumi{\arabic{enumi}.}
\tightlist
\item
\end{enumerate}

\begin{Shaded}
\begin{Highlighting}[]
\NormalTok{dogs =}\StringTok{ }\KeywordTok{readRDS}\NormalTok{(}\StringTok{"C:}\CharTok{\textbackslash{}\textbackslash{}}\StringTok{Users}\CharTok{\textbackslash{}\textbackslash{}}\StringTok{mingf}\CharTok{\textbackslash{}\textbackslash{}}\StringTok{Desktop}\CharTok{\textbackslash{}\textbackslash{}}\StringTok{git}\CharTok{\textbackslash{}\textbackslash{}}\StringTok{STAT33B}\CharTok{\textbackslash{}\textbackslash{}}\StringTok{Week 3}\CharTok{\textbackslash{}\textbackslash{}}\StringTok{data}\CharTok{\textbackslash{}\textbackslash{}}\StringTok{dogs.rds"}\NormalTok{)}
\end{Highlighting}
\end{Shaded}

\begin{enumerate}
\def\labelenumi{\arabic{enumi}.}
\setcounter{enumi}{1}
\tightlist
\item
  The mean weight is \(44.97093\).
\end{enumerate}

\begin{Shaded}
\begin{Highlighting}[]
\KeywordTok{mean}\NormalTok{(dogs}\OperatorTok{$}\NormalTok{weight, }\DataTypeTok{na.rm =} \OtherTok{TRUE}\NormalTok{)}
\end{Highlighting}
\end{Shaded}

\begin{verbatim}
## [1] 44.97093
\end{verbatim}

\begin{enumerate}
\def\labelenumi{\arabic{enumi}.}
\setcounter{enumi}{2}
\tightlist
\item
  \texttt{Chihuahua}
\end{enumerate}

\begin{Shaded}
\begin{Highlighting}[]
\NormalTok{dogs[}\KeywordTok{which.min}\NormalTok{(dogs}\OperatorTok{$}\NormalTok{height), }\DecValTok{1}\NormalTok{]}
\end{Highlighting}
\end{Shaded}

\begin{verbatim}
## [1] "Chihuahua"
\end{verbatim}

\begin{enumerate}
\def\labelenumi{\arabic{enumi}.}
\setcounter{enumi}{3}
\tightlist
\item
  \texttt{Chihuahua}
\end{enumerate}

\begin{Shaded}
\begin{Highlighting}[]
\NormalTok{dogs[}\KeywordTok{which.max}\NormalTok{(dogs}\OperatorTok{$}\NormalTok{longevity), }\DecValTok{1}\NormalTok{]}
\end{Highlighting}
\end{Shaded}

\begin{verbatim}
## [1] "Chihuahua"
\end{verbatim}

\hypertarget{factors}{%
\section{Factors}\label{factors}}

Watch the ``Factors'' lecture video.

No exercises for this video. Get up, stretch, and take a break! :)

\hypertarget{file-formats}{%
\section{File Formats}\label{file-formats}}

Watch the ``File Formats'' lecture video.

\hypertarget{exercise-5}{%
\subsection{Exercise 5}\label{exercise-5}}

\begin{enumerate}
\def\labelenumi{\arabic{enumi}.}
\tightlist
\item
  Load the volcano data set into R.
\item
  What are the column names? Use R to get these rather than typing them
  out yourself.
\item
  How many volcano eruptions are recorded in the data set?
\item
  What are the classes/types of the columns? \emph{Hint: an earlier
  lecture mentioned a function that summarizes of this information.}
\item
  Are there any columns that contain categorical data? Are these columns
  factors? If not, what are their classes?
\end{enumerate}

YOUR ANSWER GOES HERE:

\begin{enumerate}
\def\labelenumi{\arabic{enumi}.}
\tightlist
\item
\end{enumerate}

\begin{Shaded}
\begin{Highlighting}[]
\NormalTok{volcanoes =}\StringTok{ }\KeywordTok{read.csv}\NormalTok{(}\StringTok{"C:}\CharTok{\textbackslash{}\textbackslash{}}\StringTok{Users}\CharTok{\textbackslash{}\textbackslash{}}\StringTok{mingf}\CharTok{\textbackslash{}\textbackslash{}}\StringTok{Desktop}\CharTok{\textbackslash{}\textbackslash{}}\StringTok{git}\CharTok{\textbackslash{}\textbackslash{}}\StringTok{STAT33B}\CharTok{\textbackslash{}\textbackslash{}}\StringTok{Week 3}\CharTok{\textbackslash{}\textbackslash{}}\StringTok{data}\CharTok{\textbackslash{}\textbackslash{}}\StringTok{volcano.csv"}\NormalTok{)}
\end{Highlighting}
\end{Shaded}

\begin{enumerate}
\def\labelenumi{\arabic{enumi}.}
\setcounter{enumi}{1}
\tightlist
\item
\end{enumerate}

\begin{Shaded}
\begin{Highlighting}[]
\KeywordTok{colnames}\NormalTok{(volcanoes)}
\end{Highlighting}
\end{Shaded}

\begin{verbatim}
##  [1] "X"                                  "Year"                              
##  [3] "Month"                              "Day"                               
##  [5] "TSU"                                "EQ"                                
##  [7] "Name"                               "Location"                          
##  [9] "Country"                            "Latitude"                          
## [11] "Longitude"                          "Elevation"                         
## [13] "Type"                               "Status"                            
## [15] "Time"                               "VEI"                               
## [17] "Agent"                              "DEATHS"                            
## [19] "DEATHS_DESCRIPTION"                 "MISSING"                           
## [21] "MISSING_DESCRIPTION"                "INJURIES"                          
## [23] "INJURIES_DESCRIPTION"               "DAMAGE_MILLIONS_DOLLARS"           
## [25] "DAMAGE_DESCRIPTION"                 "HOUSES_DESTROYED"                  
## [27] "HOUSES_DESTROYED_DESCRIPTION"       "TOTAL_DEATHS"                      
## [29] "TOTAL_DEATHS_DESCRIPTION"           "TOTAL_MISSING"                     
## [31] "TOTAL_MISSING_DESCRIPTION"          "TOTAL_INJURIES"                    
## [33] "TOTAL_INJURIES_DESCRIPTION"         "TOTAL_DAMAGE_MILLIONS_DOLLARS"     
## [35] "TOTAL_DAMAGE_DESCRIPTION"           "TOTAL_HOUSES_DESTROYED"            
## [37] "TOTAL_HOUSES_DESTROYED_DESCRIPTION"
\end{verbatim}

\begin{enumerate}
\def\labelenumi{\arabic{enumi}.}
\setcounter{enumi}{2}
\tightlist
\item
  835
\end{enumerate}

\begin{Shaded}
\begin{Highlighting}[]
\KeywordTok{nrow}\NormalTok{(volcanoes)}
\end{Highlighting}
\end{Shaded}

\begin{verbatim}
## [1] 835
\end{verbatim}

\begin{enumerate}
\def\labelenumi{\arabic{enumi}.}
\setcounter{enumi}{3}
\tightlist
\item
\end{enumerate}

\begin{Shaded}
\begin{Highlighting}[]
\KeywordTok{str}\NormalTok{(volcanoes)}
\end{Highlighting}
\end{Shaded}

\begin{verbatim}
## 'data.frame':    835 obs. of  37 variables:
##  $ X                                 : int  1 2 3 4 5 6 7 8 9 10 ...
##  $ Year                              : int  -4360 -4350 -4050 -4000 -3580 -3550 -2420 -2040 -1900 -1890 ...
##  $ Month                             : int  NA NA NA NA NA NA NA NA NA NA ...
##  $ Day                               : int  NA NA NA NA NA NA NA NA NA NA ...
##  $ TSU                               : chr  "" "" "" "" ...
##  $ EQ                                : chr  "" "" "" "" ...
##  $ Name                              : chr  "Macauley Island" "Kikai" "Masaya" "Pago" ...
##  $ Location                          : chr  "Kermadec Is" "Ryukyu Is" "Nicaragua" "New Britain-SW Pac" ...
##  $ Country                           : chr  "New Zealand" "Japan" "Nicaragua" "Papua New Guinea" ...
##  $ Latitude                          : num  -30.2 30.78 11.98 -5.58 14 ...
##  $ Longitude                         : num  -178.5 130.3 -86.2 150.5 121 ...
##  $ Elevation                         : int  238 717 635 742 400 1486 1281 1280 1032 1905 ...
##  $ Type                              : chr  "Caldera" "Caldera" "Caldera" "Caldera" ...
##  $ Status                            : chr  "Holocene" "Historical" "Historical" "Historical" ...
##  $ Time                              : chr  "U" "D1" "D1" "D2" ...
##  $ VEI                               : int  6 7 6 6 6 6 5 6 6 6 ...
##  $ Agent                             : chr  "" "P" "" "T" ...
##  $ DEATHS                            : int  NA NA NA NA NA NA NA NA NA NA ...
##  $ DEATHS_DESCRIPTION                : int  NA 3 NA 1 NA NA NA NA NA NA ...
##  $ MISSING                           : int  NA NA NA NA NA NA NA NA NA NA ...
##  $ MISSING_DESCRIPTION               : int  NA NA NA NA NA NA NA NA NA NA ...
##  $ INJURIES                          : int  NA NA NA NA NA NA NA NA NA NA ...
##  $ INJURIES_DESCRIPTION              : int  NA NA NA NA NA NA NA NA NA NA ...
##  $ DAMAGE_MILLIONS_DOLLARS           : num  NA NA NA NA NA NA NA NA NA NA ...
##  $ DAMAGE_DESCRIPTION                : int  NA 3 NA 1 NA NA NA NA NA NA ...
##  $ HOUSES_DESTROYED                  : int  NA NA NA NA NA NA NA NA NA NA ...
##  $ HOUSES_DESTROYED_DESCRIPTION      : int  NA 3 NA NA NA NA NA NA NA NA ...
##  $ TOTAL_DEATHS                      : int  NA NA NA NA NA NA NA NA NA NA ...
##  $ TOTAL_DEATHS_DESCRIPTION          : int  NA 3 NA 1 NA NA NA NA NA NA ...
##  $ TOTAL_MISSING                     : int  NA NA NA NA NA NA NA NA NA NA ...
##  $ TOTAL_MISSING_DESCRIPTION         : int  NA NA NA NA NA NA NA NA NA NA ...
##  $ TOTAL_INJURIES                    : int  NA NA NA NA NA NA NA NA NA NA ...
##  $ TOTAL_INJURIES_DESCRIPTION        : int  NA NA NA NA NA NA NA NA NA NA ...
##  $ TOTAL_DAMAGE_MILLIONS_DOLLARS     : num  NA NA NA NA NA NA NA NA NA NA ...
##  $ TOTAL_DAMAGE_DESCRIPTION          : int  NA 3 NA 1 NA NA NA NA NA NA ...
##  $ TOTAL_HOUSES_DESTROYED            : int  NA NA NA NA NA NA NA NA NA NA ...
##  $ TOTAL_HOUSES_DESTROYED_DESCRIPTION: int  NA 3 NA NA NA NA NA NA NA NA ...
\end{verbatim}

\begin{enumerate}
\def\labelenumi{\arabic{enumi}.}
\setcounter{enumi}{4}
\tightlist
\item
  There are some columns containing categorical data. However, none have
  the class \texttt{"factor"} and are instead \texttt{"character"}. This
  is probably because the data was stored as a CSV file.
\end{enumerate}

\begin{Shaded}
\begin{Highlighting}[]
\KeywordTok{lapply}\NormalTok{(volcanoes, class) }\OperatorTok{==}\StringTok{ "factor"}
\end{Highlighting}
\end{Shaded}

\begin{verbatim}
##                                  X                               Year 
##                              FALSE                              FALSE 
##                              Month                                Day 
##                              FALSE                              FALSE 
##                                TSU                                 EQ 
##                              FALSE                              FALSE 
##                               Name                           Location 
##                              FALSE                              FALSE 
##                            Country                           Latitude 
##                              FALSE                              FALSE 
##                          Longitude                          Elevation 
##                              FALSE                              FALSE 
##                               Type                             Status 
##                              FALSE                              FALSE 
##                               Time                                VEI 
##                              FALSE                              FALSE 
##                              Agent                             DEATHS 
##                              FALSE                              FALSE 
##                 DEATHS_DESCRIPTION                            MISSING 
##                              FALSE                              FALSE 
##                MISSING_DESCRIPTION                           INJURIES 
##                              FALSE                              FALSE 
##               INJURIES_DESCRIPTION            DAMAGE_MILLIONS_DOLLARS 
##                              FALSE                              FALSE 
##                 DAMAGE_DESCRIPTION                   HOUSES_DESTROYED 
##                              FALSE                              FALSE 
##       HOUSES_DESTROYED_DESCRIPTION                       TOTAL_DEATHS 
##                              FALSE                              FALSE 
##           TOTAL_DEATHS_DESCRIPTION                      TOTAL_MISSING 
##                              FALSE                              FALSE 
##          TOTAL_MISSING_DESCRIPTION                     TOTAL_INJURIES 
##                              FALSE                              FALSE 
##         TOTAL_INJURIES_DESCRIPTION      TOTAL_DAMAGE_MILLIONS_DOLLARS 
##                              FALSE                              FALSE 
##           TOTAL_DAMAGE_DESCRIPTION             TOTAL_HOUSES_DESTROYED 
##                              FALSE                              FALSE 
## TOTAL_HOUSES_DESTROYED_DESCRIPTION 
##                              FALSE
\end{verbatim}

\end{document}
