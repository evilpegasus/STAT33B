% Options for packages loaded elsewhere
\PassOptionsToPackage{unicode}{hyperref}
\PassOptionsToPackage{hyphens}{url}
%
\documentclass[
]{article}
\usepackage{lmodern}
\usepackage{amssymb,amsmath}
\usepackage{ifxetex,ifluatex}
\ifnum 0\ifxetex 1\fi\ifluatex 1\fi=0 % if pdftex
  \usepackage[T1]{fontenc}
  \usepackage[utf8]{inputenc}
  \usepackage{textcomp} % provide euro and other symbols
\else % if luatex or xetex
  \usepackage{unicode-math}
  \defaultfontfeatures{Scale=MatchLowercase}
  \defaultfontfeatures[\rmfamily]{Ligatures=TeX,Scale=1}
\fi
% Use upquote if available, for straight quotes in verbatim environments
\IfFileExists{upquote.sty}{\usepackage{upquote}}{}
\IfFileExists{microtype.sty}{% use microtype if available
  \usepackage[]{microtype}
  \UseMicrotypeSet[protrusion]{basicmath} % disable protrusion for tt fonts
}{}
\makeatletter
\@ifundefined{KOMAClassName}{% if non-KOMA class
  \IfFileExists{parskip.sty}{%
    \usepackage{parskip}
  }{% else
    \setlength{\parindent}{0pt}
    \setlength{\parskip}{6pt plus 2pt minus 1pt}}
}{% if KOMA class
  \KOMAoptions{parskip=half}}
\makeatother
\usepackage{xcolor}
\IfFileExists{xurl.sty}{\usepackage{xurl}}{} % add URL line breaks if available
\IfFileExists{bookmark.sty}{\usepackage{bookmark}}{\usepackage{hyperref}}
\hypersetup{
  pdftitle={STAT 33B Workbook 8},
  pdfauthor={Ming Fong (3035619833)},
  hidelinks,
  pdfcreator={LaTeX via pandoc}}
\urlstyle{same} % disable monospaced font for URLs
\usepackage[margin=1in]{geometry}
\usepackage{color}
\usepackage{fancyvrb}
\newcommand{\VerbBar}{|}
\newcommand{\VERB}{\Verb[commandchars=\\\{\}]}
\DefineVerbatimEnvironment{Highlighting}{Verbatim}{commandchars=\\\{\}}
% Add ',fontsize=\small' for more characters per line
\usepackage{framed}
\definecolor{shadecolor}{RGB}{248,248,248}
\newenvironment{Shaded}{\begin{snugshade}}{\end{snugshade}}
\newcommand{\AlertTok}[1]{\textcolor[rgb]{0.94,0.16,0.16}{#1}}
\newcommand{\AnnotationTok}[1]{\textcolor[rgb]{0.56,0.35,0.01}{\textbf{\textit{#1}}}}
\newcommand{\AttributeTok}[1]{\textcolor[rgb]{0.77,0.63,0.00}{#1}}
\newcommand{\BaseNTok}[1]{\textcolor[rgb]{0.00,0.00,0.81}{#1}}
\newcommand{\BuiltInTok}[1]{#1}
\newcommand{\CharTok}[1]{\textcolor[rgb]{0.31,0.60,0.02}{#1}}
\newcommand{\CommentTok}[1]{\textcolor[rgb]{0.56,0.35,0.01}{\textit{#1}}}
\newcommand{\CommentVarTok}[1]{\textcolor[rgb]{0.56,0.35,0.01}{\textbf{\textit{#1}}}}
\newcommand{\ConstantTok}[1]{\textcolor[rgb]{0.00,0.00,0.00}{#1}}
\newcommand{\ControlFlowTok}[1]{\textcolor[rgb]{0.13,0.29,0.53}{\textbf{#1}}}
\newcommand{\DataTypeTok}[1]{\textcolor[rgb]{0.13,0.29,0.53}{#1}}
\newcommand{\DecValTok}[1]{\textcolor[rgb]{0.00,0.00,0.81}{#1}}
\newcommand{\DocumentationTok}[1]{\textcolor[rgb]{0.56,0.35,0.01}{\textbf{\textit{#1}}}}
\newcommand{\ErrorTok}[1]{\textcolor[rgb]{0.64,0.00,0.00}{\textbf{#1}}}
\newcommand{\ExtensionTok}[1]{#1}
\newcommand{\FloatTok}[1]{\textcolor[rgb]{0.00,0.00,0.81}{#1}}
\newcommand{\FunctionTok}[1]{\textcolor[rgb]{0.00,0.00,0.00}{#1}}
\newcommand{\ImportTok}[1]{#1}
\newcommand{\InformationTok}[1]{\textcolor[rgb]{0.56,0.35,0.01}{\textbf{\textit{#1}}}}
\newcommand{\KeywordTok}[1]{\textcolor[rgb]{0.13,0.29,0.53}{\textbf{#1}}}
\newcommand{\NormalTok}[1]{#1}
\newcommand{\OperatorTok}[1]{\textcolor[rgb]{0.81,0.36,0.00}{\textbf{#1}}}
\newcommand{\OtherTok}[1]{\textcolor[rgb]{0.56,0.35,0.01}{#1}}
\newcommand{\PreprocessorTok}[1]{\textcolor[rgb]{0.56,0.35,0.01}{\textit{#1}}}
\newcommand{\RegionMarkerTok}[1]{#1}
\newcommand{\SpecialCharTok}[1]{\textcolor[rgb]{0.00,0.00,0.00}{#1}}
\newcommand{\SpecialStringTok}[1]{\textcolor[rgb]{0.31,0.60,0.02}{#1}}
\newcommand{\StringTok}[1]{\textcolor[rgb]{0.31,0.60,0.02}{#1}}
\newcommand{\VariableTok}[1]{\textcolor[rgb]{0.00,0.00,0.00}{#1}}
\newcommand{\VerbatimStringTok}[1]{\textcolor[rgb]{0.31,0.60,0.02}{#1}}
\newcommand{\WarningTok}[1]{\textcolor[rgb]{0.56,0.35,0.01}{\textbf{\textit{#1}}}}
\usepackage{graphicx}
\makeatletter
\def\maxwidth{\ifdim\Gin@nat@width>\linewidth\linewidth\else\Gin@nat@width\fi}
\def\maxheight{\ifdim\Gin@nat@height>\textheight\textheight\else\Gin@nat@height\fi}
\makeatother
% Scale images if necessary, so that they will not overflow the page
% margins by default, and it is still possible to overwrite the defaults
% using explicit options in \includegraphics[width, height, ...]{}
\setkeys{Gin}{width=\maxwidth,height=\maxheight,keepaspectratio}
% Set default figure placement to htbp
\makeatletter
\def\fps@figure{htbp}
\makeatother
\setlength{\emergencystretch}{3em} % prevent overfull lines
\providecommand{\tightlist}{%
  \setlength{\itemsep}{0pt}\setlength{\parskip}{0pt}}
\setcounter{secnumdepth}{-\maxdimen} % remove section numbering
\ifluatex
  \usepackage{selnolig}  % disable illegal ligatures
\fi

\title{STAT 33B Workbook 8}
\author{Ming Fong (3035619833)}
\date{Oct 22, 2020}

\begin{document}
\maketitle

This workbook is due \textbf{Oct 22, 2020} by 11:59pm PT.

The workbook is organized into sections that correspond to the lecture
videos for the week. Watch a video, then do the corresponding exercises
\emph{before} moving on to the next video.

Workbooks are graded for completeness, so as long as you make a clear
effort to solve each problem, you'll get full credit. That said, make
sure you understand the concepts here, because they're likely to
reappear in homeworks, quizzes, and later lectures.

As you work, write your answers in this notebook. Answer questions with
complete sentences, and put code in code chunks. You can make as many
new code chunks as you like.

In the notebook, you can run the line of code where the cursor is by
pressing \texttt{Ctrl} + \texttt{Enter} on Windows or \texttt{Cmd} +
\texttt{Enter} on Mac OS X. You can run an entire code chunk by clicking
on the green arrow in the upper right corner of the code chunk.

Please do not delete the exercises already in this notebook, because it
may interfere with our grading tools.

You need to submit your work in two places:

\begin{itemize}
\tightlist
\item
  Submit this Rmd file with your edits on bCourses.
\item
  Knit and submit the generated PDF file on Gradescope.
\end{itemize}

If you have any last-minute trouble knitting, \textbf{DON'T PANIC}.
Submit your Rmd file on time and follow up in office hours or on Piazza
to sort out the PDF.

\hypertarget{printing-output}{%
\section{Printing Output}\label{printing-output}}

Watch the ``Printing Output'' lecture video.

\hypertarget{exercise-1}{%
\subsection{Exercise 1}\label{exercise-1}}

The \texttt{scan} function provides a way to collect \emph{input} from
the user.

Write a program that loops endlessly and:

\begin{enumerate}
\def\labelenumi{\arabic{enumi}.}
\tightlist
\item
  Asks the user to input their name.
\item
  Uses \texttt{scan} to collect their input.
\item
  Prints \texttt{"Hello,\ NAME!"} where \texttt{NAME} is replaced by the
  collected name.
\end{enumerate}

The program should exit the loop (and quit) if the user enters the name
\texttt{quit}.

\textbf{YOUR ANSWER GOES HERE:}

\begin{Shaded}
\begin{Highlighting}[]
\ControlFlowTok{while}\NormalTok{ (}\OtherTok{TRUE}\NormalTok{) \{}
\NormalTok{  name =}\StringTok{ }\KeywordTok{scan}\NormalTok{(}\DataTypeTok{what  =} \KeywordTok{character}\NormalTok{(), }\DataTypeTok{n =} \DecValTok{1}\NormalTok{)}
  \ControlFlowTok{if}\NormalTok{ (name }\OperatorTok{==}\StringTok{ "quit"}\NormalTok{) \{}
    \ControlFlowTok{break}
\NormalTok{  \}}
  \KeywordTok{cat}\NormalTok{(}\StringTok{"Hello,"}\NormalTok{, name)}
\NormalTok{\}}
\end{Highlighting}
\end{Shaded}

\hypertarget{messages-warnings-and-errors}{%
\section{Messages, Warnings, and
Errors}\label{messages-warnings-and-errors}}

Watch the ``Messages, Warnings, and Errors'' lecture video.

\hypertarget{exercise-2}{%
\subsection{Exercise 2}\label{exercise-2}}

A vectorized implementation of the \texttt{to\_kelvin} function from
Workbook 6 is:

\begin{Shaded}
\begin{Highlighting}[]
\NormalTok{to\_kelvin =}\StringTok{ }\ControlFlowTok{function}\NormalTok{(temperature, unit) \{}
\NormalTok{  unit =}\StringTok{ }\KeywordTok{match.arg}\NormalTok{(unit, }\KeywordTok{c}\NormalTok{(}\StringTok{"celsius"}\NormalTok{, }\StringTok{"fahrenheit"}\NormalTok{), }\DataTypeTok{several.ok =} \OtherTok{TRUE}\NormalTok{)}

  \CommentTok{\# First convert Fahrenheit to Celsius.}
\NormalTok{  is\_f =}\StringTok{ }\NormalTok{unit }\OperatorTok{==}\StringTok{ "fahrenheit"}
\NormalTok{  temperature[is\_f] =}\StringTok{ }\NormalTok{(temperature[is\_f] }\OperatorTok{{-}}\StringTok{ }\DecValTok{32}\NormalTok{) }\OperatorTok{*}\StringTok{ }\DecValTok{5} \OperatorTok{/}\StringTok{ }\DecValTok{9}

\NormalTok{  temperature }\OperatorTok{+}\StringTok{ }\FloatTok{273.15}
\NormalTok{\}}
\end{Highlighting}
\end{Shaded}

Write a modified version of \texttt{to\_kelvin} that checks for
potential problems. In particular, your version should check the
assumptions that:

\begin{itemize}
\tightlist
\item
  \texttt{temperature} is numeric.
\item
  \texttt{temperature} and \texttt{unit} are the same length, or
  \texttt{unit} has length 1.
\end{itemize}

Your version should raise an error (with a descriptive message) if
either of these assumptions don't hold.

Test your function to show that it checks for potential problems. You
can use \texttt{error\ =\ TRUE} on an RMarkdown code chunk to allow
errors when knitting.

\textbf{YOUR ANSWER GOES HERE:}

\begin{Shaded}
\begin{Highlighting}[]
\NormalTok{to\_kelvin =}\StringTok{ }\ControlFlowTok{function}\NormalTok{(temperature, unit) \{}
\NormalTok{  unit =}\StringTok{ }\KeywordTok{match.arg}\NormalTok{(unit, }\KeywordTok{c}\NormalTok{(}\StringTok{"celsius"}\NormalTok{, }\StringTok{"fahrenheit"}\NormalTok{), }\DataTypeTok{several.ok =} \OtherTok{TRUE}\NormalTok{)}

  \CommentTok{\# Check for errors in params}
  \ControlFlowTok{if}\NormalTok{ (}\KeywordTok{class}\NormalTok{(temperature) }\OperatorTok{!=}\StringTok{ "numeric"}\NormalTok{) \{}
    \KeywordTok{stop}\NormalTok{(}\StringTok{"temperature must be numeric"}\NormalTok{)}
\NormalTok{  \}}
  \ControlFlowTok{if}\NormalTok{ (}\KeywordTok{length}\NormalTok{(temperature) }\OperatorTok{!=}\StringTok{ }\KeywordTok{length}\NormalTok{(unit) }\OperatorTok{|}\StringTok{ }\KeywordTok{length}\NormalTok{(unit) }\OperatorTok{!=}\StringTok{ }\DecValTok{1}\NormalTok{) \{}
    \KeywordTok{stop}\NormalTok{(}\StringTok{"unit must be the same length as temperature or be length 1"}\NormalTok{)}
\NormalTok{  \}}

  \CommentTok{\# First convert Fahrenheit to Celsius.}
\NormalTok{  is\_f =}\StringTok{ }\NormalTok{unit }\OperatorTok{==}\StringTok{ "fahrenheit"}
\NormalTok{  temperature[is\_f] =}\StringTok{ }\NormalTok{(temperature[is\_f] }\OperatorTok{{-}}\StringTok{ }\DecValTok{32}\NormalTok{) }\OperatorTok{*}\StringTok{ }\DecValTok{5} \OperatorTok{/}\StringTok{ }\DecValTok{9}

\NormalTok{  temperature }\OperatorTok{+}\StringTok{ }\FloatTok{273.15}
\NormalTok{\}}
\KeywordTok{to\_kelvin}\NormalTok{(}\DataTypeTok{temperature =} \KeywordTok{c}\NormalTok{(}\DecValTok{1}\NormalTok{, }\DecValTok{2}\NormalTok{, }\DecValTok{3}\NormalTok{), }\DataTypeTok{unit =} \KeywordTok{c}\NormalTok{(}\StringTok{"c"}\NormalTok{, }\StringTok{"c"}\NormalTok{))}
\end{Highlighting}
\end{Shaded}

\begin{verbatim}
## Error in to_kelvin(temperature = c(1, 2, 3), unit = c("c", "c")): unit must be the same length as temperature or be length 1
\end{verbatim}

\begin{Shaded}
\begin{Highlighting}[]
\KeywordTok{to\_kelvin}\NormalTok{(}\DataTypeTok{temperature =} \StringTok{"Apple pie"}\NormalTok{, }\DataTypeTok{unit =} \StringTok{"c"}\NormalTok{)}
\end{Highlighting}
\end{Shaded}

\begin{verbatim}
## Error in to_kelvin(temperature = "Apple pie", unit = "c"): temperature must be numeric
\end{verbatim}

\hypertarget{handling-errors}{%
\section{Handling Errors}\label{handling-errors}}

Watch the ``Handling Errors'' lecture video.

No exercises for this section.

\hypertarget{global-options}{%
\section{Global Options}\label{global-options}}

Watch the ``Global Options'' lecture video.

\hypertarget{exercise-3}{%
\subsection{Exercise 3}\label{exercise-3}}

Skim \texttt{?options} and \texttt{?Startup}.

Create or edit your \texttt{.Rprofile} file to set an option (or several
options).

Check that your option is actually set when you restart R (you can call
\texttt{options} without any arguments to see your current options).

In your answer here, describe which option you set and include the code
you added to \texttt{.Rprofile}.

\textbf{YOUR ANSWER GOES HERE:}

I changed the \texttt{warn} option to 1 as recommended by the professor.
I also printed a welcome message on start.

\begin{Shaded}
\begin{Highlighting}[]
\CommentTok{\# Put your new .Rprofile code here.}
\CommentTok{\# Things you might want to change}

\CommentTok{\# options(papersize="a4")}
\CommentTok{\# options(editor="notepad")}
\CommentTok{\# options(pager="internal")}

\CommentTok{\# set the default help type}
\CommentTok{\# options(help\_type="text")}
  \KeywordTok{options}\NormalTok{(}\DataTypeTok{help\_type=}\StringTok{"html"}\NormalTok{)}

\CommentTok{\# set a site library}
\CommentTok{\# .Library.site \textless{}{-} file.path(chartr("\textbackslash{}\textbackslash{}", "/", R.home()), "site{-}library")}

\CommentTok{\# set a CRAN mirror}
\CommentTok{\# local(\{r \textless{}{-} getOption("repos")}
\CommentTok{\#       r["CRAN"] \textless{}{-} "http://my.local.cran"}
\CommentTok{\#       options(repos=r)\})}

\CommentTok{\# Give a fortune cookie, but only to interactive sessions}
\CommentTok{\# (This would need the fortunes package to be installed.)}
\CommentTok{\#  if (interactive()) }
\CommentTok{\#    fortunes::fortune()}

\KeywordTok{options}\NormalTok{(}\DataTypeTok{warn =} \DecValTok{1}\NormalTok{)}

\NormalTok{.First =}\StringTok{ }\ControlFlowTok{function}\NormalTok{() \{}
  \KeywordTok{cat}\NormalTok{(}\StringTok{"Welcome Ming Fong"}\NormalTok{, }\KeywordTok{date}\NormalTok{(), }\StringTok{"}\CharTok{\textbackslash{}n}\StringTok{"}\NormalTok{)}
\NormalTok{\}}
\end{Highlighting}
\end{Shaded}

\hypertarget{preventing-bugs}{%
\section{Preventing Bugs}\label{preventing-bugs}}

Watch the ``Preventing Bugs'' lecture video.

No exercises for this section.

\hypertarget{the-r-debugger}{%
\section{The R Debugger}\label{the-r-debugger}}

Watch the ``The R Debugger'' lecture video.

No exercises for this section.

\hypertarget{other-debugging-functions}{%
\section{Other Debugging Functions}\label{other-debugging-functions}}

Watch the ``Other Debugging Functions'' lecture video.

No exercises for this section. You're done!

\end{document}
