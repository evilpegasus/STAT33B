% Options for packages loaded elsewhere
\PassOptionsToPackage{unicode}{hyperref}
\PassOptionsToPackage{hyphens}{url}
%
\documentclass[
]{article}
\usepackage{lmodern}
\usepackage{amssymb,amsmath}
\usepackage{ifxetex,ifluatex}
\ifnum 0\ifxetex 1\fi\ifluatex 1\fi=0 % if pdftex
  \usepackage[T1]{fontenc}
  \usepackage[utf8]{inputenc}
  \usepackage{textcomp} % provide euro and other symbols
\else % if luatex or xetex
  \usepackage{unicode-math}
  \defaultfontfeatures{Scale=MatchLowercase}
  \defaultfontfeatures[\rmfamily]{Ligatures=TeX,Scale=1}
\fi
% Use upquote if available, for straight quotes in verbatim environments
\IfFileExists{upquote.sty}{\usepackage{upquote}}{}
\IfFileExists{microtype.sty}{% use microtype if available
  \usepackage[]{microtype}
  \UseMicrotypeSet[protrusion]{basicmath} % disable protrusion for tt fonts
}{}
\makeatletter
\@ifundefined{KOMAClassName}{% if non-KOMA class
  \IfFileExists{parskip.sty}{%
    \usepackage{parskip}
  }{% else
    \setlength{\parindent}{0pt}
    \setlength{\parskip}{6pt plus 2pt minus 1pt}}
}{% if KOMA class
  \KOMAoptions{parskip=half}}
\makeatother
\usepackage{xcolor}
\IfFileExists{xurl.sty}{\usepackage{xurl}}{} % add URL line breaks if available
\IfFileExists{bookmark.sty}{\usepackage{bookmark}}{\usepackage{hyperref}}
\hypersetup{
  pdftitle={STAT 33B Workbook 12},
  pdfauthor={Ming Fong (3035619833)},
  hidelinks,
  pdfcreator={LaTeX via pandoc}}
\urlstyle{same} % disable monospaced font for URLs
\usepackage[margin=1in]{geometry}
\usepackage{color}
\usepackage{fancyvrb}
\newcommand{\VerbBar}{|}
\newcommand{\VERB}{\Verb[commandchars=\\\{\}]}
\DefineVerbatimEnvironment{Highlighting}{Verbatim}{commandchars=\\\{\}}
% Add ',fontsize=\small' for more characters per line
\usepackage{framed}
\definecolor{shadecolor}{RGB}{248,248,248}
\newenvironment{Shaded}{\begin{snugshade}}{\end{snugshade}}
\newcommand{\AlertTok}[1]{\textcolor[rgb]{0.94,0.16,0.16}{#1}}
\newcommand{\AnnotationTok}[1]{\textcolor[rgb]{0.56,0.35,0.01}{\textbf{\textit{#1}}}}
\newcommand{\AttributeTok}[1]{\textcolor[rgb]{0.77,0.63,0.00}{#1}}
\newcommand{\BaseNTok}[1]{\textcolor[rgb]{0.00,0.00,0.81}{#1}}
\newcommand{\BuiltInTok}[1]{#1}
\newcommand{\CharTok}[1]{\textcolor[rgb]{0.31,0.60,0.02}{#1}}
\newcommand{\CommentTok}[1]{\textcolor[rgb]{0.56,0.35,0.01}{\textit{#1}}}
\newcommand{\CommentVarTok}[1]{\textcolor[rgb]{0.56,0.35,0.01}{\textbf{\textit{#1}}}}
\newcommand{\ConstantTok}[1]{\textcolor[rgb]{0.00,0.00,0.00}{#1}}
\newcommand{\ControlFlowTok}[1]{\textcolor[rgb]{0.13,0.29,0.53}{\textbf{#1}}}
\newcommand{\DataTypeTok}[1]{\textcolor[rgb]{0.13,0.29,0.53}{#1}}
\newcommand{\DecValTok}[1]{\textcolor[rgb]{0.00,0.00,0.81}{#1}}
\newcommand{\DocumentationTok}[1]{\textcolor[rgb]{0.56,0.35,0.01}{\textbf{\textit{#1}}}}
\newcommand{\ErrorTok}[1]{\textcolor[rgb]{0.64,0.00,0.00}{\textbf{#1}}}
\newcommand{\ExtensionTok}[1]{#1}
\newcommand{\FloatTok}[1]{\textcolor[rgb]{0.00,0.00,0.81}{#1}}
\newcommand{\FunctionTok}[1]{\textcolor[rgb]{0.00,0.00,0.00}{#1}}
\newcommand{\ImportTok}[1]{#1}
\newcommand{\InformationTok}[1]{\textcolor[rgb]{0.56,0.35,0.01}{\textbf{\textit{#1}}}}
\newcommand{\KeywordTok}[1]{\textcolor[rgb]{0.13,0.29,0.53}{\textbf{#1}}}
\newcommand{\NormalTok}[1]{#1}
\newcommand{\OperatorTok}[1]{\textcolor[rgb]{0.81,0.36,0.00}{\textbf{#1}}}
\newcommand{\OtherTok}[1]{\textcolor[rgb]{0.56,0.35,0.01}{#1}}
\newcommand{\PreprocessorTok}[1]{\textcolor[rgb]{0.56,0.35,0.01}{\textit{#1}}}
\newcommand{\RegionMarkerTok}[1]{#1}
\newcommand{\SpecialCharTok}[1]{\textcolor[rgb]{0.00,0.00,0.00}{#1}}
\newcommand{\SpecialStringTok}[1]{\textcolor[rgb]{0.31,0.60,0.02}{#1}}
\newcommand{\StringTok}[1]{\textcolor[rgb]{0.31,0.60,0.02}{#1}}
\newcommand{\VariableTok}[1]{\textcolor[rgb]{0.00,0.00,0.00}{#1}}
\newcommand{\VerbatimStringTok}[1]{\textcolor[rgb]{0.31,0.60,0.02}{#1}}
\newcommand{\WarningTok}[1]{\textcolor[rgb]{0.56,0.35,0.01}{\textbf{\textit{#1}}}}
\usepackage{graphicx}
\makeatletter
\def\maxwidth{\ifdim\Gin@nat@width>\linewidth\linewidth\else\Gin@nat@width\fi}
\def\maxheight{\ifdim\Gin@nat@height>\textheight\textheight\else\Gin@nat@height\fi}
\makeatother
% Scale images if necessary, so that they will not overflow the page
% margins by default, and it is still possible to overwrite the defaults
% using explicit options in \includegraphics[width, height, ...]{}
\setkeys{Gin}{width=\maxwidth,height=\maxheight,keepaspectratio}
% Set default figure placement to htbp
\makeatletter
\def\fps@figure{htbp}
\makeatother
\setlength{\emergencystretch}{3em} % prevent overfull lines
\providecommand{\tightlist}{%
  \setlength{\itemsep}{0pt}\setlength{\parskip}{0pt}}
\setcounter{secnumdepth}{-\maxdimen} % remove section numbering
\ifluatex
  \usepackage{selnolig}  % disable illegal ligatures
\fi

\title{STAT 33B Workbook 12}
\author{Ming Fong (3035619833)}
\date{Nov 19, 2020}

\begin{document}
\maketitle

This workbook is due \textbf{Nov 19, 2020} by 11:59pm PT.

The workbook is organized into sections that correspond to the lecture
videos for the week. Watch a video, then do the corresponding exercises
\emph{before} moving on to the next video.

Workbooks are graded for completeness, so as long as you make a clear
effort to solve each problem, you'll get full credit. That said, make
sure you understand the concepts here, because they're likely to
reappear in homeworks, quizzes, and later lectures.

As you work, write your answers in this notebook. Answer questions with
complete sentences, and put code in code chunks. You can make as many
new code chunks as you like.

In the notebook, you can run the line of code where the cursor is by
pressing \texttt{Ctrl} + \texttt{Enter} on Windows or \texttt{Cmd} +
\texttt{Enter} on Mac OS X. You can run an entire code chunk by clicking
on the green arrow in the upper right corner of the code chunk.

Please do not delete the exercises already in this notebook, because it
may interfere with our grading tools.

You need to submit your work in two places:

\begin{itemize}
\tightlist
\item
  Submit this Rmd file with your edits on bCourses.
\item
  Knit and submit the generated PDF file on Gradescope.
\end{itemize}

If you have any last-minute trouble knitting, \textbf{DON'T PANIC}.
Submit your Rmd file on time and follow up in office hours or on Piazza
to sort out the PDF.

\hypertarget{tidy-data}{%
\section{Tidy Data}\label{tidy-data}}

Watch the ``Tidy Data'' lecture video.

No exercises for this section.

\hypertarget{columns-into-rows}{%
\section{Columns into Rows}\label{columns-into-rows}}

Watch the ``Columns into Rows'' lecture video.

No exercises for this section.

\hypertarget{rows-into-columns}{%
\section{Rows into Columns}\label{rows-into-columns}}

Watch the ``Rows into Columns'' lecture video.

\hypertarget{string-processing}{%
\section{String Processing}\label{string-processing}}

Watch the ``String Processing'' lecture video.

\hypertarget{exercise-1}{%
\subsection{Exercise 1}\label{exercise-1}}

Visit the \href{https://stringr.tidyverse.org/}{stringr documentation}.

How does the \texttt{str\_sub} function work? Give 3 examples, including
1 that shows how to use \texttt{str\_sub} to reassign part of a string.

\textbf{YOUR ANSWER GOES HERE:}

The \texttt{str\_sub} function subsets strings:

\begin{Shaded}
\begin{Highlighting}[]
\KeywordTok{library}\NormalTok{(stringr)}
\NormalTok{x =}\StringTok{ "Hello world"}
\KeywordTok{str\_sub}\NormalTok{(x, }\DecValTok{1}\NormalTok{, }\DecValTok{5}\NormalTok{)  }\CommentTok{\# Get the first 5 characters ("Hello")}
\end{Highlighting}
\end{Shaded}

\begin{verbatim}
## [1] "Hello"
\end{verbatim}

\begin{Shaded}
\begin{Highlighting}[]
\KeywordTok{str\_sub}\NormalTok{(x, }\DecValTok{{-}5}\NormalTok{) }\CommentTok{\# Get the last 5 characters ("world")}
\end{Highlighting}
\end{Shaded}

\begin{verbatim}
## [1] "world"
\end{verbatim}

\begin{Shaded}
\begin{Highlighting}[]
\KeywordTok{str\_sub}\NormalTok{(x, }\DecValTok{{-}5}\NormalTok{) =}\StringTok{ "everybody"}  \CommentTok{\# Replace the last 5 characters with "everybody"}
\NormalTok{x}
\end{Highlighting}
\end{Shaded}

\begin{verbatim}
## [1] "Hello everybody"
\end{verbatim}

\hypertarget{exercise-2}{%
\subsection{Exercise 2}\label{exercise-2}}

Complete the \texttt{table3} example from the lecture by showing the
entire process for converting \texttt{table3} into a tidy data frame.

\emph{Hint: All of the functions needed to do this were mentioned in the
lecture or in other recent lectures.}

\textbf{YOUR ANSWER GOES HERE:}

\begin{Shaded}
\begin{Highlighting}[]
\KeywordTok{library}\NormalTok{(tidyr)}
\NormalTok{table3}
\end{Highlighting}
\end{Shaded}

\begin{verbatim}
## # A tibble: 6 x 3
##   country      year rate             
## * <chr>       <int> <chr>            
## 1 Afghanistan  1999 745/19987071     
## 2 Afghanistan  2000 2666/20595360    
## 3 Brazil       1999 37737/172006362  
## 4 Brazil       2000 80488/174504898  
## 5 China        1999 212258/1272915272
## 6 China        2000 213766/1280428583
\end{verbatim}

\begin{Shaded}
\begin{Highlighting}[]
\NormalTok{splits =}\StringTok{ }\KeywordTok{str\_split\_fixed}\NormalTok{(table3}\OperatorTok{$}\NormalTok{rate, }\StringTok{"/"}\NormalTok{, }\DecValTok{2}\NormalTok{)}
\NormalTok{tidy\_table3 =}\StringTok{ }\NormalTok{table3[}\OperatorTok{{-}}\DecValTok{3}\NormalTok{]}
\NormalTok{tidy\_table3[}\StringTok{"cases"}\NormalTok{] =}\StringTok{ }\NormalTok{splits[,}\DecValTok{1}\NormalTok{]}
\NormalTok{tidy\_table3[}\StringTok{"population"}\NormalTok{] =}\StringTok{ }\NormalTok{splits[,}\DecValTok{2}\NormalTok{]}
\NormalTok{tidy\_table3}
\end{Highlighting}
\end{Shaded}

\begin{verbatim}
## # A tibble: 6 x 4
##   country      year cases  population
##   <chr>       <int> <chr>  <chr>     
## 1 Afghanistan  1999 745    19987071  
## 2 Afghanistan  2000 2666   20595360  
## 3 Brazil       1999 37737  172006362 
## 4 Brazil       2000 80488  174504898 
## 5 China        1999 212258 1272915272
## 6 China        2000 213766 1280428583
\end{verbatim}

\hypertarget{regular-expressions}{%
\section{Regular Expressions}\label{regular-expressions}}

Watch the ``Regular Expressions'' lecture video.

If you use RStudio, the RegExplain RStudio addin makes it easier to
learn how to use regular expressions. You can find information about how
to install RegExplain on the documentation page for stringr.

If you decide not to install RegExplain, there are many online regular
expression testers that can be helpful when learning. For example, I
like \url{https://regex101.com}.

\hypertarget{exercise-3}{%
\subsection{Exercise 3}\label{exercise-3}}

The lecture video ``Printing Output'' described how R interprets
backslashes in strings as escape sequences. For instance, the string
\texttt{"\textbackslash{}t"} is a tab character.

Because backslashes have a special meaning, to put a literal backslash
in a string, you have to write the backslash twice:
\texttt{"\textbackslash{}\textbackslash{}"}.

Regular expressions also use backslash as a special character. In a
regular expression, putting a backslash in front of a metacharacter
causes it to be interpreted literally. For instance, the pattern
\texttt{"\textbackslash{}."} matches a single, literal dot (rather than
being a wildcard).

R's rule for backslashes interacts with regex's rule for backslashes in
an unfortunate way. Regex patterns in R are just strings, and R
interprets \texttt{"\textbackslash{}."} (or backslash followed by any
other character) as an escape sequence. Since we need a literal
backslash, we have to write \texttt{"\textbackslash{}\textbackslash{}."}
in R. Then the regular expression system sees this as
\texttt{"\textbackslash{}."}, and searches for a literal \texttt{"."}.

This interaction is especially bad if you want to search for a literal
backslash with a regular expression. You'd need to write
\texttt{"\textbackslash{}\textbackslash{}\textbackslash{}\textbackslash{}"}!

As a remedy, R version 4.0 introduced \emph{raw strings}. In a raw
string, R always interprets backslashes literally (rather than as escape
sequences).

\begin{enumerate}
\def\labelenumi{\arabic{enumi}.}
\item
  Raw strings are documented in \texttt{?Quotes}. Find the section in
  that file and read it.
\item
  Give an example of creating a raw string.
\item
  Write a vectorized function \texttt{has\_backslash} with parameter
  \texttt{x} that returns \texttt{TRUE} if \texttt{x} contains a
  backslash and \texttt{FALSE} otherwise. In your function, use
  \texttt{str\_detect} with a raw string for the pattern.
\end{enumerate}

\textbf{YOUR ANSWER GOES HERE:}

\begin{Shaded}
\begin{Highlighting}[]
\NormalTok{r}\StringTok{"(}\CharTok{\textbackslash{}t}\StringTok{elephone)"}
\end{Highlighting}
\end{Shaded}

\begin{verbatim}
## [1] "\\telephone"
\end{verbatim}

\begin{Shaded}
\begin{Highlighting}[]
\NormalTok{has\_backslash =}\StringTok{ }\ControlFlowTok{function}\NormalTok{(x) \{}
   \KeywordTok{str\_detect}\NormalTok{(x, r}\StringTok{"(}\CharTok{\textbackslash{}\textbackslash{}}\StringTok{)"}\NormalTok{)}
\NormalTok{\}}

\KeywordTok{has\_backslash}\NormalTok{(}\KeywordTok{c}\NormalTok{(r}\StringTok{"(no backslash here)"}\NormalTok{, r}\StringTok{"(here \textbackslash{}is }\CharTok{\textbackslash{}a}\StringTok{ }\CharTok{\textbackslash{}b}\StringTok{ackslash)"}\NormalTok{))}
\end{Highlighting}
\end{Shaded}

\begin{verbatim}
## [1] FALSE  TRUE
\end{verbatim}

\hypertarget{exercise-4}{%
\subsection{Exercise 4}\label{exercise-4}}

Write a function \texttt{extract\_phone} that extracts a phone number
from a string. Your function should return the phone number as a string
in the format \texttt{NNN-NNNN} without any other characters.

Test your function on the following strings:

\begin{Shaded}
\begin{Highlighting}[]
\NormalTok{ex1 =}\StringTok{ "Phone: 555{-}2920"}
\NormalTok{ex2 =}\StringTok{ "Hi! The number you\textquotesingle{}ve called is 555{-}3131!"}
\NormalTok{ex3 =}\StringTok{ "The phone for unit 4342 is 555{-}9753."}
\end{Highlighting}
\end{Shaded}

\emph{Hint 1: Use \texttt{str\_match} to extract the number.}

\emph{Hint 2: Take advantage of non-numeric characters that mark the
boundaries of a phone number.}

\textbf{YOUR ANSWER GOES HERE:}

\begin{Shaded}
\begin{Highlighting}[]
\NormalTok{extract\_phone =}\StringTok{ }\ControlFlowTok{function}\NormalTok{(x) \{}
   \KeywordTok{str\_match}\NormalTok{(x, r}\StringTok{"(\textbackslash{}d\{3\}{-}\textbackslash{}d\{4\})"}\NormalTok{)}
\NormalTok{\}}
\KeywordTok{extract\_phone}\NormalTok{(ex1)}
\end{Highlighting}
\end{Shaded}

\begin{verbatim}
##      [,1]      
## [1,] "555-2920"
\end{verbatim}

\begin{Shaded}
\begin{Highlighting}[]
\KeywordTok{extract\_phone}\NormalTok{(ex2)}
\end{Highlighting}
\end{Shaded}

\begin{verbatim}
##      [,1]      
## [1,] "555-3131"
\end{verbatim}

\begin{Shaded}
\begin{Highlighting}[]
\KeywordTok{extract\_phone}\NormalTok{(ex3)}
\end{Highlighting}
\end{Shaded}

\begin{verbatim}
##      [,1]      
## [1,] "555-9753"
\end{verbatim}

\end{document}
