% Options for packages loaded elsewhere
\PassOptionsToPackage{unicode}{hyperref}
\PassOptionsToPackage{hyphens}{url}
%
\documentclass[
]{article}
\usepackage{lmodern}
\usepackage{amssymb,amsmath}
\usepackage{ifxetex,ifluatex}
\ifnum 0\ifxetex 1\fi\ifluatex 1\fi=0 % if pdftex
  \usepackage[T1]{fontenc}
  \usepackage[utf8]{inputenc}
  \usepackage{textcomp} % provide euro and other symbols
\else % if luatex or xetex
  \usepackage{unicode-math}
  \defaultfontfeatures{Scale=MatchLowercase}
  \defaultfontfeatures[\rmfamily]{Ligatures=TeX,Scale=1}
\fi
% Use upquote if available, for straight quotes in verbatim environments
\IfFileExists{upquote.sty}{\usepackage{upquote}}{}
\IfFileExists{microtype.sty}{% use microtype if available
  \usepackage[]{microtype}
  \UseMicrotypeSet[protrusion]{basicmath} % disable protrusion for tt fonts
}{}
\makeatletter
\@ifundefined{KOMAClassName}{% if non-KOMA class
  \IfFileExists{parskip.sty}{%
    \usepackage{parskip}
  }{% else
    \setlength{\parindent}{0pt}
    \setlength{\parskip}{6pt plus 2pt minus 1pt}}
}{% if KOMA class
  \KOMAoptions{parskip=half}}
\makeatother
\usepackage{xcolor}
\IfFileExists{xurl.sty}{\usepackage{xurl}}{} % add URL line breaks if available
\IfFileExists{bookmark.sty}{\usepackage{bookmark}}{\usepackage{hyperref}}
\hypersetup{
  pdftitle={STAT 33B Homework 6 Solutions},
  pdfauthor={Ming Fong (3035619833)},
  hidelinks,
  pdfcreator={LaTeX via pandoc}}
\urlstyle{same} % disable monospaced font for URLs
\usepackage[margin=1in]{geometry}
\usepackage{color}
\usepackage{fancyvrb}
\newcommand{\VerbBar}{|}
\newcommand{\VERB}{\Verb[commandchars=\\\{\}]}
\DefineVerbatimEnvironment{Highlighting}{Verbatim}{commandchars=\\\{\}}
% Add ',fontsize=\small' for more characters per line
\usepackage{framed}
\definecolor{shadecolor}{RGB}{248,248,248}
\newenvironment{Shaded}{\begin{snugshade}}{\end{snugshade}}
\newcommand{\AlertTok}[1]{\textcolor[rgb]{0.94,0.16,0.16}{#1}}
\newcommand{\AnnotationTok}[1]{\textcolor[rgb]{0.56,0.35,0.01}{\textbf{\textit{#1}}}}
\newcommand{\AttributeTok}[1]{\textcolor[rgb]{0.77,0.63,0.00}{#1}}
\newcommand{\BaseNTok}[1]{\textcolor[rgb]{0.00,0.00,0.81}{#1}}
\newcommand{\BuiltInTok}[1]{#1}
\newcommand{\CharTok}[1]{\textcolor[rgb]{0.31,0.60,0.02}{#1}}
\newcommand{\CommentTok}[1]{\textcolor[rgb]{0.56,0.35,0.01}{\textit{#1}}}
\newcommand{\CommentVarTok}[1]{\textcolor[rgb]{0.56,0.35,0.01}{\textbf{\textit{#1}}}}
\newcommand{\ConstantTok}[1]{\textcolor[rgb]{0.00,0.00,0.00}{#1}}
\newcommand{\ControlFlowTok}[1]{\textcolor[rgb]{0.13,0.29,0.53}{\textbf{#1}}}
\newcommand{\DataTypeTok}[1]{\textcolor[rgb]{0.13,0.29,0.53}{#1}}
\newcommand{\DecValTok}[1]{\textcolor[rgb]{0.00,0.00,0.81}{#1}}
\newcommand{\DocumentationTok}[1]{\textcolor[rgb]{0.56,0.35,0.01}{\textbf{\textit{#1}}}}
\newcommand{\ErrorTok}[1]{\textcolor[rgb]{0.64,0.00,0.00}{\textbf{#1}}}
\newcommand{\ExtensionTok}[1]{#1}
\newcommand{\FloatTok}[1]{\textcolor[rgb]{0.00,0.00,0.81}{#1}}
\newcommand{\FunctionTok}[1]{\textcolor[rgb]{0.00,0.00,0.00}{#1}}
\newcommand{\ImportTok}[1]{#1}
\newcommand{\InformationTok}[1]{\textcolor[rgb]{0.56,0.35,0.01}{\textbf{\textit{#1}}}}
\newcommand{\KeywordTok}[1]{\textcolor[rgb]{0.13,0.29,0.53}{\textbf{#1}}}
\newcommand{\NormalTok}[1]{#1}
\newcommand{\OperatorTok}[1]{\textcolor[rgb]{0.81,0.36,0.00}{\textbf{#1}}}
\newcommand{\OtherTok}[1]{\textcolor[rgb]{0.56,0.35,0.01}{#1}}
\newcommand{\PreprocessorTok}[1]{\textcolor[rgb]{0.56,0.35,0.01}{\textit{#1}}}
\newcommand{\RegionMarkerTok}[1]{#1}
\newcommand{\SpecialCharTok}[1]{\textcolor[rgb]{0.00,0.00,0.00}{#1}}
\newcommand{\SpecialStringTok}[1]{\textcolor[rgb]{0.31,0.60,0.02}{#1}}
\newcommand{\StringTok}[1]{\textcolor[rgb]{0.31,0.60,0.02}{#1}}
\newcommand{\VariableTok}[1]{\textcolor[rgb]{0.00,0.00,0.00}{#1}}
\newcommand{\VerbatimStringTok}[1]{\textcolor[rgb]{0.31,0.60,0.02}{#1}}
\newcommand{\WarningTok}[1]{\textcolor[rgb]{0.56,0.35,0.01}{\textbf{\textit{#1}}}}
\usepackage{graphicx}
\makeatletter
\def\maxwidth{\ifdim\Gin@nat@width>\linewidth\linewidth\else\Gin@nat@width\fi}
\def\maxheight{\ifdim\Gin@nat@height>\textheight\textheight\else\Gin@nat@height\fi}
\makeatother
% Scale images if necessary, so that they will not overflow the page
% margins by default, and it is still possible to overwrite the defaults
% using explicit options in \includegraphics[width, height, ...]{}
\setkeys{Gin}{width=\maxwidth,height=\maxheight,keepaspectratio}
% Set default figure placement to htbp
\makeatletter
\def\fps@figure{htbp}
\makeatother
\setlength{\emergencystretch}{3em} % prevent overfull lines
\providecommand{\tightlist}{%
  \setlength{\itemsep}{0pt}\setlength{\parskip}{0pt}}
\setcounter{secnumdepth}{-\maxdimen} % remove section numbering
\ifluatex
  \usepackage{selnolig}  % disable illegal ligatures
\fi

\title{STAT 33B Homework 6 Solutions}
\author{Ming Fong (3035619833)}
\date{Nov 19, 2020}

\begin{document}
\maketitle

This homework is due \textbf{Nov 19, 2020} by 11:59pm PT.

Homeworks are graded for correctness.

As you work, write your answers in this notebook. Answer questions with
complete sentences, and put code in code chunks. You can make as many
new code chunks as you like.

Please do not delete the exercises already in this notebook, because it
may interfere with our grading tools.

You need to submit your work in two places:

\begin{itemize}
\tightlist
\item
  Submit this Rmd file with your edits on bCourses.
\item
  Knit and submit the generated PDF file on Gradescope.
\end{itemize}

If you have any last-minute trouble knitting, \textbf{DON'T PANIC}.
Submit your Rmd file on time and follow up in office hours or on Piazza
to sort out the PDF.

\hypertarget{profiling}{%
\section{Profiling}\label{profiling}}

The purpose of this homework is to practice profiling and optimizing
code.

The function in the next section is quite slow. The exercises will walk
you through various performance improvements to the function.

\emph{Note 1:} When benchmarking or profiling code, other programs
running on your computer can affect the results. So when you benchmark
or profile code, first make sure you don't have lots of other programs
running. Close your web browser, word processor, media player, etc.

\emph{Note 2:} Each time you profile code the results will vary
slightly, even if you don't change the code. Thus if you are
investigating an optimization that appears to have a small effect, you
may want to profile the code several times to determine whether the
effect is real or due to random variation.

\emph{Note 3:} The code cells in this notebook are all set up with
\texttt{eval=FALSE} so that they don't actually run the code they
contain. Your notebook may fail to knit if you put profiler code in a
cell that doesn't have \texttt{eval=FALSE}. You do not need to show us
the output from profvis; just tell us about what you observed.

\hypertarget{the-function}{%
\subsection{The Function}\label{the-function}}

The \texttt{sample\_markov} function, shown below, generates a vector
that represents the state (A or B) of a system over time. Each element
corresponds to one time step.

At each time step, the system transitions to a new state. If the system
is in state A, then at the next time step it can either stay in state A,
or switch to state B. If the system is in state B, then at the next time
step it can either switch to state A, or stay in state B. The transition
is random, with a fixed probability for each of the possible
transitions.

For instance, suppose the probability of A to B is 0.3, and the
probability of B to B is 0.4. We can use these to compute the other
probabilties. The probability of A to A must be \(1 - 0.3 = 0.7\), and
the probability of B to A must be \(1 - 0.4 = 0.6\).

This random system is called a 2-state Markov chain.

The parameters of the function are:

\begin{itemize}
\tightlist
\item
  \texttt{n}, the number of time steps to run the system, which equals
  the length of the returned vector.
\item
  \texttt{p}, a length-2 vector that contains the probability of A to B
  and B to B, respectively.
\item
  \texttt{init}, the initial state of the system.
\end{itemize}

The function is defined as:

\begin{Shaded}
\begin{Highlighting}[]
\NormalTok{sample\_markov =}\StringTok{ }\ControlFlowTok{function}\NormalTok{(n, }\DataTypeTok{p =} \KeywordTok{c}\NormalTok{(}\FloatTok{0.6}\NormalTok{, }\FloatTok{0.7}\NormalTok{), }\DataTypeTok{init =} \StringTok{"A"}\NormalTok{) \{}
\NormalTok{  chain =}\StringTok{ }\NormalTok{init}

  \ControlFlowTok{for}\NormalTok{ (i }\ControlFlowTok{in} \DecValTok{1}\OperatorTok{:}\NormalTok{n) \{}
    \CommentTok{\# Check the previous state, chain[i].}
    \ControlFlowTok{if}\NormalTok{ (chain[i] }\OperatorTok{==}\StringTok{ "A"}\NormalTok{) \{}
      \CommentTok{\# Get the probabilities for \{A to A\} and \{A to B\}.}
\NormalTok{      prob =}\StringTok{ }\KeywordTok{c}\NormalTok{(}\DecValTok{1} \OperatorTok{{-}}\StringTok{ }\NormalTok{p[}\DecValTok{1}\NormalTok{], p[}\DecValTok{1}\NormalTok{])}
      \CommentTok{\# Randomly sample new state using the probabilities.}
\NormalTok{      x =}\StringTok{ }\KeywordTok{sample}\NormalTok{(}\KeywordTok{c}\NormalTok{(}\StringTok{"A"}\NormalTok{, }\StringTok{"B"}\NormalTok{), }\DecValTok{1}\NormalTok{, }\DataTypeTok{prob =}\NormalTok{ prob)}
\NormalTok{      chain =}\StringTok{ }\KeywordTok{c}\NormalTok{(chain, x)}

\NormalTok{    \} }\ControlFlowTok{else}\NormalTok{ \{}
      \CommentTok{\# Get the probabilities for \{B to A\} and \{B to B\}.}
\NormalTok{      prob =}\StringTok{ }\KeywordTok{c}\NormalTok{(}\DecValTok{1} \OperatorTok{{-}}\StringTok{ }\NormalTok{p[}\DecValTok{2}\NormalTok{], p[}\DecValTok{2}\NormalTok{])}
      \CommentTok{\# Randomly sample new state using the probabilities.}
\NormalTok{      x =}\StringTok{ }\KeywordTok{sample}\NormalTok{(}\KeywordTok{c}\NormalTok{(}\StringTok{"A"}\NormalTok{, }\StringTok{"B"}\NormalTok{), }\DecValTok{1}\NormalTok{, }\DataTypeTok{prob =}\NormalTok{ prob)}
\NormalTok{      chain =}\StringTok{ }\KeywordTok{c}\NormalTok{(chain, x)}
\NormalTok{    \}}
\NormalTok{  \}}

\NormalTok{  chain[}\OperatorTok{{-}}\DecValTok{1}\NormalTok{]}
\NormalTok{\}}
\end{Highlighting}
\end{Shaded}

\hypertarget{round-1}{%
\subsection{Round 1}\label{round-1}}

Use profvis to profile the \texttt{sample\_markov} function. Make sure
the \texttt{n} argument in your call to \texttt{sample\_markov} is large
enough that the function runs for at least 30 seconds. You can use the
default arguments for the other parameters.

The \texttt{sample\_markov} function uses a loop, but doesn't
preallocate the vector that stores the results. Record the total time it
takes for the function to run, and describe evidence you can see in the
profile that failure to preallocate hurts the performance of the
function.

Finally, edit the \texttt{sample\_markov} function so that it
preallocates the vector that stores the results. Use profvis to profile
the edited version (with the same arguments). Record the total time it
takes for the function to run, and comment on how the performance
changed.

\textbf{YOUR ANSWER GOES HERE:}

Time to run: \texttt{54550ms}.

From profvis, it seems that the line to append to \texttt{chain} is
taking an unusally long amount of time and memory. This indicates a lack
of preallocation.

\begin{Shaded}
\begin{Highlighting}[]
\CommentTok{\# Your code to profile the original function goes here:}
\KeywordTok{library}\NormalTok{(profvis)}
\KeywordTok{profvis}\NormalTok{(\{}
\NormalTok{  sample\_markov =}\StringTok{ }\ControlFlowTok{function}\NormalTok{(n, }\DataTypeTok{p =} \KeywordTok{c}\NormalTok{(}\FloatTok{0.6}\NormalTok{, }\FloatTok{0.7}\NormalTok{), }\DataTypeTok{init =} \StringTok{"A"}\NormalTok{) \{}
\NormalTok{  chain =}\StringTok{ }\NormalTok{init}

  \ControlFlowTok{for}\NormalTok{ (i }\ControlFlowTok{in} \DecValTok{1}\OperatorTok{:}\NormalTok{n) \{}
    \CommentTok{\# Check the previous state, chain[i].}
    \ControlFlowTok{if}\NormalTok{ (chain[i] }\OperatorTok{==}\StringTok{ "A"}\NormalTok{) \{}
      \CommentTok{\# Get the probabilities for \{A to A\} and \{A to B\}.}
\NormalTok{      prob =}\StringTok{ }\KeywordTok{c}\NormalTok{(}\DecValTok{1} \OperatorTok{{-}}\StringTok{ }\NormalTok{p[}\DecValTok{1}\NormalTok{], p[}\DecValTok{1}\NormalTok{])}
      \CommentTok{\# Randomly sample new state using the probabilities.}
\NormalTok{      x =}\StringTok{ }\KeywordTok{sample}\NormalTok{(}\KeywordTok{c}\NormalTok{(}\StringTok{"A"}\NormalTok{, }\StringTok{"B"}\NormalTok{), }\DecValTok{1}\NormalTok{, }\DataTypeTok{prob =}\NormalTok{ prob)}
\NormalTok{      chain =}\StringTok{ }\KeywordTok{c}\NormalTok{(chain, x)}

\NormalTok{    \} }\ControlFlowTok{else}\NormalTok{ \{}
      \CommentTok{\# Get the probabilities for \{B to A\} and \{B to B\}.}
\NormalTok{      prob =}\StringTok{ }\KeywordTok{c}\NormalTok{(}\DecValTok{1} \OperatorTok{{-}}\StringTok{ }\NormalTok{p[}\DecValTok{2}\NormalTok{], p[}\DecValTok{2}\NormalTok{])}
      \CommentTok{\# Randomly sample new state using the probabilities.}
\NormalTok{      x =}\StringTok{ }\KeywordTok{sample}\NormalTok{(}\KeywordTok{c}\NormalTok{(}\StringTok{"A"}\NormalTok{, }\StringTok{"B"}\NormalTok{), }\DecValTok{1}\NormalTok{, }\DataTypeTok{prob =}\NormalTok{ prob)}
\NormalTok{      chain =}\StringTok{ }\KeywordTok{c}\NormalTok{(chain, x)}
\NormalTok{    \}}
\NormalTok{  \}}

\NormalTok{  chain[}\OperatorTok{{-}}\DecValTok{1}\NormalTok{]}
\NormalTok{\}}
  \KeywordTok{sample\_markov}\NormalTok{(}\DecValTok{150000}\NormalTok{)}
\NormalTok{\})}
\end{Highlighting}
\end{Shaded}

After preallocating \texttt{chain}, the function took \texttt{850ms} to
run, almost 50 times faster.

\begin{Shaded}
\begin{Highlighting}[]
\CommentTok{\# Your edited function and the code to profile it go here:}
\KeywordTok{profvis}\NormalTok{(\{}
\NormalTok{  sample\_markov =}\StringTok{ }\ControlFlowTok{function}\NormalTok{(n, }\DataTypeTok{p =} \KeywordTok{c}\NormalTok{(}\FloatTok{0.6}\NormalTok{, }\FloatTok{0.7}\NormalTok{), }\DataTypeTok{init =} \StringTok{"A"}\NormalTok{) \{}
\NormalTok{  chain =}\StringTok{ }\KeywordTok{character}\NormalTok{(n)}
\NormalTok{  chain[}\DecValTok{1}\NormalTok{] =}\StringTok{ }\NormalTok{init}

  \ControlFlowTok{for}\NormalTok{ (i }\ControlFlowTok{in} \DecValTok{1}\OperatorTok{:}\NormalTok{n) \{}
    \CommentTok{\# Check the previous state, chain[i].}
    \ControlFlowTok{if}\NormalTok{ (chain[i] }\OperatorTok{==}\StringTok{ "A"}\NormalTok{) \{}
      \CommentTok{\# Get the probabilities for \{A to A\} and \{A to B\}.}
\NormalTok{      prob =}\StringTok{ }\KeywordTok{c}\NormalTok{(}\DecValTok{1} \OperatorTok{{-}}\StringTok{ }\NormalTok{p[}\DecValTok{1}\NormalTok{], p[}\DecValTok{1}\NormalTok{])}
      \CommentTok{\# Randomly sample new state using the probabilities.}
\NormalTok{      x =}\StringTok{ }\KeywordTok{sample}\NormalTok{(}\KeywordTok{c}\NormalTok{(}\StringTok{"A"}\NormalTok{, }\StringTok{"B"}\NormalTok{), }\DecValTok{1}\NormalTok{, }\DataTypeTok{prob =}\NormalTok{ prob)}
\NormalTok{      chain[i }\OperatorTok{+}\StringTok{ }\DecValTok{1}\NormalTok{] =}\StringTok{ }\NormalTok{x}

\NormalTok{    \} }\ControlFlowTok{else}\NormalTok{ \{}
      \CommentTok{\# Get the probabilities for \{B to A\} and \{B to B\}.}
\NormalTok{      prob =}\StringTok{ }\KeywordTok{c}\NormalTok{(}\DecValTok{1} \OperatorTok{{-}}\StringTok{ }\NormalTok{p[}\DecValTok{2}\NormalTok{], p[}\DecValTok{2}\NormalTok{])}
      \CommentTok{\# Randomly sample new state using the probabilities.}
\NormalTok{      x =}\StringTok{ }\KeywordTok{sample}\NormalTok{(}\KeywordTok{c}\NormalTok{(}\StringTok{"A"}\NormalTok{, }\StringTok{"B"}\NormalTok{), }\DecValTok{1}\NormalTok{, }\DataTypeTok{prob =}\NormalTok{ prob)}
\NormalTok{      chain[i }\OperatorTok{+}\StringTok{ }\DecValTok{1}\NormalTok{] =}\StringTok{ }\NormalTok{x}
\NormalTok{    \}}
\NormalTok{  \}}

\NormalTok{  chain[}\OperatorTok{{-}}\DecValTok{1}\NormalTok{]}
\NormalTok{\}}
  \KeywordTok{sample\_markov}\NormalTok{(}\DecValTok{150000}\NormalTok{)}
\NormalTok{\})}
\end{Highlighting}
\end{Shaded}

\hypertarget{round-2}{%
\subsection{Round 2}\label{round-2}}

In your new version of the \texttt{sample\_markov} function, the most
time-consuming lines should be
\texttt{x\ =\ sample(c("A",\ "B"),\ 1,\ prob\ =\ prob)}.

In some cases, using numbers instead of strings can make a computation
significantly faster, and here it is possible to represent the two
states in the chain as 0 and 1 instead of A and B.

Edit your \texttt{sample\_markov} function from round 1 so that it uses
the numbers 0 and 1 to represent the states instead of the strings ``A''
and ``B''. Use profvis to profile the edited version (with the same
arguments). Record the total time it takes for the function to run, and
comment on how the performance changed.

\textbf{YOUR ANSWER GOES HERE:}

After changing the samples to integers, the function took \texttt{720ms}
to run. This is slightly faster than the previous version that used
strings.

\begin{Shaded}
\begin{Highlighting}[]
\CommentTok{\# Your code goes here.}
\KeywordTok{profvis}\NormalTok{(\{}
\NormalTok{  sample\_markov =}\StringTok{ }\ControlFlowTok{function}\NormalTok{(n, }\DataTypeTok{p =} \KeywordTok{c}\NormalTok{(}\FloatTok{0.6}\NormalTok{, }\FloatTok{0.7}\NormalTok{), }\DataTypeTok{init =} \DecValTok{0}\NormalTok{) \{}
\NormalTok{  chain =}\StringTok{ }\KeywordTok{integer}\NormalTok{(n)}
\NormalTok{  chain[}\DecValTok{1}\NormalTok{] =}\StringTok{ }\NormalTok{init}

  \ControlFlowTok{for}\NormalTok{ (i }\ControlFlowTok{in} \DecValTok{1}\OperatorTok{:}\NormalTok{n) \{}
    \CommentTok{\# Check the previous state, chain[i].}
    \ControlFlowTok{if}\NormalTok{ (chain[i] }\OperatorTok{==}\StringTok{ }\DecValTok{0}\NormalTok{) \{}
      \CommentTok{\# Get the probabilities for \{A to A\} and \{A to B\}.}
\NormalTok{      prob =}\StringTok{ }\KeywordTok{c}\NormalTok{(}\DecValTok{1} \OperatorTok{{-}}\StringTok{ }\NormalTok{p[}\DecValTok{1}\NormalTok{], p[}\DecValTok{1}\NormalTok{])}
      \CommentTok{\# Randomly sample new state using the probabilities.}
\NormalTok{      x =}\StringTok{ }\KeywordTok{sample}\NormalTok{(}\KeywordTok{c}\NormalTok{(}\DecValTok{0}\NormalTok{, }\DecValTok{1}\NormalTok{), }\DecValTok{1}\NormalTok{, }\DataTypeTok{prob =}\NormalTok{ prob)}
\NormalTok{      chain[i }\OperatorTok{+}\StringTok{ }\DecValTok{1}\NormalTok{] =}\StringTok{ }\NormalTok{x}

\NormalTok{    \} }\ControlFlowTok{else}\NormalTok{ \{}
      \CommentTok{\# Get the probabilities for \{B to A\} and \{B to B\}.}
\NormalTok{      prob =}\StringTok{ }\KeywordTok{c}\NormalTok{(}\DecValTok{1} \OperatorTok{{-}}\StringTok{ }\NormalTok{p[}\DecValTok{2}\NormalTok{], p[}\DecValTok{2}\NormalTok{])}
      \CommentTok{\# Randomly sample new state using the probabilities.}
\NormalTok{      x =}\StringTok{ }\KeywordTok{sample}\NormalTok{(}\KeywordTok{c}\NormalTok{(}\DecValTok{0}\NormalTok{, }\DecValTok{1}\NormalTok{), }\DecValTok{1}\NormalTok{, }\DataTypeTok{prob =}\NormalTok{ prob)}
\NormalTok{      chain[i }\OperatorTok{+}\StringTok{ }\DecValTok{1}\NormalTok{] =}\StringTok{ }\NormalTok{x}
\NormalTok{    \}}
\NormalTok{  \}}

\NormalTok{  chain[}\OperatorTok{{-}}\DecValTok{1}\NormalTok{]}
\NormalTok{\}}
  \KeywordTok{sample\_markov}\NormalTok{(}\DecValTok{150000}\NormalTok{)}
\NormalTok{\})}
\end{Highlighting}
\end{Shaded}

\hypertarget{round-3}{%
\subsection{Round 3}\label{round-3}}

Switching from strings to numbers to represent the states should give
your function a small but measurable speed boost.

Among experienced R users, it's well known that the general-purpose
\texttt{sample} function is slower than R's functions for sampling from
specific distributions. If you want to sample 0s and 1s, the
\texttt{rbinom} function is significantly faster.

The \texttt{rbinom} function samples values from the ``binomial
distribution'', which models the number of successes in a series of
identical, random trials. Each trial randomly gives a failure (0) or a
success (1). You can sample a single 0 or 1 with \texttt{rbinom} by
setting both the number of observations and the number of trials to 1.
See the documentation for details.

Edit your \texttt{sample\_markov} function from round 2 so that it uses
\texttt{rbinom} to sample the 0s and 1s instead of \texttt{sample}. In
doing so, it should be possible to eliminate the variables \texttt{prob}
and \texttt{x}.

Use profvis to profile the edited version (with the same arguments).
Record the total time it takes for the function to run, and comment on
how the performance changed.

\textbf{YOUR ANSWER GOES HERE:}

The function now takes just \texttt{240ms} to run, much faster than
before.

\begin{Shaded}
\begin{Highlighting}[]
\CommentTok{\# Your code goes here.}
\KeywordTok{profvis}\NormalTok{(\{}
\NormalTok{  sample\_markov =}\StringTok{ }\ControlFlowTok{function}\NormalTok{(n, }\DataTypeTok{p =} \KeywordTok{c}\NormalTok{(}\FloatTok{0.6}\NormalTok{, }\FloatTok{0.7}\NormalTok{), }\DataTypeTok{init =} \DecValTok{0}\NormalTok{) \{}
\NormalTok{  chain =}\StringTok{ }\KeywordTok{integer}\NormalTok{(n)}
\NormalTok{  chain[}\DecValTok{1}\NormalTok{] =}\StringTok{ }\NormalTok{init}

  \ControlFlowTok{for}\NormalTok{ (i }\ControlFlowTok{in} \DecValTok{1}\OperatorTok{:}\NormalTok{n) \{}
    \CommentTok{\# Check the previous state, chain[i].}
    \ControlFlowTok{if}\NormalTok{ (chain[i] }\OperatorTok{==}\StringTok{ }\DecValTok{0}\NormalTok{) \{}
      \CommentTok{\# Randomly sample new state using the probabilities.}
\NormalTok{      chain[i }\OperatorTok{+}\StringTok{ }\DecValTok{1}\NormalTok{] =}\StringTok{ }\KeywordTok{rbinom}\NormalTok{(}\DecValTok{1}\NormalTok{, }\DecValTok{1}\NormalTok{, p[}\DecValTok{1}\NormalTok{])}

\NormalTok{    \} }\ControlFlowTok{else}\NormalTok{ \{}
      \CommentTok{\# Randomly sample new state using the probabilities.}
\NormalTok{      chain[i }\OperatorTok{+}\StringTok{ }\DecValTok{1}\NormalTok{] =}\StringTok{ }\KeywordTok{rbinom}\NormalTok{(}\DecValTok{1}\NormalTok{, }\DecValTok{1}\NormalTok{, p[}\DecValTok{2}\NormalTok{])}
\NormalTok{    \}}
\NormalTok{  \}}

\NormalTok{  chain[}\OperatorTok{{-}}\DecValTok{1}\NormalTok{]}
\NormalTok{\}}
  \KeywordTok{sample\_markov}\NormalTok{(}\DecValTok{150000}\NormalTok{)}
\NormalTok{\})}
\end{Highlighting}
\end{Shaded}

\hypertarget{round-4}{%
\subsection{Round 4}\label{round-4}}

Switching from \texttt{sample} to \texttt{rbinom} should give your
function a signifcant speed boost.

Another strategy for improving performance is to move code out of loops,
either because it doesn't need to be done multiple times, or because it
can be vectorized instead. Generally the code to start with is whatever
takes the longest to run in the loop.

In your \texttt{sample\_markov} function, the calls to \texttt{rbinom}
take up most of the time in the loop. Code that samples random values is
an especially good candidate for vectorization, because sampling is
time-consuming, and all of R's sampling functions are vectorized.

However, we can't just sample \texttt{n} values with \texttt{rbinom}
outside the loop, because the probability of success (1) varies. At each
time step, the probability of transitioning to 1 depends on the state in
the previous time step.

Fortunately, we can solve this problem by using different strategy for
sampling 0s and 1s. Suppose we want the probabilty of getting a 1 to be
\texttt{p}. Then one sampling strategy is:

\begin{enumerate}
\def\labelenumi{\arabic{enumi}.}
\tightlist
\item
  Uniformly sample a decimal value from the interval 0 to 1. In R, we
  can do this with the \texttt{runif} function.
\item
  Compare the step 1 value to \texttt{p}. If it is less than \texttt{p},
  count it as 1; otherwise, count it as 0.
\end{enumerate}

The advantage of this strategy is that the sampling happens in step 1,
but we don't need to know the value of \texttt{p} until step 2.

Edit your \texttt{sample\_markov} function from round 3 so that it uses
the sampling strategy described above to sample 0s and 1s. In
particular, use a vectorized call to \texttt{runif} \emph{outside the
loop} to generate a step 1 value for all \texttt{n} time steps. In the
loop, use \texttt{\textless{}} to compare the appropriate step 1 value
to the appropriate probability from \texttt{p}, and thereby determine
whether the step 2 value is 0 or 1. Note that you can avoid adding an
extra if-statement by using the fact that \texttt{as.integer} converts
\texttt{FALSE} to 0 and \texttt{TRUE} to 1.

Use profvis to profile the edited version (with the same arguments).
Record the total time it takes for the function to run, and comment on
how the performance changed.

Also comment on how the performance of this final version of the
function compares to the version at the beginning of the homework.

\textbf{YOUR ANSWER GOES HERE:}

The function now takes just \texttt{50ms} to run. Compared to the
starter code, this is about 1000 times faster.

\begin{Shaded}
\begin{Highlighting}[]
\CommentTok{\# Your code goes here.}
\KeywordTok{profvis}\NormalTok{(\{}
\NormalTok{  sample\_markov =}\StringTok{ }\ControlFlowTok{function}\NormalTok{(n, }\DataTypeTok{p =} \KeywordTok{c}\NormalTok{(}\FloatTok{0.6}\NormalTok{, }\FloatTok{0.7}\NormalTok{), }\DataTypeTok{init =} \DecValTok{0}\NormalTok{) \{}
\NormalTok{  chain =}\StringTok{ }\KeywordTok{integer}\NormalTok{(n)}
\NormalTok{  chain[}\DecValTok{1}\NormalTok{] =}\StringTok{ }\NormalTok{init}
\NormalTok{  sample =}\StringTok{ }\KeywordTok{runif}\NormalTok{(n, }\DecValTok{0}\NormalTok{, }\DecValTok{1}\NormalTok{)}
  \ControlFlowTok{for}\NormalTok{ (i }\ControlFlowTok{in} \DecValTok{1}\OperatorTok{:}\NormalTok{n) \{}
    \CommentTok{\# Check the previous state, chain[i].}
    \ControlFlowTok{if}\NormalTok{ (chain[i] }\OperatorTok{==}\StringTok{ }\DecValTok{0}\NormalTok{) \{}
      \CommentTok{\# Randomly sample new state using the probabilities.}
\NormalTok{      chain[i }\OperatorTok{+}\StringTok{ }\DecValTok{1}\NormalTok{] =}\StringTok{ }\KeywordTok{as.integer}\NormalTok{(sample[i] }\OperatorTok{\textgreater{}}\StringTok{ }\NormalTok{p[}\DecValTok{1}\NormalTok{])}

\NormalTok{    \} }\ControlFlowTok{else}\NormalTok{ \{}
      \CommentTok{\# Randomly sample new state using the probabilities.}
\NormalTok{      chain[i }\OperatorTok{+}\StringTok{ }\DecValTok{1}\NormalTok{] =}\StringTok{ }\KeywordTok{as.integer}\NormalTok{(sample[i] }\OperatorTok{\textgreater{}}\StringTok{ }\NormalTok{p[}\DecValTok{2}\NormalTok{])}
\NormalTok{    \}}
\NormalTok{  \}}

\NormalTok{  chain[}\OperatorTok{{-}}\DecValTok{1}\NormalTok{]}
\NormalTok{\}}
  \KeywordTok{sample\_markov}\NormalTok{(}\DecValTok{150000}\NormalTok{)}
\NormalTok{\})}
\end{Highlighting}
\end{Shaded}


\end{document}
