% Options for packages loaded elsewhere
\PassOptionsToPackage{unicode}{hyperref}
\PassOptionsToPackage{hyphens}{url}
%
\documentclass[
]{article}
\usepackage{lmodern}
\usepackage{amssymb,amsmath}
\usepackage{ifxetex,ifluatex}
\ifnum 0\ifxetex 1\fi\ifluatex 1\fi=0 % if pdftex
  \usepackage[T1]{fontenc}
  \usepackage[utf8]{inputenc}
  \usepackage{textcomp} % provide euro and other symbols
\else % if luatex or xetex
  \usepackage{unicode-math}
  \defaultfontfeatures{Scale=MatchLowercase}
  \defaultfontfeatures[\rmfamily]{Ligatures=TeX,Scale=1}
\fi
% Use upquote if available, for straight quotes in verbatim environments
\IfFileExists{upquote.sty}{\usepackage{upquote}}{}
\IfFileExists{microtype.sty}{% use microtype if available
  \usepackage[]{microtype}
  \UseMicrotypeSet[protrusion]{basicmath} % disable protrusion for tt fonts
}{}
\makeatletter
\@ifundefined{KOMAClassName}{% if non-KOMA class
  \IfFileExists{parskip.sty}{%
    \usepackage{parskip}
  }{% else
    \setlength{\parindent}{0pt}
    \setlength{\parskip}{6pt plus 2pt minus 1pt}}
}{% if KOMA class
  \KOMAoptions{parskip=half}}
\makeatother
\usepackage{xcolor}
\IfFileExists{xurl.sty}{\usepackage{xurl}}{} % add URL line breaks if available
\IfFileExists{bookmark.sty}{\usepackage{bookmark}}{\usepackage{hyperref}}
\hypersetup{
  pdftitle={STAT 33A Workbook 1},
  pdfauthor={Ming Fong (3035619833)},
  hidelinks,
  pdfcreator={LaTeX via pandoc}}
\urlstyle{same} % disable monospaced font for URLs
\usepackage[margin=1in]{geometry}
\usepackage{color}
\usepackage{fancyvrb}
\newcommand{\VerbBar}{|}
\newcommand{\VERB}{\Verb[commandchars=\\\{\}]}
\DefineVerbatimEnvironment{Highlighting}{Verbatim}{commandchars=\\\{\}}
% Add ',fontsize=\small' for more characters per line
\usepackage{framed}
\definecolor{shadecolor}{RGB}{248,248,248}
\newenvironment{Shaded}{\begin{snugshade}}{\end{snugshade}}
\newcommand{\AlertTok}[1]{\textcolor[rgb]{0.94,0.16,0.16}{#1}}
\newcommand{\AnnotationTok}[1]{\textcolor[rgb]{0.56,0.35,0.01}{\textbf{\textit{#1}}}}
\newcommand{\AttributeTok}[1]{\textcolor[rgb]{0.77,0.63,0.00}{#1}}
\newcommand{\BaseNTok}[1]{\textcolor[rgb]{0.00,0.00,0.81}{#1}}
\newcommand{\BuiltInTok}[1]{#1}
\newcommand{\CharTok}[1]{\textcolor[rgb]{0.31,0.60,0.02}{#1}}
\newcommand{\CommentTok}[1]{\textcolor[rgb]{0.56,0.35,0.01}{\textit{#1}}}
\newcommand{\CommentVarTok}[1]{\textcolor[rgb]{0.56,0.35,0.01}{\textbf{\textit{#1}}}}
\newcommand{\ConstantTok}[1]{\textcolor[rgb]{0.00,0.00,0.00}{#1}}
\newcommand{\ControlFlowTok}[1]{\textcolor[rgb]{0.13,0.29,0.53}{\textbf{#1}}}
\newcommand{\DataTypeTok}[1]{\textcolor[rgb]{0.13,0.29,0.53}{#1}}
\newcommand{\DecValTok}[1]{\textcolor[rgb]{0.00,0.00,0.81}{#1}}
\newcommand{\DocumentationTok}[1]{\textcolor[rgb]{0.56,0.35,0.01}{\textbf{\textit{#1}}}}
\newcommand{\ErrorTok}[1]{\textcolor[rgb]{0.64,0.00,0.00}{\textbf{#1}}}
\newcommand{\ExtensionTok}[1]{#1}
\newcommand{\FloatTok}[1]{\textcolor[rgb]{0.00,0.00,0.81}{#1}}
\newcommand{\FunctionTok}[1]{\textcolor[rgb]{0.00,0.00,0.00}{#1}}
\newcommand{\ImportTok}[1]{#1}
\newcommand{\InformationTok}[1]{\textcolor[rgb]{0.56,0.35,0.01}{\textbf{\textit{#1}}}}
\newcommand{\KeywordTok}[1]{\textcolor[rgb]{0.13,0.29,0.53}{\textbf{#1}}}
\newcommand{\NormalTok}[1]{#1}
\newcommand{\OperatorTok}[1]{\textcolor[rgb]{0.81,0.36,0.00}{\textbf{#1}}}
\newcommand{\OtherTok}[1]{\textcolor[rgb]{0.56,0.35,0.01}{#1}}
\newcommand{\PreprocessorTok}[1]{\textcolor[rgb]{0.56,0.35,0.01}{\textit{#1}}}
\newcommand{\RegionMarkerTok}[1]{#1}
\newcommand{\SpecialCharTok}[1]{\textcolor[rgb]{0.00,0.00,0.00}{#1}}
\newcommand{\SpecialStringTok}[1]{\textcolor[rgb]{0.31,0.60,0.02}{#1}}
\newcommand{\StringTok}[1]{\textcolor[rgb]{0.31,0.60,0.02}{#1}}
\newcommand{\VariableTok}[1]{\textcolor[rgb]{0.00,0.00,0.00}{#1}}
\newcommand{\VerbatimStringTok}[1]{\textcolor[rgb]{0.31,0.60,0.02}{#1}}
\newcommand{\WarningTok}[1]{\textcolor[rgb]{0.56,0.35,0.01}{\textbf{\textit{#1}}}}
\usepackage{graphicx}
\makeatletter
\def\maxwidth{\ifdim\Gin@nat@width>\linewidth\linewidth\else\Gin@nat@width\fi}
\def\maxheight{\ifdim\Gin@nat@height>\textheight\textheight\else\Gin@nat@height\fi}
\makeatother
% Scale images if necessary, so that they will not overflow the page
% margins by default, and it is still possible to overwrite the defaults
% using explicit options in \includegraphics[width, height, ...]{}
\setkeys{Gin}{width=\maxwidth,height=\maxheight,keepaspectratio}
% Set default figure placement to htbp
\makeatletter
\def\fps@figure{htbp}
\makeatother
\setlength{\emergencystretch}{3em} % prevent overfull lines
\providecommand{\tightlist}{%
  \setlength{\itemsep}{0pt}\setlength{\parskip}{0pt}}
\setcounter{secnumdepth}{-\maxdimen} % remove section numbering
\ifluatex
  \usepackage{selnolig}  % disable illegal ligatures
\fi

\title{STAT 33A Workbook 1}
\author{Ming Fong (3035619833)}
\date{Sep 3, 2020}

\begin{document}
\maketitle

This workbook is due \textbf{Sep 3, 2020} by 11:59pm PT.

The workbook is organized into sections that correspond to the lecture
videos for the week. Watch a video, then do the corresponding exercises
\emph{before} moving on to the next video.

Workbooks are graded for completeness, so as long as you make a clear
effort to solve each problem, you'll get full credit. That said, make
sure you understand the concepts here, because they're likely to
reappear in homeworks, quizzes, and later lectures.

As you work, write your answers in this notebook. The first lecture
video this week will teach you more about how the notebook works.

Please do not delete the exercises already in this notebook, because it
may interfere with our grading tools.

\hypertarget{packages-r-notebooks}{%
\section{Packages \& R Notebooks}\label{packages-r-notebooks}}

Watch the ``Packages \& Notebooks'' lecture video.

\hypertarget{exercise-1}{%
\subsection{Exercise 1}\label{exercise-1}}

\emph{You don't need to write anything in the notebook for this
exercise.}

Install the \texttt{rmarkdown} and \texttt{tinytex} packages.

Confirm that your setup works by knitting this file to PDF.

If you run into any trouble, ask on Piazza or in office hours as soon as
possible. You'll need to be able to knit to turn in the assignment.

\emph{note: I used my existing MikTex distribution}

\hypertarget{markdown}{%
\section{Markdown}\label{markdown}}

Watch the ``Markdown'' lecture video.

For the remaining exercises, answer questions with complete sentences,
and put code in code chunks. You can make as many new code chunks as you
like.

In the notebook, you can run the line of code where the cursor is by
pressing \texttt{Ctrl} + \texttt{Enter} on Windows or \texttt{Cmd} +
\texttt{Enter} on Mac OS X. You can run an entire code chunk by clicking
on the green arrow in the upper right corner of the code chunk.

\hypertarget{exercise-2}{%
\subsection{Exercise 2}\label{exercise-2}}

Write 3-5 sentences introducing yourself. Try out some of the Markdown
formatting: italics, bold, lists, quotes, code chunks, and links.

After you finish, knit the notebook to check that your formatting was
applied correctly.

YOUR ANSWER GOES HERE:

Hi! My name is \textbf{Ming Fong} and I'm a first year at Berkeley. Here
are some links to personal websites:

\begin{itemize}
\tightlist
\item
  \href{https://github.com/evilpegasus}{GitHub}
\item
  \href{https://www.linkedin.com/in/mingfong/}{LinkedIn}
\item
  \href{http://mingyfong.github.io/}{Personal Website}
\end{itemize}

\begin{enumerate}
\def\labelenumi{\arabic{enumi}.}
\tightlist
\item
  One fish
\item
  Two fish
\item
  Red fish
\item
  Blue fish
\end{enumerate}

\begin{Shaded}
\begin{Highlighting}[]
\NormalTok{x =}\StringTok{ }\KeywordTok{c}\NormalTok{(}\DecValTok{888}\NormalTok{, }\DecValTok{999}\NormalTok{, }\DecValTok{1234}\NormalTok{)}
\NormalTok{x =}\StringTok{ }\KeywordTok{c}\NormalTok{(x, }\DecValTok{111}\NormalTok{)}
\NormalTok{x}
\end{Highlighting}
\end{Shaded}

\begin{verbatim}
## [1]  888  999 1234  111
\end{verbatim}

\begin{quote}
I use single character variable names so the compiler runs faster.
\end{quote}

\hypertarget{exercise-3}{%
\subsection{Exercise 3}\label{exercise-3}}

Use Markdown to make:

\begin{itemize}
\tightlist
\item
  A word that's both bold and italic
\item
  A link that's italic
\item
  A nested list (a list with indented subelements)
\end{itemize}

Try to figure these out by experimentation before asking or searching
online.

YOUR ANSWER GOES HERE:

\textbf{\emph{This is bold and italic}}

\emph{\href{http://mingyfong.github.io/}{Personal Website Link}}

\begin{itemize}
\tightlist
\item
  One

  \begin{itemize}
  \tightlist
  \item
    One.one
  \item
    One.two
  \end{itemize}
\item
  Two
\end{itemize}

\hypertarget{vectors}{%
\section{Vectors}\label{vectors}}

Watch the ``Vectors'' lecture video.

\hypertarget{exercise-4}{%
\subsection{Exercise 4}\label{exercise-4}}

Another way to create vectors is with the \texttt{rep()} function. The
\texttt{rep()} function creates a vector by replicating a value or
vector of values.

\begin{enumerate}
\def\labelenumi{\arabic{enumi}.}
\item
  The first parameter of \texttt{rep()} is the thing to replicate. The
  second parameter, \texttt{times}, is the number of times to to
  replicate. Use \texttt{rep()} to make a vector with 10 elements, all
  equal to 78.
\item
  What happens if you pass a vector as the first argument to
  \texttt{rep()}? Give some examples.
\item
  Skim the help file \texttt{?rep}. What happens if you pass a vector as
  the second argument to \texttt{rep()}? The help file might seem a bit
  cryptic, so you'll also need to experiment. Give some examples.
\end{enumerate}

YOUR ANSWER GOES HERE:

\begin{enumerate}
\def\labelenumi{\arabic{enumi}.}
\tightlist
\item
\end{enumerate}

\begin{Shaded}
\begin{Highlighting}[]
\NormalTok{x =}\StringTok{ }\KeywordTok{rep}\NormalTok{(}\DecValTok{78}\NormalTok{, }\DecValTok{10}\NormalTok{)}
\NormalTok{x}
\end{Highlighting}
\end{Shaded}

\begin{verbatim}
##  [1] 78 78 78 78 78 78 78 78 78 78
\end{verbatim}

\begin{enumerate}
\def\labelenumi{\arabic{enumi}.}
\setcounter{enumi}{1}
\tightlist
\item
  It will repeat the argument vector \texttt{times} times.
\end{enumerate}

\begin{Shaded}
\begin{Highlighting}[]
\NormalTok{y =}\StringTok{ }\KeywordTok{c}\NormalTok{(}\DecValTok{1}\NormalTok{, }\DecValTok{2}\NormalTok{, }\DecValTok{3}\NormalTok{)}
\NormalTok{y}
\end{Highlighting}
\end{Shaded}

\begin{verbatim}
## [1] 1 2 3
\end{verbatim}

\begin{Shaded}
\begin{Highlighting}[]
\NormalTok{z =}\StringTok{ }\KeywordTok{rep}\NormalTok{(y, }\DecValTok{5}\NormalTok{)}
\NormalTok{z}
\end{Highlighting}
\end{Shaded}

\begin{verbatim}
##  [1] 1 2 3 1 2 3 1 2 3 1 2 3 1 2 3
\end{verbatim}

\begin{enumerate}
\def\labelenumi{\arabic{enumi}.}
\setcounter{enumi}{2}
\tightlist
\item
  When the first argument and the \texttt{times} argument have the same
  length, the new vector will repeat the first element
  \texttt{times{[}1{]}} times, the second \texttt{times{[}2{]}} times,
  and the nth element \texttt{times{[}n{]}} times.
\end{enumerate}

\begin{Shaded}
\begin{Highlighting}[]
\NormalTok{?rep}
\end{Highlighting}
\end{Shaded}

\begin{Shaded}
\begin{Highlighting}[]
\NormalTok{x =}\StringTok{ }\KeywordTok{c}\NormalTok{(}\DecValTok{1}\NormalTok{, }\DecValTok{2}\NormalTok{, }\DecValTok{3}\NormalTok{)}
\NormalTok{x}
\end{Highlighting}
\end{Shaded}

\begin{verbatim}
## [1] 1 2 3
\end{verbatim}

\begin{Shaded}
\begin{Highlighting}[]
\NormalTok{y =}\StringTok{ }\KeywordTok{rep}\NormalTok{(}\KeywordTok{c}\NormalTok{(}\DecValTok{1}\NormalTok{, }\DecValTok{2}\NormalTok{, }\DecValTok{3}\NormalTok{), x)}
\NormalTok{y}
\end{Highlighting}
\end{Shaded}

\begin{verbatim}
## [1] 1 2 2 3 3 3
\end{verbatim}

\hypertarget{exercise-5}{%
\subsection{Exercise 5}\label{exercise-5}}

Yet another way to create vectors is with the \texttt{seq()} function.
The \texttt{seq()} function creates a vector that contains a sequence of
numbers.

Skim the help file \texttt{?seq}. Give some examples of creating vectors
with the \texttt{seq()} function.

YOUR ANSWER GOES HERE:

\begin{Shaded}
\begin{Highlighting}[]
\NormalTok{?seq}
\end{Highlighting}
\end{Shaded}

\begin{Shaded}
\begin{Highlighting}[]
\NormalTok{x =}\StringTok{ }\KeywordTok{seq}\NormalTok{(}\DecValTok{2}\NormalTok{, }\DecValTok{10}\NormalTok{, }\DecValTok{2}\NormalTok{)}
\NormalTok{x}
\end{Highlighting}
\end{Shaded}

\begin{verbatim}
## [1]  2  4  6  8 10
\end{verbatim}

\begin{Shaded}
\begin{Highlighting}[]
\NormalTok{y =}\StringTok{ }\KeywordTok{seq}\NormalTok{(}\DecValTok{2}\NormalTok{, }\DecValTok{100}\NormalTok{, }\DecValTok{2}\NormalTok{)}
\NormalTok{y}
\end{Highlighting}
\end{Shaded}

\begin{verbatim}
##  [1]   2   4   6   8  10  12  14  16  18  20  22  24  26  28  30  32  34  36  38
## [20]  40  42  44  46  48  50  52  54  56  58  60  62  64  66  68  70  72  74  76
## [39]  78  80  82  84  86  88  90  92  94  96  98 100
\end{verbatim}

\hypertarget{subsets-of-vectors}{%
\section{Subsets of Vectors}\label{subsets-of-vectors}}

Watch the ``Subsets of Vectors'' lecture video.

\hypertarget{exercise-6}{%
\subsection{Exercise 6}\label{exercise-6}}

Try out each of the following:

\begin{itemize}
\tightlist
\item
  Getting a single element of a vector
\item
  Reassigning an element of a vector
\item
  Getting multiple elements of a vector
\item
  Getting multiple elements of a vector using negative indices
\end{itemize}

YOUR ANSWER GOES HERE:

\begin{Shaded}
\begin{Highlighting}[]
\NormalTok{x =}\StringTok{ }\KeywordTok{seq}\NormalTok{(}\DecValTok{2}\NormalTok{, }\DecValTok{20}\NormalTok{, }\DecValTok{2}\NormalTok{)}
\NormalTok{x}
\end{Highlighting}
\end{Shaded}

\begin{verbatim}
##  [1]  2  4  6  8 10 12 14 16 18 20
\end{verbatim}

\begin{Shaded}
\begin{Highlighting}[]
\NormalTok{x[}\DecValTok{5}\NormalTok{]}
\end{Highlighting}
\end{Shaded}

\begin{verbatim}
## [1] 10
\end{verbatim}

\begin{Shaded}
\begin{Highlighting}[]
\NormalTok{x[}\KeywordTok{c}\NormalTok{(}\DecValTok{1}\NormalTok{, }\DecValTok{2}\NormalTok{, }\DecValTok{3}\NormalTok{)]}
\end{Highlighting}
\end{Shaded}

\begin{verbatim}
## [1] 2 4 6
\end{verbatim}

\begin{Shaded}
\begin{Highlighting}[]
\NormalTok{x[}\KeywordTok{c}\NormalTok{(}\OperatorTok{{-}}\DecValTok{1}\NormalTok{,}\OperatorTok{{-}}\DecValTok{2}\NormalTok{,}\OperatorTok{{-}}\DecValTok{3}\NormalTok{)]}
\end{Highlighting}
\end{Shaded}

\begin{verbatim}
## [1]  8 10 12 14 16 18 20
\end{verbatim}

\hypertarget{exercise-7}{%
\subsection{Exercise 7}\label{exercise-7}}

Consider this code:

\begin{Shaded}
\begin{Highlighting}[]
\NormalTok{x =}\StringTok{ }\KeywordTok{seq}\NormalTok{(}\DecValTok{1}\NormalTok{, }\DecValTok{5}\NormalTok{) }\OperatorTok{*}\StringTok{ }\DecValTok{10}
\NormalTok{x[}\KeywordTok{c}\NormalTok{(}\DecValTok{3}\NormalTok{, }\DecValTok{4}\NormalTok{, }\DecValTok{5}\NormalTok{)][}\DecValTok{1}\NormalTok{]}
\end{Highlighting}
\end{Shaded}

\begin{verbatim}
## [1] 30
\end{verbatim}

Try running the code to see what the result is. Then explain what's
happening in the second line (it may be useful to explain what
\texttt{x{[}c(3,\ 4,\ 5){]}} does first).

YOUR ANSWER GOES HERE:

\begin{enumerate}
\def\labelenumi{\arabic{enumi}.}
\tightlist
\item
  \texttt{x\ =\ seq(1,\ 5)\ *\ 10} creates a sequence of 1 to 5 and
  multiplies each value by 10: 10, 20, 30\ldots{}
\item
  \texttt{x{[}c(3,\ 4,\ 5){]}} grabs the third, fourth, and fifth
  element from x: 30, 40, 50.
\item
  \texttt{x{[}c(3,\ 4,\ 5){]}{[}1{]}} the extra \texttt{{[}1{]}} grabs
  the first element after the above step: 30.
\end{enumerate}

\hypertarget{exercise-8}{%
\subsection{Exercise 8}\label{exercise-8}}

The \texttt{sort()} function sorts the elements of a vector. For
instance:

\begin{Shaded}
\begin{Highlighting}[]
\NormalTok{x =}\StringTok{ }\KeywordTok{c}\NormalTok{(}\DecValTok{4}\NormalTok{, }\DecValTok{5}\NormalTok{, }\DecValTok{1}\NormalTok{)}
\KeywordTok{sort}\NormalTok{(x)}
\end{Highlighting}
\end{Shaded}

\begin{verbatim}
## [1] 1 4 5
\end{verbatim}

Another way to sort vectors is by using the \texttt{order()} function.
The order function returns the \emph{indices} for the sorted values
rather than the values themselves:

\begin{Shaded}
\begin{Highlighting}[]
\NormalTok{x =}\StringTok{ }\KeywordTok{c}\NormalTok{(}\DecValTok{4}\NormalTok{, }\DecValTok{5}\NormalTok{, }\DecValTok{1}\NormalTok{)}
\KeywordTok{order}\NormalTok{(x)}
\end{Highlighting}
\end{Shaded}

\begin{verbatim}
## [1] 3 1 2
\end{verbatim}

These can be used to sort the vector by subsetting:

\begin{Shaded}
\begin{Highlighting}[]
\NormalTok{x[}\KeywordTok{order}\NormalTok{(x)]}
\end{Highlighting}
\end{Shaded}

\begin{verbatim}
## [1] 1 4 5
\end{verbatim}

The key advantage of \texttt{order()} over \texttt{sort()} is that it
can also be used to sort one vector based on another, as long as the two
vectors have the same length.

Create two vectors with the same length, and use one to sort the
elements of the other.

YOUR ANSWER GOES HERE:

\begin{Shaded}
\begin{Highlighting}[]
\NormalTok{x =}\StringTok{ }\KeywordTok{seq}\NormalTok{(}\DecValTok{2}\NormalTok{, }\DecValTok{10}\NormalTok{, }\DecValTok{2}\NormalTok{)}
\NormalTok{x}
\end{Highlighting}
\end{Shaded}

\begin{verbatim}
## [1]  2  4  6  8 10
\end{verbatim}

\begin{Shaded}
\begin{Highlighting}[]
\NormalTok{y =}\StringTok{ }\KeywordTok{c}\NormalTok{(}\DecValTok{1234}\NormalTok{, }\DecValTok{100}\NormalTok{, }\DecValTok{50}\NormalTok{, }\DecValTok{23}\NormalTok{, }\DecValTok{5}\NormalTok{)}
\NormalTok{y}
\end{Highlighting}
\end{Shaded}

\begin{verbatim}
## [1] 1234  100   50   23    5
\end{verbatim}

\begin{Shaded}
\begin{Highlighting}[]
\CommentTok{\# sort x backwards because y is in reverse order}
\NormalTok{x[}\KeywordTok{order}\NormalTok{(y)]}
\end{Highlighting}
\end{Shaded}

\begin{verbatim}
## [1] 10  8  6  4  2
\end{verbatim}

\hypertarget{submitting-your-work}{%
\section{Submitting Your Work}\label{submitting-your-work}}

Congratulations, you made it through the first workbook!

You need to submit your work in two places:

\begin{itemize}
\tightlist
\item
  Submit this Rmd file with your edits on bCourses.
\item
  Knit and submit the generated PDF file on Gradescope.
\end{itemize}

\end{document}
